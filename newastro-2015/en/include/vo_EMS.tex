\section{The Virtual Observatory Registry}
\label{sec:registry}

The description of the work done by each IVOA member
in the previous section is an accurate political, organizational 
and geographical division. However, in practice the services
are sparsely distributed between universities, observatories
and institutes around the world. In fact, some of them 
are part of more than one country-VO. 
This makes very complex to track the available services, 
and to search in the VO as a whole. Fortunately, 
IVOA has developed a \emph{registry} standard to 
exploit the interoperability of the VO nodes.
Moreover, 

This section explains very briefly the registry standard,
and use the existing registry services for obtaining VO 
statistics mainly for two purposes:
\begin{itemize}
\item showing the advantages of the interoperability in the VO,
\item and to survey the types of services, data and wavebands that
the VO supports as a whole.
\end{itemize}
Also, as the final goal of the Virtual Observatory is to be a science enabler,
we list in this section some recent publications that were possible due to
VO technologies.

\subsection{The Registry}

The IVOA registry allows users and applications to discover data collections,
metadata and services in the resource layer for a particular scientific
problem. In fact, a resource it can be virtually any information that might be
relevant to astronomers, from abstract concepts to images and catalogs. 
The specification is based on a general, distributed model composed of
so-called searchable and publishing registries:
\begin{itemize}
\item A \emph{searchable} registry offers a list of resources through searching the 
metadata contained in the records for a given selection criterion.
This allows aggregating several resources from different providers 
without the need of visiting and searching in each provider's site just to
know if the data is useful for the scientific objectives.
The process of searching through all the VO sites is usually called harvesting.
\item A \emph{publishing} registry allows offering resources in a standard way
in order to be harvested by a searchable registry. This is a much more
simpler registry than the previous, where the only objective is to 
inform others which are the resources that the registry offers.
\end{itemize}

Even though there are some local searchable registries in some VOs, the
main idea of these ones is to be global: each VO can have one or more
publishing registries from which a searchable registry feeds to give
a global answer for a selection criterion. For the following sections
we will use the VO-Paris
registry\footnote{\texttt{http://voparis-registry.obspm.fr/vo/ivoa/1/voresources/search}}
because is one of the most complete and functional global searchable registries.
% description of the registry
% the registry that we will use

\subsection{VO Resources by Type of Service}

We have selected the four most known IVOA services standards described in Section~\ref{sec:popservices}, namely
\verb+SCS+, \verb+SIA+, \verb+SSA+ and \verb+TAP+. We've queried the Registry for the number of resources 
that are available for each one of them. Please note that a resource may correspond to a whole database, where
each one might have from one to a huge number of entries depending on the service. 

\begin{table}
\begin{center}
\begin{tabular}{|l|l|l|l|}
\hline
\textbf{Service} & \textbf{Resources} & \textbf{Publishers} & \textbf{No
Information} \\
\hline
SCS  &  10463 & 165 & 4.7 \% \\
SIA  &  175 & 33 & 0.6 \% \\
SSA  &  48  & 29 & 8.3 \% \\
TAP  &  33  & 14 & 18.2 \% \\
\hline
\end{tabular}
\caption{Resources and publishers for each IVOA service. The last column
presents the percentage of resources that do not inform a publisher.}
\label{tab:serviceRanking}
\end{center}
\end{table}

Table~\ref{tab:serviceRanking} shows how many resources are available for the
selected services, including the number of publishers that offer the service
and the percentage of resources that do not inform a publisher\footnote{The
publisher field is optional in the registry standard}. Clearly, the simple
cone search (SCS) service is the most popular service, yet we have to note
here that around the 85\% of the resources are concentrated in one publisher\footnote{Which corresponds to the Centre de Données astronomiques de Strasbourg},
and in fact the 90\% of them are concentrated in only two. Nevertheless, the 
10\% of the total SCS resources is still one order larger than its follower, 
the SIA standard. This result arises naturally from the fact that every
observation has an angular position and field of view, so a SCS can be 
developed for almost every astronomical data set.

As explained above, the publishers do not correspond to country-VOs, but 
to the names of the institutions that host the services, so that is why the SCS
service has 165 publishers instead of only a few country-VOs. In fact, well
organized VOs still include their organizational branches in the publisher field, such as VO-Paris that has at least 3 branches.

The relatively low number of services of SSA and TAP can be explained as follows:
\begin{itemize}
\item SSA is a very specialized service, that only allows 1-D spectroscopy data that commonly 
needs on-line transformation to the SSA format. Also, the spectroscopy instrumentation is
scarce compared to imaging instruments in the observatories. Also, is very likely that for each
SIA or SSA resource, a SCS service exists also for the same database.
\item TAP is a very generic service, because almost any public table can be queried. 
The natural implementation is to provide only one TAP service for all the databases
in the same institution (as our results confirm), because TAP is a more a technical 
service than a scientific resource. 
\end{itemize} 

\subsection{VO Resources in the Electromagnetic Spectrum}

One of the interesting queries that can be done using the registry is
to which wavelength a resource belongs to. This not only simplifies
searching for appropriate resources, but it also give us the 
\emph{big picture} of the services coverage. As expected, 
the resources are scattered over the whole spectrum, where
only TAP fails to offer services in the Gamma-ray, UV, extreme UV, and Millimeter
wavebands. All the other shows a fair coverage of the spectrum, considering
that only the 3\% of the resources does not inform wavelength band. 


\begin{table}
\begin{center}
\scriptsize
\begin{tabular}{|l|c|c|c|c|c|c|c|c|}
\hline
\textbf{VO} & GR & XR & EU & UV & OP & IR & ML & RD \\
\hline
\hline
  ESO - Science Archive Facility  & & & & X & X & X & & \\
\hline
  Space Telescope Science Institute & & & & X & & & & \\
\hline
  VO-Paris - LUTH & X & & & & & & & \\
\hline
  INTA/LAEFF & & & & & X & & & \\
\hline
  WFAU, Institute for Astronomy & & & & & X & & & \\
\hline
  MAST & & & X & X & X & & & \\
\hline
  Institut d'Astrophysique de Paris & & & X & & & & & \\
\hline
  Space Telescope Science Institute & & & & X & X & X & & \\
\hline
  VO-Paris - IMCCE & & & & & X & X & & \\
\hline
  WFAU & X & X & & & & & & \\
\hline
  JVO & & & & & X & & & \\
\hline
  Instituto de Astrofsica de Andaluca & & & & X & X & X & & \\
\hline
  VO-Paris - GEPI  & & & & & X & & & \\
\hline
  Lab. d'Astrophysique de Marseille & & & & & X & & & \\
\hline
  Instituto de Astrofsica de Andaluca & & X & & X & X & X & & \\
\hline
  VO-Paris - LESIA & & & & X & X & & & \\
\hline
  GAVO DC team  & & & & X & X & X & & \\
\hline
  NASA/IPAC Extragalactic Database & X & X & X & X & X & X & X & X \\
\hline
  LUPM & & & & X & X & & & \\
\hline
  CRAL-Observatoire de Lyon & & & & & X & & & \\
\hline
  Ivo://Bass2000 & & & & & X & & & \\
\hline
\end{tabular}
\caption{Publishers in the Electromagnetic Spectrum for SSA resources.
GR = Gamma-rays, XR = X-rays, EU = Extreme Ultraviolet, UV = Ultraviolet, OP
= Optical, IR = Infrared, ML = Millimeter, RD = Radio.}
\label{tab:SSAEMS}
\end{center}
\end{table}

In general, and as expected, the resources are concentrated around the optical waveband as can be
seen in the example Table~\ref{tab:SSAEMS}. This table shows the number of SSA resources by publisher,
including the waveband where they inform that the data belongs to. SSA was selected as an example due to
space reasons, but the same result can be obtained for SCS, SIA and TAP. In particular, we can observe that
millimeter and radio resources are scarce, which is strange in the sense that most of the instruments in these
wavebands can perform spectroscopy. This is because the SSA service is very specific to certain type of
data, so efforts are been made to extend this service to support spectral data cubes as the ones that
the ALMA observatory is generating. 

\subsection{VO Resources by Scientific Subject}

Each resource is commonly focused on certain science subjects, so a field with this name was included
as a suggestion in the registry standard. Impressively, this suggestion has been strongly followed by the VO.
In fact, only less than 0.4\% of all the resources (including the four standards) do not inform a subject. 

\begin{table}
\begin{center}
\begin{tabular}{|l|c|}
\hline
\textbf{Subject} & \textbf{Resources} \\
\hline
Galaxies  &  1994  \\
Stars  &  1566  \\
Photometry  &  1400  \\
Photometry:wide-band  &  1176  \\
Redshifts  &  1018  \\
Stars:variable  &  985  \\
Velocities  &  919  \\
Open\_Clusters  &  787  \\
Spectroscopy  &  715  \\
\hline
\end{tabular}
\caption{Most popular subjects, considering all the resources (12357) from the four services.}
\label{tab:resSub}
\end{center}
\end{table}

In Table~\ref{tab:resSub}, the scientific subjects with more resources are presented. Besides the ones
presented in the table there are 837 other subjects, where one resource can have many subjects. 
The most popular subjects are obviously very general categorizations (like stars and galaxies), but
more specific ones, such as variable stars or open clusters, can be very useful tags for selecting
resources. An interesting result here is that 715 resources declare to be spectroscopy data\footnote{Moreover, this number does not include tags that are implicitly spectroscopy data such as ``Galaxies:spectra''.}, but we
found only 48 SSA resources. This means that most of the spectroscopy data is offered through other
services (like SCS), but not using SSA. There are two complementary reasons for this issue: the first
is that SSA is a very specific service as explained above, and the second is that the SSA service requires
much more effort and metadata than SCS, so the VOs may not consider a priority to mount this service.
In any case, an improved spectral access standard is being developed.

Please note that the queries performed in the past sections can be easily combined. For example,
the number of resources that are tagged as Galaxies that have data in optical band are 1485, while the ones in
the infrared band are 493 and in the UV band are only 127 (again this does not sum 1994 because a resource can have data
in more than one band). 

%TODO:

% \textbf{Others} &  837    \\
%total amount:  12357

%Most of virtual observatories usually contribute to development of tools for
%for national astronomical instruments. These tools can be new source codes for
%the exploitation of the capability of some specific instrument, facilitation or
%improvement of the queries, reading, writing, conversion of the astronomical
%data of files like VOTable, ASCII and FITS formats, among other. 
%This means that each VO concentrates in providing services for specific
%data types, which in astronomy are highly correlated with the wavelengths values
%on which their work. This section surveys the ranges for which 
%each virtual observatory is developing their tools, 
%
%\begin{landscape}
%\begin{figure}%[h]
%\begin{center}
%\begin{tikzpicture}
%\pgfmathsetmacro{\lineLenght}{15}
%\pgfmathsetmacro{\lineDivNum}{17}
%\pgfmathsetmacro{\markPlace}{\lineLenght/\lineDivNum}
%\def\limitsList{5,8,10.3,10.6,13,15}
%\def\regionArray{{2.5,(5+8)/2,(8+10.3)/2,(10.3+10.6)/2,(10.6+13)/2,(13+15)/2,
%                  (15+17)/2}}
%\foreach \x in {0,1} {
%  \draw[<->,thick] (0,\x) -- (\lineLenght,\x);
%}
%\foreach \x in \limitsList {
%  \draw[thick] (\x*\markPlace,0.8) -- (\x*\markPlace,1.2);
%}
%\foreach \x in {1,2,...,16} {
%  \pgfmathtruncatemacro{\b}{\x-2*\x+4} 
%  \draw[thick] (\x*\markPlace,-0.2) -- (\x*\markPlace,0.2)
%  node[pos=-0.6]{$10^{\b}$};
%}
%\node [anchor=north east,align=center] at (0,0) {Wavelenght\\(in meters)};
%\node [anchor=south east,align=center] at (0,1) {Radiation\\type};
%\node [above] at (\regionArray[0]*\markPlace,1.5) {Radio};
%\node [above] at (\regionArray[1]*\markPlace,1.5) {Microwave};
%\node [above] at (\regionArray[2]*\markPlace,1.5) {Infrared};
%\node [above] at (\regionArray[3]*\markPlace,2) {Visible};
%\node [above] at (\regionArray[4]*\markPlace,1.5) {Ultraviolet};
%\node [above] at (\regionArray[5]*\markPlace,1.5) {X-ray};
%\node [above] at (\regionArray[6]*\markPlace,1.5) {Gamma ray};
%
%% Spectrum value or range for each instrument where some virtual observatory
%% contributes
%
%\pgfmathsetmacro{\factor}{\markPlace/2}
%% ALMA
%\draw[<->,thick] (7.7*\markPlace,-4) -- (8.6*\markPlace,-4);
%\pgfmathparse{multiply(8.6-7.7,\factor)}
%\node at (\pgfmathresult+7.7*\markPlace,-3.5) {ALMA};
%\node at (\pgfmathresult+7.7*\markPlace,-4.5) {\textbf{ChiVO}};
%\node at (\pgfmathresult+7.7*\markPlace,-5.0) {\textbf{JVO}};
% BAO
%\draw[<->,thick] (3.58*\markPlace,-2) -- (3.88*\markPlace,-2);
%\pgfmathparse{multiply(3.88-3.58,\factor)}
%\node at (\pgfmathresult+3.58*\markPlace,-1.5) {BAO};
%\node at (\pgfmathresult+3.58*\markPlace,-2.5) {\textbf{ArVO}};
%% Chandra X-ray Telescope 
%\draw[<->,thick] (13*\markPlace,-2) -- (15*\markPlace,-2);
%\pgfmathparse{multiply(15-13,\factor)}
%\node at (\pgfmathresult+13*\markPlace,-1.5) {Chandra X-ray Telescope};
%\node at (\pgfmathresult+13*\markPlace,-2.5) {\textbf{VAO}};
%% HASTA 
%\node at (3.4312*\markPlace,-4) [inner sep=1pt,circle,draw] {};
%\node at (3.4312*\markPlace,-3.5) {HASTA};
%\node at (3.4312*\markPlace,-4.5) {\textbf{NOVA}};
%% HCT
%\draw[<->,thick] (8*\markPlace,-6) -- (10.3*\markPlace,-6);
%\pgfmathparse{multiply(10.3-8,\factor)}
%\node at (\pgfmathresult+8*\markPlace,-5.5) {HCT};
%\node at (\pgfmathresult+8*\markPlace,-6.5) {\textbf{VOI}};
%% HST  
%\draw[<->,thick] (8*\markPlace,-10) -- (13*\markPlace,-10);
%\pgfmathparse{multiply(13-8,\factor)}
%\node at (\pgfmathresult+8*\markPlace,-9.5) {HST};
%\node at (\pgfmathresult+8*\markPlace,-10.5) {\textbf{VAO, VObs.it}};
%% SOAR 
%\draw[<->,thick] (9*\markPlace,-4) -- (10.68*\markPlace,-4);
%\pgfmathparse{multiply(10.68-9,\factor)}
%\node at (\pgfmathresult+9*\markPlace,-3.5) {SOAR};
%\node at (\pgfmathresult+9*\markPlace,-4.5) {\textbf{BRAVO}};
%% SST
%\draw[<->,thick] (8.82*\markPlace,-2) -- (9.7*\markPlace,-2);
%\pgfmathparse{multiply(9.7-8.82,\factor)}
%\node at (\pgfmathresult+8.82*\markPlace,-1.5) {SST};
%\node at (\pgfmathresult+8.82*\markPlace,-2.5) {\textbf{VAO, ArVO}};
%% Subaru Telescope
%\draw[<->,thick] (8*\markPlace,-8) -- (10.6*\markPlace,-8);
%\node at (\pgfmathresult+8*\markPlace,-7.5) {Subaru Telescope};
%\node at (\pgfmathresult+8*\markPlace,-8.5) {\textbf{JVO}};
%
%% Perpendicular long dotted lines
%
%\def\firstLinesEnds{3.58,3.88,8.82,9.7,13,15}
%\foreach \x in \firstLinesEnds {
%  \draw[dotted] (\x*\markPlace,0) -- (\x*\markPlace,-2);
%}
%\def\secondLinesEnds{3.4312,7.7,8.6,9,10.68}
%\foreach \x in \secondLinesEnds {
%  \draw[dotted] (\x*\markPlace,0) -- (\x*\markPlace,-4);
%}
%\draw[dotted] (8*\markPlace,0) -- (8*\markPlace,-6);
%\draw[dotted] (10.3*\markPlace,0) -- (10.3*\markPlace,-6);
%\draw[dotted] (8*\markPlace,0) -- (8*\markPlace,-8);
%\draw[dotted] (10.6*\markPlace,0) -- (10.6*\markPlace,-8);
%\draw[dotted] (8*\markPlace,0) -- (8*\markPlace,-10);
%\draw[dotted] (13*\markPlace,0) -- (13*\markPlace,-10);
%\end{tikzpicture}
%\caption{Virtual observatories that contributes to specifics astronomical
%instruments.}
%\end{center}
%\label{figure:wavelength}
%\end{figure}
%\end{landscape}
%
%In Figure \ref{figure:wavelength}, the different types of
%radiation (in order) with their common names are shown. Below, the value of
%the wavelengths in meters and the respective telescope/instrument are presented. 

%A higher $ \lambda $
%value is related with a lower frequency and a
%lower $ \lambda $ is related with a higher frequency. 
%The electromagnetic
%spectrum value or range for each instrument (above) and virtual observatory
%(below) is represented with a dot or line, respectively.\\

%\scriptsize
%\begin{table*}[h!t]
%\centering
%\begin{tabular}{|p{1cm}|p{4cm}|p{5cm}|p{6cm}|}
%    \hline                                                                      
%    \textbf{VO} & \textbf{Instrument} & \textbf{Location} &
%    \textbf{Spectrum value or range} \\
%    \hline                                                                      
%    BRAVO & Southern Astrophysical Research
%    Telescope (SOAR) & Cerro Pach\'{o}n, Chile & blue (320 nm) to near infrared
%    \cite{website:SOAR_EMS} \\
%    \hline                                                                      
%    NOVA & H-Alpha Solar Telescope for
%    Argentina (HASTA) & Leoncito, San Juan, Argentina & H-Alpha (656.28 nm)
%    \cite{website:HASTA_EMS} \\
%    \hline
%    ChiVO & Atacama Large Milimeter/submilimeter
%    Array (ALMA) & Llano de Chajnantor Observatory, Atacama Desert, Chile &
%    initially in 400 $ \mu $m to 3 mm \cite{website:ALMA_EMS} \\
%    \hline
%    \multirow{3}{3cm}{VAO} & Hubble Space
%    Telescope (HST) & 569 km above the surface of Earth & near-ultraviolet,
%    visible and near-infrared light with WFC3; ultraviolet light with COS;
%    visible light with ACS; ultraviolet, visible and near-infrared light with
%    STIS; infrared light with NICMOS \cite{website:HST_EMS} \\
%     \cline{2-4}
%     & Chandra X-ray Observatory & 139,000 km above the surface of Earth & X-ray
%     \cite{website:Chandra_EMS} \\
%     \cline{2-4}
%     & Spitzer Space Telescope (SST) & 176,602,814 km above the surface of Earth
%     \cite{website:SST_EMS_1} & 3 $ \mu $m to 180 $ \mu $m
%     \cite{website:SST_EMS_2} \\
%    \hline
%    ArVO & Byurakan Astrophysical Observatory
%    (BAO) & Mount Aragats, Armenia & 1.2 m and 4.2 m \cite{website:BAO_EMS} \\
%    \hline
%    ArVO & Spitzer Space Telescope (SST) &
%    176,602,814 km above the surface of Earth & 3 $ \mu $m to 180 $ \mu $m \\
%    \hline
%    Vobs.it & Hubble Space Telescope (HST) & 569
%    km above the surface of Earth & near-ultraviolet, visible and near-infrared
%    light with WFC3; ultraviolet light with COS; visible light with ACS;
%    ultraviolet, visible and near-infrared light with STIS; infrared light with
%    NICMOS \\
%    \hline
%    \multirow{2}{3cm}{UkrVO} & Main Astronomical
%    Observatory (MAO) & Kiev, Ukraine & \\
%     \cline{2-4}
%     & Mykolaiv Astronomical Observatory & Mykolaiv, Ukraine & \\
%    \hline
%    \multirow{3}{3cm}{JVO} & Subaru Telescope &
%    Mauna Kea, Big Island, Hawaii & visible and infrared light
%    \cite{website:Subaru_EMS} \\
%     \cline{2-4}
%     & Sloan Digital Sky Survey (SDSS) & Sacramento Mountains, Sunspot, New
%     Mexico, USA & \\
%     \cline{2-4} 
%     & Atacama Large Milimeter/submilimeter Array (ALMA) & Llano de Chajnantor
%     Observatory, Atacama Desert, Chile & initially in 400 $ \mu $m to 3 mm \\
%    \hline
%    VOI & Himalayan Chandra Telescope (HCT) & Mount
%    Saraswati, Digpa-ratsa Ri, Hanle, India & infrared light 
%    \cite{website:HCT_EMS} \\
%    \hline
% \end{tabular}
%\caption{Virtual observatories that contributes to specifics astronomical
%instruments.}
%\label{table:vo_EMS}
%\end{table*}
%%\end{longtable}
%%\end{center}
%\normalsize

