\section{Primeros pasos del Chilean VO}

\subsection{Objetivos del sistema}

\begin{frame}
\frametitle{Objetivos del sistema}
\begin{multicols}{2}
Los volúmenes de datos han generado nuevas necesidades que
permiten el desarrollo de nuevas herramientas y técnicas de análisis.

Por ejemplo, para el proyecto \textbf{ALMA}, se estima que
serán generados más de 1 TB de datos diarios.

La manipulación de altos volúmenes de datos generan complicaciones en diversos
aspectos:
\begin{itemize}
    \item <2->\textbf{Almacenamiento}, es necesario tener un centro de procesamiento de datos con
        la capacidad de almacenamiento suficiente.\\
    \item <3->\textbf{Acceso}, se deben establecer mecánicas y normativas de accesos para
        cualquier persona, ya sean astrónomos o no, lo cual requiere de un sistema que
        permita acceder a él desde cualquier lugar. \\
    \item <4->\textbf{Procesamiento}, al tener grandes volúmenes de datos, el procesamiento a
        realizar, ya sean correcciones, calibraciones, análisis, etc., exigen más tiempo
        del habitual y por supuesto que más recursos computacionales.\\
\end{itemize}

\end{multicols}
\end{frame}

\subsection{Iniciativa que lo desarrolla}

\begin{frame}
\frametitle{Iniciativa que lo desarrolla}

Actualmente se está desarrollando el proyecto titulado ``Desarrollo de una
plataforma astroinformática para la administración y análisis de datos de gran
escala'', financiado con fondos gubernamentales (FONDEF D11I1060) con duración
de 28 meses en el cual participan las siguientes instituciones:
\begin{itemize}
    \item Atacama Large Milimeter/submilimeter Array
    \item Consorcio Red Universitaria Nacional (Reuna)
    \item Universidad Técnica Federico Santa María
    \item Universidad de Chile
    \item Universidad Católica de Chile
    \item Universidad de Concepción
    \item Universidad de Santiago de Chile
\end{itemize}

\end{frame}
