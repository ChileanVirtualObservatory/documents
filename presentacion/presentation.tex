\documentclass[10pt]{beamer}
\usepackage{graphicx}
\usepackage[spanish,activeacute]{babel}
\usepackage[utf8]{inputenc}
\usepackage{color}
\usepackage{tikz}
\usepackage{multicol}

\usetheme[pageofpages=of,
          alternativetitlepage=true,
          titlepagelogo=,
          watermark=,
          watermarkheight=80px,
          watermarkheightmult=1,
          ]{Torino}

\titlegraphic{
   \includegraphics[height=1cm]{img/src/csrg.pdf}
   \hfill
   \includegraphics[height=1cm]{img/src/utfsm.pdf}
   \hfill
   \includegraphics[height=1.2cm]{img/src/fondef.pdf}
}

\usecolortheme{lirae}

%title%
\title[Primeros Pasos Chivo]{Primeros pasos del Observatorio Virtual Chileno}
\author{Jonathan Antognini C. \\
        % email
        \small{\textcolor{gray}{\texttt{jantogni@csrg.cl}}}}
\institute[CSRG-UTFSM]{
Computer Systems Research Group,\\
Universidad Técnica Fedrico Santa María
}
\date{\today}


\begin{document}

%first slide%
\begin{frame}[t,plain]
\titlepage
\end{frame}


\begin{frame}
	\frametitle{Tabla de contenidos}
	\tableofcontents
\end{frame}

\section{Introducción}

%\subsection{}

\begin{frame}
\frametitle{}
\end{frame}

\newpage

\section{Distribución de los VO's en el mundo}

%\subsection{IVOA}

\begin{frame}
\frametitle{Distribución de los VO's en el mundo}

Desde el año 2002, proyectos de VO's comenzaron a integrar la Alianza
Internacional Observatorio Virtual bajo el Guidelines for Participation
\footnote{\url{http://www.ivoa.net/documents/latest/IVOAParticipation.html}}.
\newline
\newline
Estos fueron fundados bajo programas privados y gubernamentales nacionales e
internacionales en colaboración con centro de estudios científicos,
universidades y otros. Quienes integran este proyecto, el Observatorio Virtual,
comparten conocimientos entre ellos y la comunidad de modo estandarizado. Son
ellos mismos quienes desarrollan estos estándares para el intercambio de
información e interoperabilidad.

\end{frame}

\newpage

\begin{frame}
\frametitle{Distribución actual por continente}
La siguiente figura muestra la distribución de los miembros de IVOA por continente.
\begin{multicols}{2}
\begin{figure}[h!t]
    \begin{center}
        \includegraphics[width=0.5\textwidth]{img/ivoa_vos_distribution.png}
        %\caption{Distribución por continente de IVOA.}
    \end{center}
\end{figure}

Casi la mitad de los observatorios virtuales de IVOA están en Europa: 9 del
total; 1 pertenece a Oceanía, 4 a América y 5 de ellos a Asia \footnote{Como la
mayor parte de Rusia está en territorio Asiático, es considerado como uno de
los VO's de ese contintente.} La figura 1 muestra la distribución de los
miembros de IVOA por continente.
\end{multicols}
\end{frame}

\newpage

\begin{frame}
\frametitle{Distribución si Chile es aceptado}
Si Chile se convirtiera en miembro de IVOA, la distribución sería la siguiente:
\newline

\begin{multicols}{2}
\begin{figure}[h!t]
    \begin{center}
       \includegraphics[width=0.5\textwidth]{img/if_chile_is_accepted.png}
       %\caption{International Virtual Observatory Alliance Distribución por continente incluyendo a Chile.}
    \end{center}
\end{figure}

La membresía de Chile igualaría la cantidad de VO's de Asia.  Este hecho
sería muy significativo, ya que un gran número de centros astronómicos como
observatorios están instalados en este país. Por ahora, se pretende trabajar
con datos del proyecto ALMA.
\end{multicols}

\end{frame}

\newpage

\section{Concepto de VO}

\subsection{Protocolos y estándares IVOA}

\begin{frame}
\frametitle{}
\end{frame}

\newpage

\section{Arquitectura VO}

\subsection{Arquitectura VO según IOA}

\begin{frame}
\frametitle{Arquitectura VO según VOA}
\end{frame}

\newpage

\section{Primeros pasos del Chilean VO}

\subsection{Objetivos del sistema}

\begin{frame}
\frametitle{Objetivos del sistema}
\begin{multicols}{2}
Los volúmenes de datos han generado nuevas necesidades que
permiten el desarrollo de nuevas herramientas y técnicas de análisis.

Para el proyecto \textbf{ALMA}, se estima que serán generados más de 1 TB de
datos diarios.

La manipulación de altos volúmenes de datos generan complicaciones en diversos
aspectos:
\begin{itemize}
    \item \textbf{Almacenamiento}, es necesario tener un centro de procesamiento de datos con
        la capacidad de almacenamiento suficiente.\\
    \item \textbf{Acceso}, se deben establecer mecánicas y normativas de accesos para
        cualquier persona, ya sean astrónomos o no, lo cual requiere de un sistema que
        permita acceder a él desde cualquier lugar. \\
    \item \textbf{Procesamiento}, al tener grandes volúmenes de datos, el procesamiento a
        realizar, ya sean correcciones, calibraciones, análisis, etc., exigen más tiempo
        del habitual y por supuesto que más recursos computacionales.\\
\end{itemize}

\end{multicols}
\end{frame}

\subsection{Iniciativa que lo desarrolla}

\begin{frame}
\frametitle{Iniciativa que lo desarrolla}

``Desarrollo de una plataforma astroinformática para la administración y
análisis de datos de gran escala'', financiado con fondos gubernamentales
(FONDEF D11I1060) en el cual participa:
\vspace{0.3cm}
\begin{itemize}
\setlength{\itemindent}{1.0cm}
    \item Atacama Large Milimeter/submilimeter Array
    \item Consorcio Red Universitaria Nacional
    \item Universidad Técnica Federico Santa María
    \item Universidad de Chile
    \item Universidad Católica de Chile
    \item Universidad de Concepción
    \item Universidad de Santiago de Chile
\end{itemize}
\vspace{0.5cm}
Director del proyecto: Mauricio Solar (\small{\textcolor{gray}{\texttt{msolar@inf.utfsm.cl}}})

\end{frame}

\newpage

\subsection{Conclusión y Trabajo Futuro}
%Se cumplen requerimientos.
El enfoque de desarrollo actual está orientado a prototipos incrementales, de
tal forma que se van generando entregables frecuentemente a los usuarios del
sistema (Astrónomos) los cuales se ven satisfechos por los avances realizados,
sin embargo, siempre quedan cosas por implementar o aparecen nuevas ideas.

%Cosas por implementar.
Hasta el momento el prototipo va en su primera entrega y a mediados del 2014
tendrá su segunda iteración, la cual contará con: los datos públicos del ciclo
0. Esto permitirá que más usuarios se familiaricen con el sistema por lo que se
podrá testear en terreno las funcionalidades y limitaciones actuales de las
implementaciones.

\newpage

%TODO
%¿está abierta la participación a otras universidades a colaborar en los
%desafíos que se deriven de este proyecto?
%
%¿está pensado trabajar con mas observatorios en el futuro?
%
%¿qué lecciones o guías se pueden extraer de la experiencia en la
%implementación de los otros observatorios virtuales alrededor del mundo,y del
%provecho que de estos han sacado los investigadores de las ciencias de la
%computación?


%final slide%
\begin{frame}[t,plain]
\titlepage
\end{frame}
\end{document}
