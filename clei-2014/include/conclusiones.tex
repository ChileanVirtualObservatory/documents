\section{Conclusión y Trabajo Futuro}
%Problema Resuelto
A pesar de la cantidad de datos que generan los observatorios en Chile, no
existía un VO hace un año atrás, y en particular ALMA hasta el momento tampoco
posee servicios compatibles con VO. La implementación de ChiVO generó una serie
de complicaciones, partiendo con entender las necesidades de los astrónomos y
cómo estas se traducen a sistemas de datos y servicios usando las normativas
internacionales de IVOA. 

 %Resultados
El enfoque de desarrollo actual está orientado a prototipos incrementales, de
tal forma que se van generando entregables frecuentemente a los usuarios del
sistema (Astrónomos) los cuales se ven satisfechos por los avances realizados,
sin embargo, siempre quedan cosas por implementar o aparecen nuevas ideas.
Hasta el momento se capturaron los requerimientos de los astrónomos y se
estudiaron los protocolos y estándares para abordar estos, creando así, los
servicios de acceso a datos (SCS, SIA, SSA, TAP) los que acceden a una base de
datos queusa  un modelo de datos relacional (ObsCore), sujetos a la
arquitectura propia de ChiVO la cual permite que los servicios sean
interoperables.

%Cosas por implementar.
Hasta el momento el prototipo va en su primera entrega y a mediados del 2014
tendrá su segunda iteración, la cual contará con los datos públicos del ciclo 0
de ALMA. Esto permitirá que más usuarios se familiaricen con el sistema por lo
que se podrá testear en terreno las funcionalidades y limitaciones actuales de
las implementaciones. Para las siguientes versiones se tiene pensado
implementar otros protocolos de la arquitectura de IVOA, como la sección de
Registros y estándares de acceso a recursos, como VOSpace
\cite{graham2007vospace}. En cuanto a modelos y acceso de datos
multidimensionales como los cubos de ALMA, será necesario trabajar en la
creación o adaptación de estándares de IVOA. 
