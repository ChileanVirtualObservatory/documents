\documentclass[conference]{IEEEtran}
\usepackage[spanish, es-tabla, activeacute]{babel}
\usepackage[utf8]{inputenc}
\usepackage{amsmath, amssymb}
\usepackage{graphicx}
\usepackage{anysize}
\usepackage{hyperref}
\usepackage{caption}
\usepackage[table]{xcolor}
\usepackage{booktabs}

\usepackage{color}
\newcommand{\todo}{\textcolor{red}{TO DO:}\textcolor{blue}}
\newcommand{\ror}{\emph{Ruby on Rails}}

%\renewcommand{\listtablename}{Índice de tablas}
\renewcommand{\tablename}{Tabla}

\title{ChiVO: Estado actual del Observatorio Virtual de Chile}
\author{
\IEEEauthorblockN{
    Jonathan Antognini \IEEEauthorrefmark{1},
    Mauricio Araya     \IEEEauthorrefmark{1},
    Mauricio Solar     \IEEEauthorrefmark{1} \\
}
\IEEEauthorblockA{
    \IEEEauthorrefmark{1} Universidad Técnica Federico Santa María,Valparaiso, Chile}
}

\begin{document}

\maketitle

\begin{abstract}
This paper presents the challenges, architecture and current status of the
Chilean Virtual Observatory (ChiVO), which is a software infrastructure 
for accessing and processing astronomical data generated in Chile. As ChiVO
is part of the International Virtual Observatory Alliance (IVOA), 
we strictly follow the protocols and standards that this
organization produce. However, there are always open challenges due to the
new observational technologies and local requirements 
that motivates research on every new 
virtual observatory, such as the complex data models and Big Data problems 
that the ALMA Observatory is confronting. 
The current ChiVO prototype includes IVOA compliant services as well as
new solutions designed for ALMA data, all of them using modern software
technologies. 
\end{abstract}
% Maybe agregar que lo esperamos lanzar pronto.

\begin{IEEEkeywords}
ChiVO, Virtual Observatory, Astronomy, IVOA, ALMA.
\end{IEEEkeywords}

\section{Introducción}

\subsection{Big Data en Astronomía}

En los últimos años, el problema de la avalancha de datos se ha manifestado
tanto en la ciencia como en los ámbitos empresariales, haciendo cada
día más relevante el apropiado manejo y gestión de lo que se ha denominado 
\emph{Big Data}.
Este concepto engloba a la investigación en el manejo grandes volúmenes de información, los cuales no son sencillos de procesar con las herramientas
y procedimientos tradicionales. Cuando el volumen de datos llega a ordenes
de TeraBytes a ZetaBytes, los algoritmos y procedimientos deben adaptarse
para ser usados en las nuevas plataformas de computación de alto desempeño, 
con herramientas Cloud, de forma distribuida y on-line.
%Con la idea de procesamiento de información a gran escala,
%la evolución de los métodos y recursos habitualmente utilizados
%ha sido la responsable de ser capaces de manipular grandes volúmenes de datos,
%los cuales pueden llegar a ser del orden de TeraBytes a ZetaBytes.
Adicionalmente, no sólo estamos tratando con grandes volúmenes de datos
de manera estacionaria, sino que la frecuencia con la cual éstos son generados,
crea nuevos componentes críticos para las el desarrollo de soluciones,
como lo son el almacenamiento, variabilidad de formato y tiempo de respuesta.

\emph{Big Data} no es una tecnología en sí misma, sino más bien un planteamiento de
trabajo para la obtención de valor y beneficios de los grandes volúmenes de
datos que se están generando hoy en día. Se deben contemplar aspectos como:

\begin{itemize}
    \item Cómo capturar, gestionar y explotar
    \item Cómo asegurar, verificar validez y fiabilidad.
    \item Cómo compartir para obtener mejoras y beneficios.
    \item Cómo comunicar para facilitar la toma de decisión y posteriores análisis.
\end{itemize}


Uno de los dominios donde el problema del \emph{Big Data} está llegando
a su punto crítico, es la astronomía. Las instalaciones de última generación 
en operaciones, como el Atacama Large Millimeter/submillimeter (ALMA),
y las que están en construcción, como el Large Synoptic Survey Telescope (LSST) y el Square Kilometer Array (SKA)
producen y producirán datos en gran escala, proyectándose que para el año 2020
serán más de 60 PetaBytes de información accesible para la comunidad astronómica.

%Considerando los aspectos técnicos y tecnológicos de la manipulación
%de grandes volúmenes de datos en los ejemplos anteriormente señalados, requieren
%el desarrollo de sistemas específicos para cada uno de ellos,
%los cuales consideran aspectos de captura, almacenamiento, distribución, gestión
%y análisis de la información.


\subsection{Observatorios en Chile}

Las privilegiadas condiciones atmosféricas hacen de Chile uno de los lugares
más propicios para la realización de investigaciones científicas en astronomía.
Existen más de una docena de instalaciones astronómicas de gran envergadura a
lo largo de Chile {\bf ref observatorios chile}, como por
ejemplo, el anteriormente nombrado ``Atacama Large Milimeter/submilimeter
Array'' (ALMA), el ``Very Large Telescope'' (VLT), y en los próximos años el
``European Extremely Large Telescope'' (E-ELT), con el cual se estima que el
$60\%$ de la observación astronómica mundial se realice en Chile.

Una de las condiciones que se establecen a nivel país, es que el $10\%$ del
tiempo de observación pertenece a la comunidad astronómica chilena, lo cual
justifica la necesidad a nivel país del desarrollo de una plataforma
astroinformática para una inteligente administración y análisis.

La necesidad de un sistema con éstas características no es algo nuevo, debido
que desde el $2002$ se planteó éste tipo de problemática, lo cual sugirió la
creación de un Observatorio Virtual (VO, por sus siglas en inglés) como una
solución.

El VO es una iniciativa internacional que permite el acceso de datos
astronómicos, a cargo de centros especializados para su almacenamiento y
procesamiento, a los cuales pueden acceder tanto astrónomos, como personas
comunes.  Con la estandarización de métodos e información es posible estudiar
los registros astronómicos sin requerimientos físicos de instrumentos y
localización.

\subsection{ALMA + Cubos + Formatos propios}

ALMA Consiste en recolectar una señal proveniente del cielo usando dos o más
antenas y combinarlas para analizar la señal y así obtener información de la
fuente de la emisión (ya sea una estrella, planeta, o galaxia), para ello usa
66 atenas ubicadas en Chajnantor a más de 5 mil metros de altura. Al combinar
ondas de radio capturadas por dos o más antenas, es posible obtener datos de
alta precisión y actualmente estas ondas de radio van entre frecuencias de 84
GHz y 950 GHz (ALMA Band 3 - ALMA Band 10). 

Del punto de vista de producción de datos, este proceso genera cubos en tres
dimensiones dadas por: 2 ejes de posición y un eje de frecuencia
\ref{fig:cube}. La particularidad de estos cubos de datos es su tamaño dada por
la alta precisión del observatorio, por lo que son datos de gran escala (orden
de GB y TB).

\begin{figure}[ht]
    \centering
    \includegraphics[width=0.45\textwidth]{images/cube.png}
    \caption{Estructura de cubos de datos de ALMA \cite{dent20132}}
    \label{fig:cube}
\end{figure}

En cuanto a formatos de datos en astronomía, todos se rigen por una estructura
similar compuesta por metadatos (información que describe los datos) mas datos
binarios. ALMA posee dos modelos de datos especiales:
\begin{description}
    \item[ALMA Science Data Model:] \hfill \\
        Formato en XML diseñado para guardar los metadatos obtenidos del
        proceso de observación y generar un link hace los datos binarios.
    \item[Measurement Set:] \hfill \\
        Formato basado en tablas binarias (1 principal y varias secundarias),
        el que también guarda metadata y datos binarios. Este formato se usa
        para reducir en Common Astronomy Software Applications (CASA), software
        creado por ALMA.
\end{description}

\subsection{¿Por qué es nuevo?}
%VO con datos de ALMA.
Chile siendo uno de los países con más actividad astronómica no poseía hasta
hace 1 año un VO. Y por el momento ALMA tampoco tiene posee servicios
compatibles con los protocolos y estándares de VO, por lo que es un desafío
plantear necesidades y requerimientos de este tipo de sistema.

%Resumen de lo importante del paper.
Este paper tiene como objetivo introducir al lector la arquitectura general de
Observatorios Virtuales y en particular el desarrollo actual que está
realizando el Chilean Virtual Observatory (ChiVO), desde el punto de vista de
desarrollo de software. Además se presentará el estado actual y cómo se
abarcaron las restricciones y problemáticas que están involucradas.


%\subsection{IVOA Standards} No creo que sea necesario

\section{Arquitectura de ChiVO}

\subsection{Requerimientos}

Para la creación del ChiVO se identificaron las necesidades actuales de la comunidad
astronómica:

\begin{description}
    \item[Descubrir:] \hfill \\
        Encontrar datos astronómicos de un objeto o instrumento sobre una región
        específica del espacio de alta dimensión, en base a parámetros de los ejes
        espaciales, temporales, espectrales, corrimiento al rojo, polarización, etc,
        ya sea por búsqueda o por exploración.
    \item[Obtener:] \hfill \\
        Enlace a descarga de los datos requeridos en distintos formatos, ya sea en
        el VO o en un servicio externo.
    \item[Comparar:] \hfill \\
        Cruzamiento de información de datos obtenidos entre distintas fuentes de
        información.
\end{description}

\textbf{TODO: poner los req?}

\subsection{Arquitectura}

Según las necesidades de lo radioastrónomos chilenos, los requerimientos, casos de
uso  y los modelos de datos (compatibles con los estándares de IVOA),
conllevan a la creación de la siguiente arquitectura y modelo de desarrollo:

\begin{figure}[h]
    \centering
    \includegraphics[width=0.45\textwidth]{images/chivo_capas.png}
    \caption{Arquitectura de ChiVO}
    \label{fig:chivoarch}
\end{figure}

\textbf{Capa de abstración: Clientes}

Esta capa representa al usuario final y cómo se facilita la comunicación entre el
usuario y los datos.
En esta capa el usuario realiza consultas a través de los protocolos de acceso
ofrecidos por ChiVO o a través de un formulario avanzado, utilizando aplicaciones
compatibles con VO y el portal web.
Una vez realizada la consulta, el sistema le retornará al usuario una lista que
describe objetos u observaciones encontrados (metadatos) y podrá acceder a ellos a
través de un enlace de descarga asociado a cada resultado.
Cabe destacar que gracias a la separación por capas de abstracción se logra la
flexibilidad y escalabilidad en el sistema, para que independientemente de la capa,
nuevas aplicaciones puedan interactuar con ChiVO, así como la adición de nuevas
fuentes de información a parte de ALMA.

Las consultas son recibidas por ChiVO a través de su \emph{endpoint} de datos que
recibe consultas en \texttt{HTTP}, \texttt{GET} o \texttt{POST}, ante lo cual el
endpoint retorna la lista de resultados en una tabla en el formato XML (VOTable).
Para el caso del portal web, el VOTable es desplegado mostrado al usuario a través
de una herramienta web que permite la manipulación simple y eficiente de VOTables
llamada VOView.

\textbf{Capa de abstracción: Aplicaciones}

En esta capa se encuentran los programas que procesan las consultas entre los
usuarios y los datos.
Cada estándar de IVOA requiere un mínimo de su propia implementación para ser
compatible con el VO, en el caso de estos protocolos de acceso sólo es necesaria la
recepción de consultas HTTP básicas junto a los parámetros de búsqueda requeridos.

El elemento que representa a las herramientas de análisis es fundamental en la
eficiencia de ChiVO, esto es debido a que los datos a analizar por los astrónomos
suelen tener un gran tamaño y es costosa su transferencia, este problema se
resuelve acercando las herramientas de análisis y procesamiento al lugar donde están
almacenados los datos a procesar.

Considerando que en el futuro será necesario ofrecer búsquedas
basadas en otros datos, no sólo provenientes de ALMA, es necesaria cierta
abstracción al momento de implementar esta capa, ya que debe permitir la interacción
nuevas fuentes de información, siempre y cuando se mantenga la compatibilidad
de IVOA.

En esta capa también está en desarrollo un sistema capaz de resolver nombres
(tipo sesame\textbf{ref} pero para datos de ALMA) y el registro de ChiVO.

\textbf{Capa de abstracción: Datos}

En esta capa se encuentran los recursos que tienen los datos y metadatos.
El trabajo está asociado a una base de datos relacional para almacenar los
metadatos asociados al modelo de datos recomendado por \emph{IVOA Observation Core
Data Model}, usando un framework desarrollado por el VO Alemán.
Con respecto a rendimiento, nos encontramos en la sección que consume más
recursos, tanto en tiempo de computación (resuelve las consultas hechas a las base
de datos) y además el almacenamiento físico de los datos.
A modo de verificación momentanea,
la actual implementación trabaja con un conjunto de datos, con un tamaño
de 1 TeraByte, los que provienen de la reducción del ciclo 0 de ALMA.
Debido a esta limitación, se propone el esquema de funcionamiento de la figura \textbf{X}

\begin{figure}[h]
    \centering
    \includegraphics[width=0.45\textwidth]{images/interaccion.png}
    \caption{Configuración de distintas máquinas con base de datos replicadas o distribuidas}
    \label{fig:dachs}
\end{figure}

\textbf{Arquitectura IVOA}

La arquitectura de software, está basada en el uso de protocolos y estándares de
IVOA, actuales que se están usando son:

\begin{description}
    \item[Capa Aplicación:] \hfill \\
        Un Servicio Web compatible con VO necesita al menos un \emph{Table Access
        Protocol} \textbf{ref} para acceder al modelo de datos de ChiVO usando el
        \emph{Astronomical Data Query Language} \textbf{ref}.
        Además para el cumplimiento de los requerimientos del sistema,
        actualmente se ha implementado:
        el protocolo para realizar búsquedas cónicas \emph{Simple Cone Search}
        \textbf{ref}, el protocolo para realizar acceder a datos
        espectrales \emph{Simple Spectral Access} \textbf{ref} y el protocolo de
        acceso a imágenes \emph{Simple Image Access} \textbf{ref}.

    \item[Capa de datos:] \hfill \\
        En esta capa se requiere la configuración de la base de datos relacional con
        un modelo de datos recomendado por IVOA llamado \emph{Observation Core Data
        Model} \textbf{ref} que permite que los VO sean interoperables,
        ya que definen una cantidad mínima de atributos en las tablas con cierto
        nombre y tipo de dato, de forma que acceder a diferentes servicios mediante
        \emph{TAP + Obscore} \textbf{ref} es estándar.
        Además el formato de transferencia de información (metadata) es con el
        formato \emph{XML VOTable}.
\end{description}

\begin{figure}[h]
    \centering
    \includegraphics[width=0.45\textwidth]{images/arquitectura_2.png}
    \caption{Arquitectura de IVOA con los Protocolos y Estandares usados}
    \label{fig:ivoarch}
\end{figure}

\subsection{Metadatos de los datos de ALMA}

Para poder construir la base de datos relacional con el modelo de datos ObsCore fue
necesario mapear campos desde el ASDM.

\begin{table}[h!t]
    \centering
    \begin{tabular}{lr}
        \textbf{Campo ObsCore} & \textbf{ASDM} \\
        dataproduct\_type      & visibility \\
        calib\_level           & 1 \\
        obs\_collection        & ALMA \\
        obs\_id                & [ExecBlock.execBlockUID] \\
        obs\_publisher\_did    & [Cycle ID] \\
        access\_url            & [URL de ChiVO] \\
        access\_format         & application/x-asdm \\
        access\_estsize        & [main.dataSize] \\
        target\_name           & [Source.sourceName] \\
        s\_ra                  & [Source.direction] \\
        s\_dec                 & [Source.direction] \\
        s\_fov                 & [1.2 * lambda / Diametro antena] \\
        s\_region              & circle \\
        s\_resolution          & [1.2*lambda/(ExecBlock.baseRangeMax)] \\
        t\_min                 & [ExecBlock.startTime] \\
        t\_max                 & [ExecBlock.endTime] \\
        t\_exptime             & [main.interval] \\
        t\_resolution          & [mainTable.interval] \\
        em\_min                & [ExecBlock.baseRangeMin] \\
        em\_max                & [ExecBlock.baseRangeMax] \\
        em\_res\_power         & [spectralWindow.resolution] \\
        o\_ucd                 & em.mm \\
        pol\_states            & [Source.stokesParameter[numStokes]] \\
        facility\_name         & ALMA \\
        instrument\_name       & ALMA \\
    \end{tabular}
    \caption{Campos del ObsCore y origen desde ASDM}
    \label{table:obsasdm}
\end{table}

% Re-redactar...
En la Tabla \ref{table:obsasdm} se muestra el resultado de la investigación,
a la izquierda se despliegan las columnas de la clase Observation,
la segunda columna indica de donde se obtienen los datos para asignar la los campos
de la primera columna para el caso de los ASDM.

Para poder llenar los campos de la clase Observation es necesario escribir una
rutina capaz de operar sobre las tablas del ASDM (XML).
Actualmente existen múltiples herramientas en el Paquete de Aplicaciones de Software
Comunes de Astronomía (CASA, debido a sus siglas en inglés) \textbf{ref}.

\subsection{Tecnologías utilizadas}

Para el desarrollo de ChiVO se evaluaron distintas herramientas posibles de las
cuales se concluyó en cada capa:

\textbf{Endpoint}

Los framework que se evaluaron para la implementación del endpoint fueron:

\begin{description}
    \item[Ruby on Rails (RoR):] \hfill \\
        Framework de desarrollo web ampliamente utilizado el día de hoy,
        se basa en el concepto Modelo-Vista-Controlador (MVC).
        Sin embargo, ésta herramienta no será utilizada debido
        a que muchas funcionalidades no son necesarias para el presente proyecto.
    \item[Python/Flask:] \hfill \\
        Flask es un microframework diseñado especialmente para servicios y
        herramientas web pequeñas.
        La presente solución provee un marco de trabajo para la creación de
        aplicaciones web que puedan ser accedidas mediante distintos métodos HTTP.
        Existe mucha documentación y comunidad activa que permite implementar y
        solucionar problemas de forma rápida.
\end{description}

\textbf{ALMA Resource}

Dentro de los \emph{toolkits} de DAL recomendados por IVOA,
se probaron y verificaron los siguientes:

\begin{description}
    \item[SAADA:] \hfill \\
        Desarrollado por el VO Francés, es una herramienta bastante útil del punto
        de vista del usuario del sistema.
        Posee excelente documentación y un conveniente proceso de instalación.
        Está desarrollado en Java y su correspondiente despliegue se lleva a cabo
        mediante Tomcat.
        Es posible configurar servicios SCS/SIA/SSA/TAP y no es un proyecto
        OpenSource.

    \item[VO-Dance:] \hfill \\
        Desarrollado por el VO Italiano, es una herramienta en Java en su Backend,
        y Python en su Frontend (Framework Django).
        Lo destacable de esta herramienta es que trabaja usando MySQL como motor de
        base de datos principal, y de acuerdo a las últimas noticias relacionadas
        a su desarrollo, podría existir un soporte para PostgreeSQL en el futuro.
        La herramienta no es OpenSource y la documentación es precaria debido
        a que aún está en desarrollo. Compatible con servicios SCS/SIA/SSA/TAP.

    \item[openCADC:] \hfill \\
        Desarrollado por el VO Canadiense, es una herramienta OpenSource escrita en
        Java, utilizada actualmente en el ALMA Science Archive.
        Este toolkit es uno de los más robustos, contiene distintos paquetes con
        servicios a ser utilizados en el webservice, sin embargo posee una
        documentación precaria, lo que es compensado por abierta comunidad de
        desarrollo. Es posible configurar servicios TAP.

    \item[DaCHS:] \hfill \\
        Desarrollado por el VO Alemán, es una herramienta OpenSource escrita en
        Python.
        Es uno de los toolkits DAL más usados por los VO, ya que posee una amplia
        documentación de instalación y configuración.
        Es posible configurar servicios SCS/SIA/SSA/TAP.
\end{description}

%\vspace{0.5cm}
\begin{table*}[h!t]
\centering
\begin{tabular}{lrrrrr}
    {\bf Toolkits} & {\bf Lenguaje} & {\bf OpenSource} & {\bf Documentación} & {\bf Servicios} & {\bf Último update}  \\
    SAADA          & Java           & No               & Si                  & SCS/SIA/SSA/TAP & Mayo 2012     \\
    VO-Dance       & Java/Python    & No               & No                  & SCS/SIA/SSA/TAP & Dicimbre 2012 \\
    openCADC       & Java           & Si               & No                  & TAP             & ---           \\
    DaCHS          & Python         & Si               & Si                  & SCS/SIA/SSA/TAP & Junio 2013    \\
\end{tabular}
\caption{Resumen de los toolkits en tabla comparativa}
\label{table:toolkits}
\end{table*}

\textbf{Interfaz Usuario}

Inicialmente la interfaz usuario o frontend poseería solo vistas, por lo que
el desarrollo podía ser en prácticamente cualquier lenguaje o framework, como por
ejemplo HTML+PHP, Django o RoR. Sin embargo con los requerimientos de la
plataforma, especialmente el de capa de usuarios, se decidió escoger un
framework MVC que fuese lo suficientemente ágil y compatible con el resto de
servicios, RoR.

\section{Estado de avance}

El diagrama de secuencia de interacción entre el usuario y el ChiVO se puede
ver en la figura \textbf{citar}. En base a este diagrama, requerimientos y
tecnologías utilizadas el estado de avance se especificará por cada capa.

\begin{figure}[h]
    \centering
    \includegraphics[width=0.45\textwidth]{images/secuencia.png}
    \caption{Diagrama de secuencia entre Usuario y ChiVO}
    \label{fig:secuencia}
\end{figure}

\subsection{Backend}
Dibujo de asdm -> ms -> fits \}metadata

Data Models

\subsection{Endpoint}
ChiVO Sesame, Registros externos.

\subsection{Frontend}
Tecnologías actuales

\subsection{Conclusión y Trabajo Futuro}
%Se cumplen requerimientos.
El enfoque de desarrollo actual está orientado a prototipos incrementales, de
tal forma que se van generando entregables frecuentemente a los usuarios del
sistema (Astrónomos) los cuales se ven satisfechos por los avances realizados,
sin embargo, siempre quedan cosas por implementar o aparecen nuevas ideas.

%Cosas por implementar.
Hasta el momento el prototipo va en su primera entrega y a mediados del 2014
tendrá su segunda iteración, la cual contará con: los datos públicos del ciclo
0. Esto permitirá que más usuarios se familiaricen con el sistema por lo que se
podrá testear en terreno las funcionalidades y limitaciones actuales de las
implementaciones.


\section*{Agradecimientos}
Este trabajo ha sido financiado parcialmente con el proyecto FONDEF D11I1060.

\section{Referencias}
\bibliographystyle{plain}
\bibliography{ccs}

\end{document}
