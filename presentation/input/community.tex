\section{Involved community}

\begin{frame}
\frametitle{Involved community}
Currently, a project entitled ”Development of an Astro-Informatic Platform for Management and
Intelligent Analysis of Large-scale Data” is being developed, financed by government funding, for a period of 28 months, 
and with the participation of the following institutions:
\begin{itemize}
	\addtolength{\itemindent}{1cm}
	\item Atacama Large Millimiter/submillimeter Array (ALMA)
	\item Consorcio Red Universitaria Nacional Reuna (REUNA)
	\item Universidad Técnica Federico Santa María
	\item Universidad de Chile
	\item Universidad Catolica de Chile
	\item Universidad de Concepción
	\item Universidad de Santiago de Chile
\end{itemize}
\end{frame}

\begin{frame}
%\frametitle{Development of an Astro-Informatic Platform for Management and Intelligent Analysis of Large-scale Data}

The \emph{\textbf{Goals}} of the project are related to the design and implementation of a virtual observatory, which
shall comply the standard of the International Virtual Observatory Alliance (IVOA), in closely collaboration with the
Chilean astronomical community.\\
\vspace{0.7cm}
Also ”Consorcio Red Universitaria Nacional” will provide solution for the problems related to the connectivity of high
rate data transmition, and the National Laboratory for High Performance Computing, will helps storing the data will require large storage
capabilities.\\
\vspace{0.7cm}
These institutions join efforts to achieve, create and establish in time a platform unprecedented in the
Chilean astroinformatics field, the Chilean Virtual Observatory (\emph{\textbf{ChiVO}}).
\end{frame}
