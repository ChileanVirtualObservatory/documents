\section{Conclusiones}

\begin{frame}
\frametitle{Conclusiones}

El observatorio virtual es un \textit{framework} que le permite a los astrónomos y
comunidad en general buscar en múltiples servidores de datos de forma
transparente, pero un foco de mayor interés para la comunidad informática es
que guía la construcción de un sistema robusto a partir de tecnologías,
estándares y protocolos unificados. Esto enmarcado en su arquitectura orientada
a la intercomunicación mediante 3 capas: usuarios, intermedia (\textit{virtual
observatory}) y de recursos, y que gracias a la especificación de cada una,
están formalizados los formatos de representación de datos y los métodos por
los cuales se puede acceder a los mismos.

\end{frame}

\begin{frame}
%\frametitle{}
\begin{beamercolorbox}[center,shadow=true,rounded=true,]{note} 
        \huge ¿Preguntas? 
\end{beamercolorbox} 
\end{frame}
