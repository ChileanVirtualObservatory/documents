\section{Introducción}
\begin{frame}
\frametitle{Introducción}
Existe una gran cantidad de observatorios importantes en Chile, como por
ejemplo: ``Atacama Large Milimeter/submilimeter Array'' (ALMA), ``Very Large
Telescope'' (VLT), y en los próximos años el``European Extremely Large
Telescope'' (E-ELT), con el cual se estima que el 60\% de la observación
astronómica mundial se realice en Chile.  
\newline

El \textbf{Observatorio Virtual}(VO) es una iniciativa internacional que permite el
acceso a archivos astronómicos y centros de datos a astrónomos y personas
comunes a través de Internet. 

\end{frame}

\begin{frame}
La \textbf{International Virtual Observatory Alliance} (IVOA) fue creada para
``facilitar la coordinación internacional y colaboración necesaria para el
desarrollo y distribución de herramientas, sistemas y estructuras
organizacionales necesarias para permitir la utilización internacional de
archivos astronómicos como un observatorio virtual integrado e
interoperable``\footnote{\url{http://www.ivoa.net/about/what-is-ivoa.html}}.
\newline
\newline
Presentación:
\begin{itemize}
\setlength{\itemindent}{0.5cm}
    \item Dar a conocer la distribución de los VO's en el mundo.
    \item Explicar el concepto de VO.
    \item Explicar la arquitectura a grandes rasgos de un
        VO.
    \item Presentar los primeros pasos para llevar a cabo el
        Chilean Virtual Observatory (ChiVO).
    \item Conclusiones y trabajo a futuro.
\end{itemize}

\end{frame}
