\section{Metodología de Trabajo}

El presente informe se generó de forma iterativa mediante reuniones periodicas
entre los participantes del proyecto, mezclando los conocimientos en astronomía
como del área técnica informática.

El objetivo fue poder formalizar de manera técnica los requerimientos que
pleantean los futuros usuarios del sistema, aterrizando esto a la duración del
proyecto. El futuro ChiVO, busca estar entre las filas de IVOA, por ende, cada
requerimiento debió contemplar los protocolos y estándares regidos en esta
organización. Es por ello que previamente a cada reunión, los integrantes del
área informática estudiaron cada estándar relacionado, con el fin de poder
explicar de qué manera se podían implementar las soluciones.

El flujo de reuniones fue:
\begin{itemize}
	\item \emph{Marzo}: 2 Reuniónes Presenciales. Iniciales para la definición de
requerimientos. Estas reuniones fueron más informativas de parte de los
astrónomos, en donde fueron explicando conceptos generales.	
	\item \emph{Abril}: 2 Reuniónes Presenciales. En base a los requerimientos
iniciales, se fueron estudiando y explicando los estándares relacionados de
IVOA.
	\item \emph{Mayo}: 3 Reuniones Presenciales. Estas reuniones tuvieron por
objetivo profundizar los requerimientos y definirlos de forma más detallada.
	\item \emph{Junio}: 2 Reuniones Presenciales y 1 online. Estas últimas
reuniones tuvieron por objetivo compartir la captura de requerimientos con
otras personas entendidas en el área, para ver su opinión o posibles mejoras.
Ya en la última reunión estaban definidos todos los requerimientos con
prioridad y su grado de necesidad.
\end{itemize}

En las reuniones anteriorires no se consideraron las reuniones de los
investigadores ni del comité directivo. Estas últimas tienen un enfoque de
presentación de avances y se toman los comentarios hechos por el comité para
mejorar cada entregable. En estas reuniones están presente todas las
universidades, REUNA y ALMA.  
