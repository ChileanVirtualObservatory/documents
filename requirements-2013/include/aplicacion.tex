\section{Capa de aplicación}


La definición de requerimientos de datos involucra en particular a un área de
arquitectura de software recomendada por IVOA. Esta capa es la de
estándares y protocolos de acceso a los datos (Data Access Protocol Standards).

Los datos y metadatos astronómicos están siendo distribuidos mediante distintos
mecanismos que le permitan a la comunidad astronómica y científica acceder a
ellos de forma fácil y estandarizada. Por lo general, estos accesos son mediante
páginas web, aplicaciones de escritorio, servicios File Transfer Protocol
(FTP), etc. La colaboración de distintos observatorios virtuales ha permitido
crear protocolos estandarizados para consultar y acceder a datos y metadatos de
distintas fuentes de información. Estos protocolos fueron creados para permitir
que la implementación fuese más fácil.

Durante la ejecución del proyecto se han estudiado estos protocolos, enmarcado
en la idea de VO, y la arquitectura que se debe respetar.

\textbf{Estándares y Protocolos de acceso de datos} \begin{enumerate} \item
\textbf{Buscar por coordenadas o región del cielo}: las búsquedas por
coordenadas en el espacio, involucra la definición de una coordenada RA/DEC, y
además de un radio angular. Este tipo de consultas están consideradas dentro de
un protocolo de acceso de datos definido por IVOA, el cual se llama Simple Cone
Search. Otra búsqueda puede ser por región del cielo definiendo una región
rectangular, el protocolo relacionado es Simple Image Access Protocol.

	\item \textbf{Buscar por metadatos espaciales}: las búsquedas
espaciales requieren definir parámetros de rangos de resolución angular y
campos de visión en base a coordenadas. El protocolo de acceso con el cual se
deberá implementar este requerimiento es mediante Table Access Protocol (TAP),
basándose en Astronomical Data Query Language.

	\item \textbf{Buscar por nombre o tipo de objeto}: este requerimiento
considera buscar en ciertos catálogos astronómicos los cuales se definirán a lo
largo del proyecto, teniendo en consideración qué tipo de datos necesitará la
comunidad astronómica y los participantes de este proyecto. Los catálogos
astronómicos son objetos astronómicos agrupados por características en común.
Para este punto se tendrá que estudiar y aprovechar el marco de trabajo que
provee IVOA en su grupo de Semánticas.

	\item \textbf{Buscar por metadatos espectrales}: las búsquedas por
metadatos espectrales se basará en búsqueda por banda o rango de frecuencia,
líneas espectrales, búsquedas por resolución espectral y redshift. Este tipo de
búsqueda se realizará mediante el uso de los protocolos Table Access Protocol y
Simple Spectra Access Protocol.

	\item \textbf{Buscar por metadatos temporales}: la búsqueda por
metadatos temporales se enfocará a fechas de cuando se hizo cierta observación,
duración, veces que se observó, etc. Esto se cubrirá usando Table Access
Protocol.

	\item \textbf{Cruzamiento de información}: este punto está relacionado
al punto 3. Lo que se busca es poder brindarle a la comunidad astronómica
distintas fuentes de datos, aparte de los datos de ALMA. Para ello se definirán
a lo largo del proyecto, cuales son los de interés, qué catálogos y de qué
observatorios virtuales se obtendrán. Cabe destacar que estos procesos son
estándares ya que todos los servicios están normados por los mismos protocolos.

	\item \textbf{Simulaciones}: se buscará sobre servicios que posean
datos de simulaciones para permitir comparaciones teóricas con reales. En IVOA
el protocolo relacionado a Simulations Data Access Layer, está aún en
desarrollo, sin embargo hay versiones iniciales del estándar.

	\item \textbf{Servicios Bibliográficos}: durante la ejecución del
proyecto, y después cuando ChiVO esté en funcionamiento, existirán fuentes de
documentación científica, ya sean reportes técnicos o papers. Se implementará
un buscador que a partir de información como: rango de año, autor, fuente en
estudio, se le entregue al usuario una lista de artículos que hayan calzado con
la búsqueda. Algo similar a como funciona SIMBAD.

	\item \textbf{Herramientas de análisis}: durante la ejecución del
proyecto se implementarán herramientas de análisis de datos de gran escala, las
cuales serán integradas dentro de lo posible (si es que se pueden usar de forma
web) a la plataforma online, en otro caso, se proveerá acceso y documentación
de como usar cada herramienta, tutoriales y casos de ejemplo.
\end{enumerate}
