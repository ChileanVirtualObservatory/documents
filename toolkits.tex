\documentclass[conference]{IEEEtran}
\usepackage[spanish, activeacute]{babel}
\usepackage[utf8]{inputenc}
\usepackage{amsmath, amssymb}
\usepackage{graphicx}
\usepackage{anysize}
\usepackage{hyperref}
\usepackage{caption}
\usepackage[table]{xcolor}
\usepackage{booktabs}
\usepackage{color}



\newcommand{\todo}{\textcolor{red}{TO DO:}\textcolor{blue}}
\newcommand\R{R}
%\renewcommand{\refname}{References}



%\title{Comparación entre Data Access Layer Toolkits bajo estándares de IVOA}
\title{A comparison between Data Access Layer Toolkits under IVOA standard}
\author{
\IEEEauthorblockN{
	Mauricio Solar \IEEEauthorrefmark{1}, Jonathan Antognini \IEEEauthorrefmark{1}, Marcelo Mendoza \IEEEauthorrefmark{1}, Cristián Maureira \IEEEauthorrefmark{1} \\
	Jorge Ibsen \IEEEauthorrefmark{2}, Lars Nyman \IEEEauthorrefmark{2},
	Eduardo Vera \IEEEauthorrefmark{3}, Diego Mardones \IEEEauthorrefmark{3}, Guillermo Cabrera \IEEEauthorrefmark{3},\\
	Paola Arellano \IEEEauthorrefmark{4},
	Karim Pichara \IEEEauthorrefmark{5}, Nelson Padilla \IEEEauthorrefmark{5},
	Ricardo Contreras \IEEEauthorrefmark{6}, \\ Neil Nagar \IEEEauthorrefmark{6},
	Victor Parada \IEEEauthorrefmark{7}.
}

\\

\IEEEauthorblockA{\IEEEauthorrefmark{1} Universidad Técnica Federico Santa María, Valparaiso, Chile}
\IEEEauthorblockA{\IEEEauthorrefmark{2} Atacama Large Millimeter/submillimeter Array, San Pedro de Atacama, Chile}
\IEEEauthorblockA{\IEEEauthorrefmark{3} Universidad de Chile, Santiago, Chile}
\IEEEauthorblockA{\IEEEauthorrefmark{4} Red Universitaria Nacional, Santiago, Chile}
\IEEEauthorblockA{\IEEEauthorrefmark{5} Universidad Católica de Chile, Santiago, Chile}
\IEEEauthorblockA{\IEEEauthorrefmark{6} Universidad de Concepción, Concepción, Chile}
\IEEEauthorblockA{\IEEEauthorrefmark{7} Universidad de Santiago de Chile, Santiago, Chile}
}

\begin{document}

\maketitle

\begin{abstract}
%Este trabajo presenta el aspecto computacional de un observatorio astronómico virtual,
%centrándose en la capa de acceso a los datos, y muestra cómo la International Virtual
%Observatory Alliance maneja este aspecto en su arquitectura. Se expone el
%trabajo que se está realizando en la implementación del Chilean Virtual Observatory.
The current work presents the computational aspect of a Astronomical Virtual Observatory,
focusing in the data access layer, and shows how the International Virtual Observatory Alliance
handles this aspect in its architecture.
The on-going work related to the implementation of the Chilean Virtual Observatory
is presented.
\end{abstract}

\begin{IEEEkeywords}
IVOA, Data Access Protocol, Virtual Observatory, Chivo.
\end{IEEEkeywords}

\section{Introducción}

En la actualidad se está implementando el primer observatorio astronómico virtual chileno
para los datos del observatorio Atacama Large Millimeter/submillimeter Array (ALMA), que
busca entregar los datos científicos a la comunidad astronómica chilena y
del mundo. Este proyecto se realiza en colaboración con cinco universidades
chilenas: Universidad Técnica Federico Santa María, Universidad de Chile,
Pontificia Universidad Católica, Universidad de Concepción y la Universidad de
Santiago de Chile. También participa la Red Universitaria Nacional (REUNA) y el
Centro de Modelamiento Matemático (CMM).

\subsection{Datos Astronómicos}
Los telescopios e instrumentos astronómicos están afectos a los avances
tecnológicos, lo cual a nivel de información es una ventaja, dado que
se logra captar datos que antes no existían. Sin embargo, computacionalmente, la
preocupación está centrada en el vertiginoso crecimiento de los volúmenes de datos
generados, que ya pasó de los gigabytes (GB) a los terabytes (TB) en la década pasada, y
que pasará de los TB a los petabytes (PB) en la actual década. Por ejemplo:
\begin{itemize}
	\item Galaxy Evolution Explorer (GALEX) \cite{galex}: primer telescopio
orbitante en el espacio que observa galaxias en luz ultravioleta. En los primeros 3 años 
generó 30 TB de datos.
	\item Sloan Digital Sky Survey (SDSS) \cite{sloan}: es un proyecto de inspección de
imágenes en el espectro visible y de corrimiento al rojo, que en su séptima versión publicó más de 60 TB de datos:
15.7 TB en imágenes, 18 TB en catálogos, y 26.8 TB en otros productos.
	\item Atacama Large Millimeter/submillimeter Array (ALMA) \cite{alma}: es un
interferómetro que trabaja con 66 radio telescopios de platos de distintos
tamaños, obteniendo datos de radio, polarización, etc. Se espera que genere 1
TeraBytes de datos por día de observación.
	\item Panoramic Survey Telescope \& Rapid Response System (Pan-STARRS) \cite{pan}:
es un sistema de 4 telescopios (PS4) que permitirá obtener imágenes a gran campo de manera continua.
Su objetivo es caracterizar objetos que se aproximan a la tierra, como
asteroides, cometas, etc. Se espera que genere 10 TB de datos por noche de observación.
\end{itemize}

Se podrían seguir listando observatorios, pero en resumen, existe una
avalancha creciente de datos astronómicos \cite{kborne}, los cuales están cambiando la forma
en que se hace astronomía. Desde el punto de vista computacional, es necesario
implementar servicios estándares que permitan acceder, procesar, y modelar los
datos producidos en cada observatorio (que generan datos públicos) de forma
estándar, naciendo así la idea de Observatorio Virtual. Los estándares y
protocolos de cómo operar están a cargo de la International Virtual Observatory
Alliance (IVOA). 

\subsection{Observatorio Virtual}
El observatorio virtual (VO, por sus siglas en inglés) no es un paquete de software (escritorio o web) que
permite a los usuarios acceder a los datos como si fuese un repositorio, que es
lo que generalmente se esperaría al escuchar la palabra. Técnicamente, un VO
puede ser descrito como una arquitectura integral, que formaliza en cada
nivel de la aplicación los estándares y protocolos necesarios, de tal manera
que los VO del mundo sean interoperables entre si. Por lo tanto, cuando un
observatorio tiene datos para publicar no será necesario inventar una nueva
arquitectura, sino que se puede aprovechar la ya existente.

Así nace el año 2002 IVOA \cite{ivoa}, cuya misión es facilitar
la coordinación y colaboración necesaria para permitir el acceso global e
integrado a los datos recogidos por los observatorios astronómicos
internacionales. En esta organización están comprometidos actualmente 19 VO,
los cuales trabajan mediante working groups en la creación y versionamiento de
estándares y protocolos que comprende su arquitectura.

\subsection{Arquitectura IVOA}

IVOA propone tres niveles de arquitectura, los cuales se pueden describir
dentro del Nivel 2, ya que el grado de especificación va aumentando \cite{arch}.

En esta definición, se identifican principalmente 3 capas:
\begin{itemize}
	\item Resource Layer: compuesto por colección de datos y metadatos.
	\item User Layer: consumidores de datos a través de navegador,
aplicación de escritorio, etc.
	\item Middle Layer: esta capa es necesaria para conectar las dos
anteriores en forma transparente tanto para los usuarios como para los
publishers. Para este trabajo el foco será la sección de \emph{Data Access Protocol}.
\end{itemize}

\subsection{Data Access Protocol}

Definen una familia de interfaces de servicios de acceso a todos los datos
astronómicos disponibles vía VO.  El grupo de Data Access, describe cómo los
proveedores de datos harán disponible la información a los usuarios, y cómo los
usuarios recuperarán la información. Existen varios estándares involucrados
en esta sección, algunos de ellos son:

\begin{itemize}
	\item Simple Image Access \cite{sia}: una consulta que define una región
rectangular en el cielo para obtener imágenes candidatas. El servicio devuelve
una lista de imágenes candidatas presentada en formato VOTable (Referencia).
	\item Simple Cone Search \cite{scs}: define una interfaz sencilla donde los
usuarios pueden recuperar información tabular alrededor de una posición dada en
el cielo. Por ejemplo, los usuarios pueden encontrar todos los objetos cercanos
a un punto determinado de un catálogo de imágenes. La respuesta devuelve una
lista de fuentes en formato VOTable.
	\item Simple Spectral Access \cite{ssa}: define una interfaz uniforme para
descubrir y acceder remotamente espectros de una dimensión. Se basa en un
modelo de datos más general, capaz de describir los datos espectrofotométricos,
incluyendo series de tiempo y las distribuciones espectrales de energía, así
como 1-D de espectros. Los conjuntos de datos de candidatos disponibles se
describen de manera uniforme en un documento de formato de VOTable que se
devuelve en la respuesta a la consulta.
	\item Table Access Protocol \cite{tap}: define un protocolo de servicio de acceso
a datos de la tabla general. El acceso se proporciona tanto para base de datos
y metadatos de la tabla, así como para los datos reales de la tabla.
\end{itemize}

\section{What will be done?}


\section{References}
\bibliographystyle{plain}
\bibliography{toolkits}

\end{document}
