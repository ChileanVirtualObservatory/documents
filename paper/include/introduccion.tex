\section{Introducción}

Las provilegiadas condiciones atmosféricas hacen de Chile uno de los lugares
más propicios para la realización de investigaciones científicas en astronomía.
Organizaciones astronómicas como el Observatorio Europeo Austral (ESO por sus
siglas en inglés) han elegido asentar sus enormes telescopios en este país para 
llevar a cabo sus descubrimientos\footnote{Acerca de ESO. (s.f.). En
\textit{ESO}. \url{http://www.eso.org/public/chile/about-eso.html}}. 

Existen más de una docena de instalaciones astronómicas de gran envergadura a
lo largo de nuestro territorio nacional \cite{observatorios_chile}, como por
ejemplo ``Atacama Large Milimeter/submilimeter Array'' (ALMA), ``Very Large
Telescope'' (VLT), y en los próximos ``European Extremely Large Telescope''
(E-ELT), con el cual se estima que el 60\% de la observación astronómica
mundial se realice en Chile.  Una de las condiciones que se establecen a nivel
país, es que el 10\% del tiempo de observación pertenece a la comunidad
astronómica chilena. Estos generan datos a gran escala, justificando a nivel
país, el desarrollo de una plataforma astroinformática para su administración y
análisis inteligente.

El Observatorio Virtual (VO por sus siglas en inglés) es una iniciativa
internacional que permite el acceso a archivos astronómicos y centros de datos
a astrónomos y personas comunes a través de Internet. Con la estandarización de
métodos e información es posible estudiar los registros astronómicos sin
requerimientos físicos de instrumentos y locación.

La International Virtual Observatory Alliance (IVOA) fue creada para
``facilitar la coordinación internacional y colaboración necesaria para el
desarrollo y distribución de herramientas, sistemas y estructuras
organizacionales necesarias para permitir la utilización internacional de
archivos astronómicos como un observatorio virtual integrado e 
interoperable\footnote{Traducido de \url{http://www.ivoa.net/about/what-is-ivoa.html}}.
Actualmente, IVOA está compuesta por 19 proyectos\footnote{En el sitio web oficial en la
sección \textbf{What is the IVOA} \textit{``the IVOA now comprises 17 VO
projects''}, pero en \textbf{Members Organizations} aparecen 19 miembros
listados.} de América, Asia, Europa y Oceanía; sus miembros se reunen
dos veces cada año en \textbf{Interoperability Workshops} para entablar
discusiones cara-a-cara y resolver preguntas técnicas.

Actualmente una iniciativa liderada por la UTFSM propone
desarrollar una plataforma astro-informática para la administración y análisis
inteligente de datos a gran escala basados en los estándares de IVOA.  Por lo
anterior, este documento tiene como objetivo:

\begin{itemize}
	\item Dar a conocer la distribución de los VO's en el mundo.
	\item Explicar el concepto de VO.
	\item Explicar la arquitectura a grandes rasgos de un
		VO.
	\item Presentar los primeros pasos para llevar a cabo el
		Chilean Virtual Observatory (ChiVO).
	\item Conclusiones y trabajo a futuro.
\end{itemize}
