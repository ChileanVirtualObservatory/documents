\section{Resumen Ejecutivo}

En el marco de la creación de una ``plataforma
astroinformática para la administración y análisis inteligente de datos a gran
escala``, se dio comienzo al desarrollo del Chilean Virtual Observatory (ChiVO).

El desarrollo de software a medida consiste en la creación y fabricación de
sistemas informáticos que satisfacen necesidades específicas de un área. Es por
esto, que la creación de ChiVO se ha enfocado en satisfacer las necesidades que
se le presentan a la comunidad astronómica y se puso en marcha el proceso de
captura y especificación de requerimientos, los cuales han recibido
modificaciones, detalles, comentarios, de distintos expertos del área afin.

En conjunto, las universidades participantes del proyecto han mantenido
reuniones de trabajo con el fin de especificar y clarificar las funcionalidades
que esperan de un VO, siendo partícipe también el observatorio Atacama Large
Millimeter/submillimeter Array (ALMA), los cuales dispondrán los datos públicos
para que sean accesibles mediante esta plataforma. En el presente documento se
detalla:
\begin{itemize}
	\item \textbf{La especificación de requerimientos}: cada requerimiento está bien
fundado y explicado de tal forma que quede claro tanto para la comunidad, como
para los desarrolladores finales del sistema. Cada requerimiento a su vez,
tiene grado de necesidad y prioridad temporal. Este documento permite el
análisis de los estándares de IVOA que deben considerarse.
	\item \textbf{Análisis capa de modelos de datos}: en base a los requerimientos
se analizó qué estándares de IVOA permiten modelar de forma correcta los datos,
los cuales se citarán y resumirán.
	\item \textbf{Análisis capa de aplicación}: en base a los requerimientos se
analizó qué estándares de IVOA permiten el acceso a los datos de forma correcta
e interoperable, los cuales se citarán y resumirán.
	\item \textbf{Análisis modelo vista controlador}: en base a los requerimientos
se creó una maqueta no funcional de cómo operaría la plataforma de acceso a los
datos de ChiVO.
\end{itemize}

Considerando estas 4 especificaciones, se genera un marco de trabajo para la implementación de ChiVO.
