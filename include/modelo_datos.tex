\section{Capa de modelo de datos}
\subsection{Estándares y Protocolos IVOA}
Breve descripción de los protocolos y estándares a considerar. Es recomendado
el estudio de éstos en el siguiente orden:

\begin{enumerate}
	\item[a.] \textbf{Observation Core Data Model}: Define los componentes
principales de todos los metadatos accesibles que tienen un rol en el
descubrimiento de las observaciones.  Requerimientos: Todos, debido a que es el
Modelo de Datos base. \url{http://www.ivoa.net/documents/ObsCore/index.html} 

	\item[b.] \textbf{Units}: Define las prácticas comunes en la
manipulación de unidades en los metadatos astronómicos y define una
representación consistente en los servicios dentro de un Observatorio Virtual.
Necesario para: Observation Core Data Model, Simple Spectral Line Data Model,
Characterisation Data Model, Simulations Data Model.
\url{http://www.ivoa.net/documents/VOUnits/index.html}

	\item[c.] \textbf{Utypes}: Relacionado con Units, Utypes son nombres
que definen inequívocamente los elementos de los metadatos. Necesario para:
Observation Core Data Model, Simple Spectral Line Data Model, Characterisation
Data Model, Simulations Data Model.
\url{http://www.ivoa.net/documents/Notes/UTypesUsage/index.html}

	\item[d.] \textbf{Simple Spectral Line Data Model}: Describe las
transiciones de líneas espectrales. Podría ser necesario, de otra manera, la
información puede estar contenida en el Observation Core Data Model.
Requerimientos: (3) Buscar por metadatos espectrales (frecuencia y resolución).
\url{http://www.ivoa.net/documents/SSLDM/}

	\item[e.] Characterisation Data Model: Define y organiza todos los
metadatos necesarios para describir cómo un conjunto de datos ocupa un espacio
físico multidimensional, cuantitativamente y, cuando es relevante,
cualitativamente. Requerimientos: (3) Buscar por metadatos espectrales
(frecuencia y resolución), (4) Buscar por metadatos espaciales (resolución
angular y campos de visión).
\url{http://www.ivoa.net/Documents/latest/ImplementationCharacterisation.html}

	\item Space Time Coordinate metadata: Describe metadatos espaciales y
temporales. Podría ser necesario para usos específicos de tiempo-espacio.
Requerimientos: (1) Buscar por coordenadas o región del cielo, (2) Buscar por
nombre o  tipo de objeto, (4) Buscar por metadatos espaciales (resolución
angular y campos de visión), (5) Buscar por metadatos temporales.
\url{http://www.ivoa.net/Documents/latest/STC-Model.html}

	\item Simulations Data Model: Define y organiza todos los metadatos
necesarios para describir conjuntos de datos de simulaciones. Requerimientos:
(7) Simulaciones. \url{http://www.ivoa.net/documents/SimDM/index.html}
\end{enumerate}

\subsection{Modelo de datos}

Se especificará una tabla con los atributos y la descripción de cada uno, en
concordancia a los recomendado con IVOA.

\begin{table}[h!t]
	\centering
	\begin{tabular}{|l|l|l|} 
		\hline
		Nombre de la columna & Unidad & Descripción \\
		\hline
		dataproduct\_type & sin unidad & Tipo de dato lógico \\
		calib\_level & sin unidad & Nivel de calibración \\
		obs\_collection & sin unidad & Nombre de la colección de datos \\
		obs\_id & sin unidad & ID de observación \\
		obs\_publisher\_id & sin unidad & Identificador de datos entregado por el publisher \\
		access\_url & sin unidad & URL para acceder a los datos \\
		access\_format & sin unidad & Formato del archivo \\
		access\_estsize & kbyte & Tamaño estimado en KB \\
		target\_name & sin unidad & Objeto astronómico observado \\
		s\_ra & grados & RA central \\
		s\_dec & grados & Dec central \\
		s\_fov & grados & Diámetros de la región cubierta \\
		s\_region & sin unidad & Región cubierta en ADQL \\
		s\_resolution & arco segundos & Resolución espacial \\
		t\_min & días & Hora de comienzo \\
		t\_max & días & Hora de fin \\
		t\_exptime & segundos & Tiempo de exposición \\
		t\_resolution & segundos & Resolución temporal \\
		em\_min & minutos & Comienzo en coordenadas espectrales \\
		em\_max & minutos & Fin en coordenadas espectrales \\
		em\_res\_power & sin unidad & Poder de resolución espectral \\
		o\_ucd & sin unidad & UCD observable \\
		pol\_states & sin unidad & Lista de estados de polarización \\
		facility\_name & sin unidad & Nombre del lugar usado para la observación \\
		instrument\_name & sin unidad & Nombre del instrumento usado para la observación \\
		\hline
	\end{tabular}
	\caption{Integrantes de IVOA}
	\label{table:tap_column_name}
\end{table}

