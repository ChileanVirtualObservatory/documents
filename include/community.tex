%Comunidad
%
%Actualmente se está desarrollando el proyecto titulado “Desarrollo de una
%plataforma astroinformática para la administración y análisis de datos de gran
%escala”, financiado por fondos gubernamentales (FONDEF D11I1060), con duración
%de 28 meses, en el cual participan las siguientes instituciones:
%
%- Atacama Large Milimiter/submilimiter Array
%
%- Consorcio Red Universitaria Nacional Reuna
%
%- Universidad Técnica Federico Santa María
%
%- Universidad de Chile
%
%- Universidad Católica de Chile
%
%- Universidad de Concepción
%
%- Universidad de Santiago de Chile
%
%Los objetivos del proyecto están relacionados con el diseño e implementación de
%un observatorio virtual, el cual deberá cumplir con los estándares de la
%“International Virtual Observatory Alliance” (IVOA). Además los astrónomos
%investigadores del proyecto, crearán instancias donde presentarán problemáticas
%que enfrentan como comunidad, ante el procesamiento de datos, los cuales se
%resolverán mediante técnicas computacionales conocidas por los investigadores
%del área de computación.
%
%El presente proyecto se realiza en estrecha colaboración con ALMA, quienes
%aportan su visión desde el punto de vista de observatorio, comparten
%conocimiento respecto a los modelos y tipos de datos que se usan, y además se
%establecerán políticas de colaboración para facilitar el acceso a los datos.
%
%Por otro lado el Consorcio Red Universitaria Nacional Reuna y el
%National Laboratory for High Performance Computing juega otro rol
%importante, ya que una de las problemáticas a resolver es la conectividad de
%altas tasas de transmisión de datos (REUNA) y almacenar datos que exigen
%grandes capacidades de almacenamiento (NLHPC).
%
%En conjunto estas instituciones unen esfuerzos para lograr crear y establecer
%en el tiempo una plataforma sin precedente en el área astroinformática Chilena,
%el Chilean Virtual Observatory (ChiVO http://www.chivo.cl/).

\section{Community involved}

Currently it is developing the project entitled "Development of an Astro-Informatic Platform 
for Management and Intelligent Analysis of Large-scale Data", financed by government funds (FONDEF D11I1060), with
duration of 28 months, and with the participation of the following institutions:

\begin{itemize}
	\item Atacama Large Milimiter/submilimiter Array (ALMA)
	\item Consorcio Red Universitaria Nacional Reuna (REUNA)
	\item Universidad Técnica Federico Santa María
	\item Universidad de Chile
	\item Universidad Católica de Chile
	\item Universidad de Concepción
	\item Universidad de Santiago de Chile
\end{itemize}

The project's objectives are related to the design and implementation of a
virtual observatory, which shall comply with the standars of the International
Virtual Observatory Allience (IVOA). In addition the astronomers of the
project, will create instances where present problem faced as a community, to
the processing of data, which will be resolved by computational techniques
known by the researchers in the computing field.

This project is conducted in close collaboration with ALMA who bring their
vision from the point of view of observatroy, share knowledge about models and
data types that are used, and also establish collaborative policies to
facilitate access to data.

On the other hand, Consorcio Red Universitaria Nacional Reuna and the National
Laboratory for High Performance Computing, plays another important role, as one
of the problem to solve is the connectivity of high data transmission rates
(REUNA) and store data that require large capacities of storage (NLHPC).

In joint these institutions join efforts to achieve create and establish in
time a platform unprecedented in the Chilean Astroinformatics area, the Chilean
Virtual Observatory (ChiVO).
