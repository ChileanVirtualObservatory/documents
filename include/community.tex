%Comunidad
%
%Actualmente se está desarrollando el proyecto titulado “Desarrollo de una
%plataforma astroinformática para la administración y análisis de datos de gran
%escala”, financiado por fondos gubernamentales (FONDEF D11I1060), con duración
%de 28 meses, en el cual participan las siguientes instituciones:
%
%- Atacama Large Milimiter/submilimiter Array
%
%- Consorcio Red Universitaria Nacional Reuna
%
%- Universidad Técnica Federico Santa María
%
%- Universidad de Chile
%
%- Universidad Católica de Chile
%
%- Universidad de Concepción
%
%- Universidad de Santiago de Chile
%
%Los objetivos del proyecto están relacionados con el diseño e implementación de
%un observatorio virtual, el cual deberá cumplir con los estándares de la
%“International Virtual Observatory Alliance” (IVOA). Además los astrónomos
%investigadores del proyecto, crearán instancias donde presentarán problemáticas
%que enfrentan como comunidad, ante el procesamiento de datos, los cuales se
%resolverán mediante técnicas computacionales conocidas por los investigadores
%del área de computación.
%
%El presente proyecto se realiza en estrecha colaboración con ALMA, quienes
%aportan su visión desde el punto de vista de observatorio, comparten
%conocimiento respecto a los modelos y tipos de datos que se usan, y además se
%establecerán políticas de colaboración para facilitar el acceso a los datos.
%
%Por otro lado el Consorcio Red Universitaria Nacional Reuna (REUNA) y el
%National Laboratory for High Performance Computing (NLHPC) juega otro rol
%importante, ya que una de las problemáticas a resolver es la conectividad de
%altas tasas de transmisión de datos (REUNA) y almacenar datos que exigen
%grandes capacidades de almacenamiento (NLHPC).
%
%En conjunto estas instituciones unen esfuerzos para lograr crear y establecer
%en el tiempo una plataforma sin precedente en el área astroinformática Chilena,
%el Chilean Virtual Observatory (ChiVO http://www.chivo.cl/).

\section{Community involved}
