\section{State of the Art}

En la actualidad se está implementando el primer observatorio virtual chileno
para los datos del Atacama Large Milimeter/submilimeter Array (ALMA), quien
busca poder entregar sus datos científicos a la comunidad astronómica Chilena y
del mundo. Este proyecto se realiza en colaboración a cinco universidades
chilenas: Universidad Técnica Federico Santa María, Universidad de Chile,
Pontificie Universidad Católica, Universidad de Concepción y la Universidad de
Santiago de Chile. También participa la Red Universitaria Nacional (REUNA) y el
Centro de Modelamiento Matemático (CMM).

\subsection{Astronomical Data}
%avalancha de datos
Los telescopios e instrumentos en general en la astronomía, están afectos a los
avances tecnológicos. Este aspecto a nivel de información es muy provechoso, ya
que logran captar datos que antes no existían. Sin embargo computacionalmente
la preocupación está centrada en el crecimiento de volúmenes de datos generados, que ya pasó de los gigabytes a
los terabytes en la decada pasada, y que pasará de los terabytes a petabytes en
la actual década. Por ejemplo:
\begin{itemize}
	\item Galaxy Evolution Explorer (GALEX): es un telescopio orbitante en
el espacio, que observa galaxias en luz ultra violeta. Se espera que genere 30
TB de Datos.
	\item Sloan Digital Sky Survey (SDSS): es un proyecto de inspección de
imágenes en el espectro visible y de corrimiento al rojo. Se espera que genere
40 TB de Datos.
	\item Atacama Large Milimeter/submilimeter Array (ALMA): es un
interferómetro que trabaja con 66 radio telescopios de platos de distintos
tamaños, obteniendo datos de radio, polarización, etc. Se espera que genere 1
TB de Datos por día de observación.
	\item Panoramic Survey Telescope \& Rapid Response System (Pan-STARRS):
es un sistema que permite obtener imágenes a gran campo de manera contínua. Su
objetivo es caracterizar objetos que se aproximen a la tierra, como asteroides,
cometas, etc. Se espera que genera 40 Petabytes de datos.
\end{itemize}

%descripción observatorio virtual
Técnicamente un VO, puede ser descrito como un servicio integral, que provee datos en sus diferentes formatos como también herramientas de software de interés en la comunidad científica.

%arquitectura observatorio virtual

%Capa de acceso de datos

%Estándares de acceso de datos






























