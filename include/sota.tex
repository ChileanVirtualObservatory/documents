\section{Introduction}

%En la actualidad se está implementando el primer observatorio astronómico
%virtual chileno para los datos del observatorio Atacama Large
%Millimeter/submillimeter Array (ALMA), que busca entregar los datos científicos
%a la comunidad astronómica chilena y del mundo. Este proyecto se realiza en
%colaboración con cinco universidades chilenas: Universidad Técnica Federico
%Santa María, Universidad de Chile, Pontificia Universidad Católica, Universidad
%de Concepción y la Universidad de Santiago de Chile. También participa la Red
%Universitaria Nacional (REUNA) y el Centro de Modelamiento Matemático (CMM).

%Inglés
Currently its being developing the first Chilean Virtual Observatory for the
Atacama Large Millimeter/submillimeter Array (ALMA) data, that it looks to
provide the scientific data to the chilean and world community. This project
its being developed in collaboration between five universities: Universidad
Técnica Federico Santa María, Universidad de Chile, Pontificia Universidad
Católica, Universidad de Concepción y la Universidad de Santiago de Chile.
Also participates the Red Universitaria Nacional (REUNA) and the Centro de
Modelamiento Matemático (CMM).

\subsection{Astronomical Data}
%Los telescopios e instrumentos astronómicos están afectos a los avances
%tecnológicos, lo cual a nivel de información es una ventaja, dado que se logra
%captar datos que antes no existían. Sin embargo, computacionalmente, la
%preocupación está centrada en el vertiginoso crecimiento de los volúmenes de
%datos generados, que ya pasó de los gigabytes (GB) a los terabytes (TB) en la
%década pasada, y que pasará de los TB a los petabytes (PB) en la actual
%década. Por ejemplo: \begin{itemize}
%	\item Galaxy Evolution Explorer (GALEX) \cite{galex}: primer telescopio
%orbitante en el espacio que observa galaxias en luz ultravioleta. En los
%primeros 3 años generó 30 TB de datos.
%	\item Sloan Digital Sky Survey (SDSS) \cite{sloan}: es un proyecto de
%inspección de imágenes en el espectro visible y de corrimiento al rojo, que en
%su séptima versión publicó más de 60 TB de datos: 15.7 TB en imágenes, 18 TB en
%catálogos, y 26.8 TB en otros productos.
%	\item Atacama Large Millimeter/submillimeter Array (ALMA) \cite{alma}: es un
%interferómetro que trabaja con 66 radio telescopios de platos de distintos
%tamaños, obteniendo datos de radio, polarización, etc. Se espera que genere 1
%TeraBytes de datos por día de observación.
%	\item Panoramic Survey Telescope \& Rapid Response System (Pan-STARRS) \cite{pan}:
%es un sistema de 4 telescopios (PS4) que permitirá obtener imágenes a gran campo de manera continua.
%Su objetivo es caracterizar objetos que se aproximan a la tierra, como
%asteroides, cometas, etc. Se espera que genere 10 TB de datos por noche de observación.
%\end{itemize}

%Inglés
Telescopes and astronomical instrument are affected by te tecnological
advances, which at information level its a adventage, because is achieved to
capture inexistent data. However, computationally, the concern has focused on
the rapid growth of the generated data volumes, that has passed of gigabyte
(GB) to the terabytes (TB) in the past decade, and will pass of TB to the
petabytes (PB) in the actual decade. For example:
\begin{itemize}
	\item Galaxy Evolution Explorer (GALEX) \cite{galex}: is the first
orbitant telescope in the space which observe galaxies in UV light. In the
first 3 years was created 30 TB of data.
	\item Sloan Digital Sky Survey (SDSS) \cite{sloan}: is a project of
image inspection in the visible specter and redshift, which in it 7th version
60 TB of data was published: 15.7 TB of image, 18 TB of catalogue, and 26.8 TB
in other products.
	\item Atacama Large Millimeter/submillimeter Array (ALMA) \cite{alma}:
its a interferometer which use 66 radio telescopes with dishes of different
size, obtaining radio data, polarization, etc. 1 TB of data its expected per
day of observatrion.
	\item Panoramic Survey Telescope \& Rapid Response System (Pan-STARRS)
\cite{pan}: its a system with 4 telescopes (PS4) which enable to obtain images
on a large field in continuous manner. Its goal is characterize objects that
will come closer to the earth, as asteroids, comets, etc. 10 TB of data its
expected per night of observation.
\end{itemize}

%Se podrían seguir listando observatorios, pero en resumen, existe una
%avalancha creciente de datos astronómicos \cite{kborne}, los cuales están cambiando la forma
%en que se hace astronomía. Desde el punto de vista computacional, es necesario
%implementar servicios estándares que permitan acceder, procesar, y modelar los
%datos producidos en cada observatorio (que generan datos públicos) de forma
%estándar, naciendo así la idea de Observatorio Virtual. Los estándares y
%protocolos de cómo operar están a cargo de la International Virtual Observatory
%Alliance (IVOA). 

%Inglés
Observatories could continue to be listing, but in summary, there is a growing
avalanche of astronomical data \cite{kborne}, which are changing the way that
astronomy is done. With respect to the computational issues, its necessary to
implement standard services which allow to access, process, and modeling the
data produced by each observatory (which generates public data). Thus was born
the idiea of Virtual Observatory (VO). The standards and protocols of how to
operate are in charf of International Virtual Observatory Alliance (IVOA). 

\subsection{Virtual Observatory}
%El observatorio virtual (VO, por sus siglas en inglés) no es un paquete de
%software (escritorio o web) que permite a los usuarios acceder a los datos como
%si fuese un repositorio, que es lo que generalmente se esperaría al escuchar la
%palabra. Técnicamente, un VO puede ser descrito como una arquitectura integral,
%que formaliza en cada nivel de la aplicación los estándares y protocolos
%necesarios, de tal manera que los VO del mundo sean interoperables entre si.
%Por lo tanto, cuando un observatorio tiene datos para publicar no será
%necesario inventar una nueva arquitectura, sino que se puede aprovechar la ya
%existente.

%Así nace el año 2002 IVOA \cite{ivoa}, cuya misión es facilitar
%la coordinación y colaboración necesaria para permitir el acceso global e
%integrado a los datos recogidos por los observatorios astronómicos
%internacionales. En esta organización están comprometidos actualmente 19 VO,
%los cuales trabajan mediante working groups en la creación y versionamiento de
%estándares y protocolos que comprende su arquitectura.

%Inglés
The Virtual Observatory (VO), its not a software package (desktop or web) thats
allow to the user access to the data like a web repository, which is usually
expected from that word. Technically, a VO can be described as an integral
architecture, which formalizes in each application level protocols and
standards necessary, such a way that the world VO will be interoperable with
each other.  Therefore, when an observatory want to publish data will not be
necessary to invent a new architecture, they can use the existing one.

Thus in the 2002 IVOA was born \cite{ivoa}, whose basic mission is focussed on
facilitate coordination and collaboration necessary to enable global and
integrated access to data collected by the international astronomical
observatories. In this organization are currently engaged 19 VO, which work by
working groups in the creation and versioning of standards and protocols which
includes its architecture.

\subsection{IVOA Architecture}

%IVOA propone tres niveles de arquitectura, los cuales se pueden describir
%dentro del Nivel 2, ya que el grado de especificación va aumentando \cite{arch}.

%En esta definición, se identifican principalmente 3 capas:
%\begin{itemize}
%	\item Resource Layer: compuesto por colección de datos y metadatos.
%	\item User Layer: consumidores de datos a través de navegador,
%aplicación de escritorio, etc.
%	\item Middle Layer: esta capa es necesaria para conectar las dos
%anteriores en forma transparente tanto para los usuarios como para los
%publishers. Para este trabajo el foco será la sección de \emph{Data Access Protocol}.
%\end{itemize}

%Inglés
IVOA proposes three levels architecture, which can be described within the
level 2, because the level of detail is increasing \ cite {arch}.

In this definition, are identified primarily 3 layers:
\begin{itemize}
	\item Resource layer: composed by collection of data and metadata.
	\item User Layer: data consumers through web browser, desktop application, script based, etc.
	\item Middle Layer: This layer is required to connect the two previous
transparently for both users and publishers. In this work the focus will be the
section of \emph{Data Access Protocols}.
\end{itemize}

\subsection{Data Access Protocol}

%Definen una familia de interfaces de servicios de acceso a todos los datos
%astronómicos disponibles vía VO.  El grupo de Data Access, describe cómo los
%proveedores de datos harán disponible la información a los usuarios, y cómo los
%usuarios recuperarán la información. Existen varios estándares involucrados
%en esta sección, algunos de ellos son:
%
%\begin{itemize}
%	\item Simple Image Access \cite{sia}: una consulta que define una región
%rectangular en el cielo para obtener imágenes candidatas. El servicio devuelve
%una lista de imágenes candidatas presentada en formato VOTable (Referencia).
%	\item Simple Cone Search \cite{scs}: define una interfaz sencilla donde los
%usuarios pueden recuperar información tabular alrededor de una posición dada en
%el cielo. Por ejemplo, los usuarios pueden encontrar todos los objetos cercanos
%a un punto determinado de un catálogo de imágenes. La respuesta devuelve una
%lista de fuentes en formato VOTable.
%	\item Simple Spectral Access \cite{ssa}: define una interfaz uniforme para
%descubrir y acceder remotamente espectros de una dimensión. Se basa en un
%modelo de datos más general, capaz de describir los datos espectrofotométricos,
%incluyendo series de tiempo y las distribuciones espectrales de energía, así
%como 1-D de espectros. Los conjuntos de datos de candidatos disponibles se
%describen de manera uniforme en un documento de formato de VOTable que se
%devuelve en la respuesta a la consulta.
%	\item Table Access Protocol \cite{tap}: define un protocolo de servicio de acceso
%a datos de la tabla general. El acceso se proporciona tanto para base de datos
%y metadatos de la tabla, así como para los datos reales de la tabla.
%\end{itemize}

%Inglés
Define a family of access services interfaces to the astronomical data
available through VO. The Data Access group, describes how the data providers
will share the information to users, and how users retrieve information. There
are several standards involved in this section, some of these are:
\begin{itemize}
	\item Simple Image Acess \cite{sia}: A query defining a rectangular
region on the sky is used to query for candidate images. The service returns a
list of candidate images formatted as VOTable \cite{votable}.
	\item Simple Cone Search \cite{scs}: The query describes sky position
and an angular distance, defining a cone on the sky. The response returns a
list of astronomical sources formatted as VOTable.
	\item Simple Spectral Access \cite{ssa}: defines an interface to
remotely discover and access one-dimensional spectra. It is based on a more
general data model, able to describe the spectrophotometric data, including
time series and spectral energy distributions as well as 1-D spectra. The
response returns a list of astronomical sources formatted as VOTable.
	\item Table Access Protocol \cite{tap}: defines a service protocol for
accessing general table data. The access is provided for both database and
table metadata as well as for actual table data.
\end{itemize}
