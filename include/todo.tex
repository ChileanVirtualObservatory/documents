\section{What has been done?}

%Hasta el momento se han estudiado los estándares y protocolos de IVOA
%necesarios para saber como trabajar con cada toolkit: Simple Image Access (SIA),
%Simple Cone Search, Simple Spectra Access, Table Access Protocol (TAP), Observation
%Data Model Core Components \cite{obscore} y su implementación en TAP, entre otros.

%Además se han configurado las etapas iniciales de los toolkits: openCADC,
%SAADA, DaCHS.

%Inglés
So far, the IVOA standards and protocols have been studied, which is needed to know how to
work with each toolkit: Simple Image Access (SIA), Simple Cone Search, Simple
Spectra Access, Table Access Protocol (TAP), Observation Data Model Core
Components and its Implementation in the Table Access Protocol, among others.

In addition, the initial stages of the toolkits have been configure: openCADC,
SAADA, DaCHS.

\section{What will be done?}

%Se van a testear los toolkits recomendados por IVOA: OpenCADC, SAADA, DaCHS,
%DALServer. los cuales en su mayoría soportan TAP, SIA, SSA, SCS.

%Inglés
The recommended toolkits by IVOA will be tested: OpenCADC, SAADA, Dachs,
DALServer. Which mostly support: TAP, SIA, SSA, SCS.

\section{Expected Results}

%Se construirá a partir de estos toolkits un web service con el mismo modelo de
%datos y la misma base de datos, buscando ver cual es más escalable, estable, fácil
%de manejar; además de cual tiene mejor documentación, y si es que tienen código
%abierto, qué mejoras se podrían hacer. Se espera que la curva de aprendizaje de
%uso de este tipo de toolkits comience con una pendiente bastante baja, esto
%porque es necesario conocer algunos lenguajes de programación desde el punto de
%vista práctico más que teórico, es decir, hay que lidiar con varias
%bibliotecas, etc. Sin embargo se cree que una vez superada una etapa inicial la
%curva tome una pendiente más alta y finalmente decaiga.

%Inglés
It will be built from these toolkits a web service with the same data model and
database, looking to see which is more scalable, stable, easy to handle; in
addition to which of them has better documentation, and if they are open source, what
improvements could be made, etc. It is expected that the learning curve of using
toolkits start with a fairly low slope, because it is necessary to know
some programming languages, but from practical point of view, rather than
theoretical, i.e, using multiple libraries, etc. However it is believed that
after passing the initial stages, the curve takes a higher slope and eventually
decays.

