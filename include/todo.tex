\section{¿Qué se ha hecho?}

Hasta el momento se han estudiado los estándares y protocolos de IVOA
necesarios para saber como trabajar con cada toolkit: Simple Image Access (SIA),
Simple Cone Search, Simple Spectra Access, Table Access Protocol (TAP), Observation
Data Model Core Components \cite{obscore} y su implementación en TAP, entre otros.

Además se han configurado las etapas iniciales de los toolkits: openCADC,
SAADA, DaCHS.

\section{¿Qué se hará?}

Se van a testear los toolkits recomendados por IVOA: OpenCADC, SAADA, DaCHS,
DALServer. los cuales en su mayoría soportan TAP, SIA, SSA, SCS.

\section{Resultados esperados}

Se construirá a partir de estos toolkits un web service con el mismo modelo de
datos y la misma base de datos, buscando ver cual es más escalable, estable, fácil
de manejar; además de cual tiene mejor documentación, y si es que tienen código
abierto, qué mejoras se podrían hacer. Se espera que la curva de aprendizaje de
uso de este tipo de toolkits comience con una pendiente bastante baja, esto
porque es necesario conocer algunos lenguajes de programación desde el punto de
vista práctico más que teórico, es decir, hay que lidiar con varias
bibliotecas, etc. Sin embargo se cree que una vez superada una etapa inicial la
curva tome una pendiente más alta y finalmente decaiga.
