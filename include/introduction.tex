%\section{Introducción}
%
%La astronomía en Chile es uno de los campos científicos que más ha crecido en
%los últimos años, debido a que es considerado como uno de los mejores sitios en
%el mundo para llevar a cabo esta ciencia, ya que posee una serie de condiciones
%climáticas privilegiadas.  Existen más de una docena de instalaciones
%astronómicas a lo largo de nuestro territorio nacional, como por ejemplo el
%“Atacama Large Milimiter/submilimiter Array” (ALMA), el “Very Large Telescope”
%(VLT), y en los próximos el “European Extremely Large Telescope” (E-ELT), con
%el cual se estima que el 60\% de la observación astronómica mundial se realice
%en Chile.  Una de las condiciones que se establecen a nivel país, es que el
%10\% del tiempo de observación pertenece a la comunidad astronómica chilena, lo
%que genera datos a gran escala, siendo suficientes para tener la necesidad a
%nivel de país, de desarrollar una plataforma astroinformática para su
%administración y análisis inteligente.

\section{Introduction}

In Chile the astronomy is one of the cientific field that has 
