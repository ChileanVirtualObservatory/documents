%\section{Introducción}
%
%La astronomía en Chile es uno de los campos científicos que más ha crecido en
%los últimos años, debido a que es considerado como uno de los mejores sitios en
%el mundo para llevar a cabo esta ciencia, ya que posee una serie de condiciones
%climáticas privilegiadas.
%Existen más de una docena de instalaciones
%astronómicas a lo largo de nuestro territorio nacional, como por ejemplo el
%“Atacama Large Milimiter/submilimiter Array” (ALMA), el “Very Large Telescope”
%(VLT), y en los próximos el “European Extremely Large Telescope” (E-ELT), con
%el cual se estima que el 60\% de la observación astronómica mundial se realice
%en Chile.  Una de las condiciones que se establecen a nivel país, es que el
%10\% del tiempo de observación pertenece a la comunidad astronómica chilena, lo
%que genera datos a gran escala, siendo suficientes para tener la necesidad a
%nivel de país, de desarrollar una plataforma astroinformática para su
%administración y análisis inteligente.

\section{Introduction}

Nowadays in  Chile, the astronomy has become one of the scientific fields with a
surprising growth rate in the last years, because it is considered one of the best
places in the world to study this field, due the climatic conditions and
the clear skies.
There are more than a dozen astronomical facilities in our country, such as the
Atacama Large Millimeter/submillimeter Array (ALMA), the Very Large
Telescope (VLT), and soon the European Extremely Large Telescope (E-ELT),
which will give to Chile an estimation of the $60\%$  of the worldwide astronomical
observation.
One of the conditions established in our country, is
that the $10\%$ of the observation time belongs to the Chilean astronomical community,
which generates large-scale data, being enough to have the need as a country,
to develop an astroinformatic platform for the data administration and
intelligent analysis.
