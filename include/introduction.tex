%\section{Introducción}
%
%La astronomía en Chile es uno de los campos científicos que más ha crecido en
%los últimos años, debido a que es considerado como uno de los mejores sitios en
%el mundo para llevar a cabo esta ciencia, ya que posee una serie de condiciones
%climáticas privilegiadas.  Existen más de una docena de instalaciones
%astronómicas a lo largo de nuestro territorio nacional, como por ejemplo el
%“Atacama Large Milimiter/submilimiter Array” (ALMA), el “Very Large Telescope”
%(VLT), y en los próximos el “European Extremely Large Telescope” (E-ELT), con
%el cual se estima que el 60\% de la observación astronómica mundial se realice
%en Chile.  Una de las condiciones que se establecen a nivel país, es que el
%10\% del tiempo de observación pertenece a la comunidad astronómica chilena, lo
%que genera datos a gran escala, siendo suficientes para tener la necesidad a
%nivel de país, de desarrollar una plataforma astroinformática para su
%administración y análisis inteligente.

\section{Introduction}

In Chile the astronomy is one of the cientific field that has more growth in
the last years, because it is considered one of the best places in the world to
make this science, as it has a series of privileged climatic conditions.  There
are more then dozen astronomical facilities over our country, such as the
"Atacama Large Milimiter / submilimiter Array" (ALMA), the "Very Large
Telescope" (VLT), and soon the "European Extremely Large Telescope" (E-ELT),
with this it is estimated that 60\% of the worldwide astronomical observation
will be held in Chile. One of the conditions established in the country, is
that 10\% of the observing time belongs to the Chilean astronomical community,
which generates large-scale data, being sufficient to have the need at country
level, to develop a astroinformatics platform for data administration and
intelligent analysis.
