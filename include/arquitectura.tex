\section{Arquitectura de ChiVO}

\subsection{Requerimientos}

Para la creación del ChiVO se identificaron las necesidades actuales de la comunidad
astronómica:

\begin{description}
    \item[Descubrir:] \hfill \\
        Encontrar datos astronómicos de un objeto o instrumento sobre una región
        específica del espacio de alta dimensión, en base a parámetros de los ejes
        espaciales, temporales, espectrales, corrimiento al rojo, polarización, etc,
        ya sea por búsqueda o por exploración.
    \item[Obtener:] \hfill \\
        Enlace a descarga de los datos requeridos en distintos formatos, ya sea en
        el VO o en un servicio externo.
    \item[Comparar:] \hfill \\
        Cruzamiento de información de datos obtenidos entre distintas fuentes de
        información.
\end{description}

\textbf{TODO: poner los req?}

\subsection{Arquitectura}

Según las necesidades de lo radioastrónomos chilenos, los requerimientos, casos de
uso  y los modelos de datos (compatibles con los estándares de IVOA),
conllevan a la creación de la siguiente arquitectura y modelo de desarrollo:

\begin{figure}[h]
    \centering
    \includegraphics[width=0.45\textwidth]{images/chivo_capas.png}
    \caption{Arquitectura de ChiVO}
    \label{fig:chivoarch}
\end{figure}

\textbf{Capa de abstración: Clientes}

Esta capa representa al usuario final y cómo se facilita la comunicación entre el
usuario y los datos.
En esta capa el usuario realiza consultas a través de los protocolos de acceso
ofrecidos por ChiVO o a través de un formulario avanzado, utilizando aplicaciones
compatibles con VO y el portal web.
Una vez realizada la consulta, el sistema le retornará al usuario una lista que
describe objetos u observaciones encontrados (metadatos) y podrá acceder a ellos a
través de un enlace de descarga asociado a cada resultado.
Cabe destacar que gracias a la separación por capas de abstracción se logra la
flexibilidad y escalabilidad en el sistema, para que independientemente de la capa,
nuevas aplicaciones puedan interactuar con ChiVO, así como la adición de nuevas
fuentes de información a parte de ALMA.

Las consultas son recibidas por ChiVO a través de su \emph{endpoint} de datos que
recibe consultas en \texttt{HTTP}, \texttt{GET} o \texttt{POST}, ante lo cual el
endpoint retorna la lista de resultados en una tabla en el formato XML (VOTable).
Para el caso del portal web, el VOTable es desplegado mostrado al usuario a través
de una herramienta web que permite la manipulación simple y eficiente de VOTables
llamada VOView.

\textbf{Capa de abstracción: Aplicaciones}

En esta capa se encuentran los programas que procesan las consultas entre los
usuarios y los datos.
Cada estándar de IVOA requiere un mínimo de su propia implementación para ser
compatible con el VO, en el caso de estos protocolos de acceso sólo es necesaria la
recepción de consultas HTTP básicas junto a los parámetros de búsqueda requeridos.

El elemento que representa a las herramientas de análisis es fundamental en la
eficiencia de ChiVO, esto es debido a que los datos a analizar por los astrónomos
suelen tener un gran tamaño y es costosa su transferencia, este problema se
resuelve acercando las herramientas de análisis y procesamiento al lugar donde están
almacenados los datos a procesar.

Considerando que en el futuro será necesario ofrecer búsquedas
basadas en otros datos, no sólo provenientes de ALMA, es necesaria cierta
abstracción al momento de implementar esta capa, ya que debe permitir la interacción
nuevas fuentes de información, siempre y cuando se mantenga la compatibilidad
de IVOA.

En esta capa también está en desarrollo un sistema capaz de resolver nombres
(tipo sesame\textbf{ref} pero para datos de ALMA) y el registro de ChiVO.

\textbf{Capa de abstracción: Datos}

En esta capa se encuentran los recursos que tienen los datos y metadatos.
El trabajo está asociado a una base de datos relacional para almacenar los
metadatos asociados al modelo de datos recomendado por \emph{IVOA Observation Core
Data Model}, usando un framework desarrollado por el VO Alemán.
Con respecto a rendimiento, nos encontramos en la sección que consume más
recursos, tanto en tiempo de computación (resuelve las consultas hechas a las base
de datos) y además el almacenamiento físico de los datos.
A modo de verificación momentanea,
la actual implementación trabaja con un conjunto de datos, con un tamaño
de 1 TeraByte, los que provienen de la reducción del ciclo 0 de ALMA.
Debido a esta limitación, se propone el esquema de funcionamiento de la figura \textbf{X}

\begin{figure}[h]
    \centering
    \includegraphics[width=0.45\textwidth]{images/interaccion.png}
    \caption{Configuración de distintas máquinas con base de datos replicadas o distribuidas}
    \label{fig:dachs}
\end{figure}

\textbf{Arquitectura IVOA}

La arquitectura de software, está basada en el uso de protocolos y estándares de
IVOA, actuales que se están usando son:

\begin{description}
    \item[Capa Aplicación:] \hfill \\
        Un Servicio Web compatible con VO necesita al menos un \emph{Table Access
        Protocol} \textbf{ref} para acceder al modelo de datos de ChiVO usando el
        \emph{Astronomical Data Query Language} \textbf{ref}.
        Además para el cumplimiento de los requerimientos del sistema,
        actualmente se ha implementado:
        el protocolo para realizar búsquedas cónicas \emph{Simple Cone Search}
        \textbf{ref}, el protocolo para realizar acceder a datos
        espectrales \emph{Simple Spectral Access} \textbf{ref} y el protocolo de
        acceso a imágenes \emph{Simple Image Access} \textbf{ref}.

    \item[Capa de datos:] \hfill \\
        En esta capa se requiere la configuración de la base de datos relacional con
        un modelo de datos recomendado por IVOA llamado \emph{Observation Core Data
        Model} \textbf{ref} que permite que los VO sean interoperables,
        ya que definen una cantidad mínima de atributos en las tablas con cierto
        nombre y tipo de dato, de forma que acceder a diferentes servicios mediante
        \emph{TAP + Obscore} \textbf{ref} es estándar.
        Además el formato de transferencia de información (metadata) es con el
        formato \emph{XML VOTable}.
\end{description}

\begin{figure}[h]
    \centering
    \includegraphics[width=0.45\textwidth]{images/arquitectura_2.png}
    \caption{Arquitectura de IVOA con los Protocolos y Estandares usados}
    \label{fig:ivoarch}
\end{figure}

\subsection{Metadatos de los datos de ALMA}

Para poder construir la base de datos relacional con el modelo de datos ObsCore fue
necesario mapear campos desde el ASDM.

\begin{table}[h!t]
    \centering
    \begin{tabular}{lr}
        \textbf{Campo ObsCore} & \textbf{ASDM} \\
        dataproduct\_type      & visibility \\
        calib\_level           & 1 \\
        obs\_collection        & ALMA \\
        obs\_id                & [ExecBlock.execBlockUID] \\
        obs\_publisher\_did    & [Cycle ID] \\
        access\_url            & [URL de ChiVO] \\
        access\_format         & application/x-asdm \\
        access\_estsize        & [main.dataSize] \\
        target\_name           & [Source.sourceName] \\
        s\_ra                  & [Source.direction] \\
        s\_dec                 & [Source.direction] \\
        s\_fov                 & [1.2 * lambda / Diametro antena] \\
        s\_region              & circle \\
        s\_resolution          & [1.2*lambda/(ExecBlock.baseRangeMax)] \\
        t\_min                 & [ExecBlock.startTime] \\
        t\_max                 & [ExecBlock.endTime] \\
        t\_exptime             & [main.interval] \\
        t\_resolution          & [mainTable.interval] \\
        em\_min                & [ExecBlock.baseRangeMin] \\
        em\_max                & [ExecBlock.baseRangeMax] \\
        em\_res\_power         & [spectralWindow.resolution] \\
        o\_ucd                 & em.mm \\
        pol\_states            & [Source.stokesParameter[numStokes]] \\
        facility\_name         & ALMA \\
        instrument\_name       & ALMA \\
    \end{tabular}
    \caption{Campos del ObsCore y origen desde ASDM}
    \label{table:obsasdm}
\end{table}

% Re-redactar...
En la Tabla \ref{table:obsasdm} se muestra el resultado de la investigación,
a la izquierda se despliegan las columnas de la clase Observation,
la segunda columna indica de donde se obtienen los datos para asignar la los campos
de la primera columna para el caso de los ASDM.

Para poder llenar los campos de la clase Observation es necesario escribir una
rutina capaz de operar sobre las tablas del ASDM (XML).
Actualmente existen múltiples herramientas en el Paquete de Aplicaciones de Software
Comunes de Astronomía (CASA, debido a sus siglas en inglés) \textbf{ref}.

\subsection{Tecnologías utilizadas}

Para el desarrollo de ChiVO se evaluaron distintas herramientas posibles de las
cuales se concluyó en cada capa:

\textbf{Endpoint}

Los framework que se evaluaron para la implementación del endpoint fueron:

\begin{description}
    \item[Ruby on Rails (RoR):] \hfill \\
        Framework de desarrollo web ampliamente utilizado el día de hoy,
        se basa en el concepto Modelo-Vista-Controlador (MVC).
        Sin embargo, ésta herramienta no será utilizada debido
        a que muchas funcionalidades no son necesarias para el presente proyecto.
    \item[Python/Flask:] \hfill \\
        Flask es un microframework diseñado especialmente para servicios y
        herramientas web pequeñas.
        La presente solución provee un marco de trabajo para la creación de
        aplicaciones web que puedan ser accedidas mediante distintos métodos HTTP.
        Existe mucha documentación y comunidad activa que permite implementar y
        solucionar problemas de forma rápida.
\end{description}

\textbf{ALMA Resource}

Dentro de los \emph{toolkits} de DAL recomendados por IVOA,
se probaron y verificaron los siguientes:

\begin{description}
    \item[SAADA:] \hfill \\
        Desarrollado por el VO Francés, es una herramienta bastante útil del punto
        de vista del usuario del sistema.
        Posee excelente documentación y un conveniente proceso de instalación.
        Está desarrollado en Java y su correspondiente despliegue se lleva a cabo
        mediante Tomcat.
        Es posible configurar servicios SCS/SIA/SSA/TAP y no es un proyecto
        OpenSource.

    \item[VO-Dance:] \hfill \\
        Desarrollado por el VO Italiano, es una herramienta en Java en su Backend,
        y Python en su Frontend (Framework Django).
        Lo destacable de esta herramienta es que trabaja usando MySQL como motor de
        base de datos principal, y de acuerdo a las últimas noticias relacionadas
        a su desarrollo, podría existir un soporte para PostgreeSQL en el futuro.
        La herramienta no es OpenSource y la documentación es precaria debido
        a que aún está en desarrollo. Compatible con servicios SCS/SIA/SSA/TAP.

    \item[openCADC:] \hfill \\
        Desarrollado por el VO Canadiense, es una herramienta OpenSource escrita en
        Java, utilizada actualmente en el ALMA Science Archive.
        Este toolkit es uno de los más robustos, contiene distintos paquetes con
        servicios a ser utilizados en el webservice, sin embargo posee una
        documentación precaria, lo que es compensado por abierta comunidad de
        desarrollo. Es posible configurar servicios TAP.

    \item[DaCHS:] \hfill \\
        Desarrollado por el VO Alemán, es una herramienta OpenSource escrita en
        Python.
        Es uno de los toolkits DAL más usados por los VO, ya que posee una amplia
        documentación de instalación y configuración.
        Es posible configurar servicios SCS/SIA/SSA/TAP.
\end{description}

%\vspace{0.5cm}
\begin{table*}[h!t]
\centering
\begin{tabular}{lrrrrr}
    {\bf Toolkits} & {\bf Lenguaje} & {\bf OpenSource} & {\bf Documentación} & {\bf Servicios} & {\bf Último update}  \\
    SAADA          & Java           & No               & Si                  & SCS/SIA/SSA/TAP & Mayo 2012     \\
    VO-Dance       & Java/Python    & No               & No                  & SCS/SIA/SSA/TAP & Dicimbre 2012 \\
    openCADC       & Java           & Si               & No                  & TAP             & ---           \\
    DaCHS          & Python         & Si               & Si                  & SCS/SIA/SSA/TAP & Junio 2013    \\
\end{tabular}
\caption{Resumen de los toolkits en tabla comparativa}
\label{table:toolkits}
\end{table*}

\textbf{Interfaz Usuario}

Inicialmente la interfaz usuario o frontend poseería solo vistas, por lo que
el desarrollo podía ser en prácticamente cualquier lenguaje o framework, como por
ejemplo HTML+PHP, Django o RoR. Sin embargo con los requerimientos de la
plataforma, especialmente el de capa de usuarios, se decidió escoger un
framework MVC que fuese lo suficientemente ágil y compatible con el resto de
servicios, RoR.
