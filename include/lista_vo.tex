\section{Lista of los observatorios virtuales de IVOA y sus proyectos}
\subsection{América}
\subsubsection{Brazilian Virtual Observatory (BRAVO)}
BRAVO, es una iniciativa que empezó formalmente por medio de una declaración de
intenciones de participar en agosto del 2006, por los representantes legales de
seis institutos de investigación brazileños, más la sociedad astronómica de
Brazil. Algunos proyectos: 
	\begin{itemize}
		\item BRAVO @Instituto de Astronomía, Geofísica y Ciencias Atmosféricas:
			\begin{itemize}
				\item StarLight: es un mirror del webservice Starlight alojado en \url{www.starlight.ufsc.br}
				\item GALExtin: es una herramienta que permite determinar la extinción a lo largo de una línea de visión.
				\item ZPhot: herramienta de redshifts fotométricos.
				\item RedSpec: herramienta de reddening y reddening.
				\item Be Atlas: esta herramienta pone a disposición de la comunidad 70.000 espectros sintéticos de Be stars.
			\end{itemize}
		\item BRAVO @Instituto Nacional de Investigación Espacial de Brasil: Generar inversiones en tecnología de información en infraestuctura computacinal, Grid de datos, Procesamiento de datos y 
				Minería de datos.
		%\begin{itemize}
		%	\item Descripción: generate investment in information technology on Computational Infraestructure, Date Grid, Data Processing and Data Mining.
		%\end{itemize}
		\item BRAVO @Laboratorio Nacinal de Astrofísica: creando un observatorio virtual dedicado al data del Southern Astrophysical Research Telescope (SOAR), para los astrónomos brasileños.
		%\begin{itemize}
		%	\item Descripción: Making of a virtual observatory dedicated to Southern Astrophysical Research Telescope (SOAR) data from Brazilian astronomers.
		%\end{itemize}
		\item BRAVO @Universidad Federal de Santa Catarina: investigando en la síntesis espectral de potencia como un medio para estimar las propiedades físicas de las galaxias.
		%\begin{itemize}
		%	\item Descripción: researching of the of the power spectral synthesis as a mean to estimate the physical properties of the galaxies.
		%\end{itemize}
		\item Cyclotron Emission of Polars (CYCLOPS): han desarrollado un nuevo código, para modelar la emisión óptica de estos sistemas, incluyendo los cuatros parámetros de Stokes.
	\end{itemize}
\subsubsection{Canadian Virtual Observatory (CVO)}
	\begin{itemize}
		\item Proyectos
		\begin{itemize}
			\item Data Sharing (VOSpace 2.0)
			\begin{itemize}
				\item Descripción: a service which allows users to share files and collaborate with team members.  
			\end{itemize}
			\item Table Access Protocol (TAP-1.0)
			\begin{itemize}
				\item Descripción: a service which allows the access to all the data described by the Common Archive Observation Model (CAOM) in use at the CADC and tables from other projects.
			\end{itemize}
			\item Observation Model Core Components (ObsCore-1.0)
			\begin{itemize}
				\item Descripción: a model which implements a standard view for \textbf{Table Access Protocol (TAP-1.0)}.
			\end{itemize}
			\item Simple Image Access (SIA-1.0)
			\begin{itemize}
				\item Descripción: a SIA-1.0 compliant query service for easy access to calibrated images from most our data collections.
			\end{itemize}
		\end{itemize}
	\end{itemize}

\subsubsection{Nuevo Observatorio Virtual Argentino (NOVA)}
	\begin{itemize}
		\item Fundadores: Observatorio Astronómico de Córdova (OAC),
			Facultad de Ciencias Astronómicas y Geofísicas de La Plata/Universidad de
			Nacional de la Plata (FCAGLP/UNLP), Instituto de Astrofísica de La Plata
			(IALP), Instituto Argentino de Radioastronomía (IAR), Instituto de Astronomía y
			Física del Espacio (IAFE), Instituto de Ciencias Astronómicas, de la Tierra y
			del Espacio (ICAFE), Instituto de Astronomía Teórica y Experimental (IATE),
			Complejo Astronómico El Leoncito (CASLEO).
		\item Comienzo: January, 2009
		\item Proyectos
		\begin{itemize}
			\item NOVA@CASLEO
			\begin{itemize}
				\item Estado: active.
				\item Descripción: not obtained for now. Not available at NOVA's website or similar.
			\end{itemize}
			\item NOVA@IAFE
			\begin{itemize}
				\item Estado: unknown.
				\item Descripción: building a database for the spectroscopic observations available at ICATE.
				\item Outstanding: until 1987, the database was stored in photographic plates. After that year, the information was stored in CDs and DVDs.
			\end{itemize}
			\item NOVA@ICATE
			\begin{itemize}
				\item Estado: unknown.
				\item Descripción: building a database for the spectroscopic observations available at ICATE.
				\item Outstanding: until 1987, the database was stored in photographic plates.
					After that year, the information was stored in CDs and DVDs.
			\end{itemize}
			\item NOVA@OAC
			\begin{itemize}
				\item Estado: active.
				\item Descripción: not obtained for now. Not available at NOVA's website or similar.
			\end{itemize}
			\item NOVA@FCAGLP, NOVA@IALP, NOVA@IAR, NOVA@IATE are referred in the website but without status and description.
		\end{itemize}
	\end{itemize}

\subsubsection{US Virtual Astronomical Observatory (VAO)}
	\begin{itemize}
		\item Fundadores: NSF, NASA.
		\item Proyectos
		\begin{itemize}
			\item Data Discovery Tool
			\begin{itemize}
				\item Descripción: 
			\end{itemize}
			\item Iris: SED Analysis Tool
			\begin{itemize}
				\item Descripción: 
			\end{itemize}
			\item Cross-Comparision Tool
			\begin{itemize}
				\item Descripción: 
			\end{itemize}
			\item Time Series Search Tool
			\begin{itemize}
				\item Descripción: 
			\end{itemize}
		\end{itemize}
	\end{itemize}

\subsection{Europe}
\subsubsection{Armenian Virtual Observatory}
	\begin{itemize}
		\item Proyectos
	\end{itemize}

\subsubsection{Hungarian Virtual Observatory (HVO)}
	\begin{itemize}
		\item Proyectos
	\end{itemize}

\subsubsection{AstroGrid}
	\begin{itemize}
		\item Comienzo: 2001
		\item Fundadores: PPARC, STFC
		\item Proyectos
		\begin{itemize}
			\item VODesktop
			\begin{itemize}
				\item Descripción: an analysis tools wich allows limit the choice of resources through specific data saving.
			\end{itemize}
			\item Astro Runtime (AR)
			\begin{itemize}
				\item Descripción: an API implemented in JAVA wich facilitates the access to the \textbf{VODesktop} services from any programming language.
			\end{itemize}
		\end{itemize}
	\end{itemize}

\subsubsection{European Space Agency Virtual Observatory (ESA-VO)}
	\begin{itemize}
		\item Proyectos
	\end{itemize}

\subsubsection{European Virtual Observatory (EURO-VO)}
	\begin{itemize}
		\item
	\end{itemize}

\subsubsection{German Astrophysical Virtual Observatory (GAVO)}
	\begin{itemize}
		\item Proyectos
		\begin{itemize}
			\item GAVO Data Center
			\begin{itemize}
				\item Descripción: A growing collection of data and services provided on behalf of third parties. Some of the GAVO services are also available on http://dc.zah.uni-heidelberg.de/
			\end{itemize}
			\item MPA Simulations access
			\begin{itemize}
				\item Descripción: a web service for querying the results of the Millennium simulation using SQL.
			\end{itemize}
			\item MultiDark Database
			\begin{itemize}
				\item Descripción: a service wich gives access to data from MultiDark and Bolshoi simulations using SQL queries. It based on the Millennium Web Application.
			\end{itemize}
			\item RAVE archive search
				\begin{itemize}
				\item Descripción: an access to a growing archive of radial velocities for more than 400 000 stars.
			\end{itemize}
			\item TheoSSA
				\begin{itemize}
				\item Descripción: a service for providing spectral energy distributions based on model atmosphere calculations.
			\end{itemize}
		\end{itemize}
	\end{itemize}

\subsubsection{Observatoire Virtuel France (VO-France)}
	\begin{itemize}
		\item Proyectos
	\end{itemize}

\subsubsection{Spanish Virtual Observatory (SVO)}
	\begin{itemize}
		\item Proyectos
		\begin{itemize}
			\item VOSA
			\begin{itemize}
				\item Descripción: 
			\end{itemize}
			\item VOSED
			\begin{itemize}
				\item Descripción: 
			\end{itemize}
			\item TESELA
			\begin{itemize}
				\item Descripción: 
			\end{itemize}
			\item Filter Profile Service
			\begin{itemize}
				\item Descripción: 
			\end{itemize}
		\end{itemize}
	\end{itemize}

\subsubsection{Italian Virtual Observatory (VObs.it)}
	\begin{itemize}
		\item Comienzo: 2005
		\item Fundadores: INAF
		\item Proyectos
		\begin{itemize}
			\item SIAP
			\begin{itemize}
				\item Descripción: 
			\end{itemize}
			\item SSAP
			\begin{itemize}
				\item Descripción: 
			\end{itemize}
			\item CONE SEARCH
			\begin{itemize}
				\item Descripción: 
			\end{itemize}
			\item SKYNODE
			\begin{itemize}
				\item Descripción: 
			\end{itemize}
		\end{itemize}
	\end{itemize}

\subsection{Asia}
\subsubsection{Chinese Virtual Observatory}
	\begin{itemize}
		\item Proyectos
	\end{itemize}

\subsubsection{Japanese Virtual Observatory}
	\begin{itemize}
		\item Proyectos
	\end{itemize}

\subsubsection{Russian Virtual Observatory}
	\begin{itemize}
		\item Proyectos
	\end{itemize}

\subsubsection{Virtual Observatory India}
	\begin{itemize}
		\item Fundadores: Persistent Systems Ltd.
		\item Colaboradores: Inter-University Centre for Astronomy and Astrophysics (IUCAA).
		\item Soporte: Ministry of Communication and Information Technology, Government of India.
		\item Comienzo: January, 2009
		\item Proyectos
		\begin{itemize}
			\item VOIPortal
			\item Mosaic Service
			\item PyMorph Service
			\item VOPlot
			\item VOMegaPlot (Client-Server Version)
			\item AstroStat
			\item VOCat
			\item VOPlatform
			\item VOConvert (ConVOT)
			\item Android Cosmological Calculator
			\item Android Name Resolver
			\item CSharpFITS Package
			\item VOTable JAVA Streaming Writer
			\item C++ parser for VOTable
			\item Fits Manager
			\item HCT Data Archival Sys
		\end{itemize} 	
	\end{itemize}

