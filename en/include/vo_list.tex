\section{List of IVOA Virtual Observatories}
\subsection{America}
\subsubsection{Brazilian Virtual Observatory (BRAVO)}
The BRAVO was born with the Declarations of Intentions signed on
18th of August, 2008 by six research institutes and the Brazilian Astronomical
Societyi (SBA, in its Portuguese acronym). Later, the Brazilian Virtual
Observatory was founded by the National Institute for Science and Technology in
Astrophysics (INCT-A, in its Portugese acronym).

\begin{itemize}
	\item \textbf{Projects}
	\begin{itemize}
		\item BRAVO@IAG
		\item BRAVO@INPE
			\begin{itemize}
				\item \textbf{Description\footnote{On the \textbf{List of IVOA
Virtual Observatories} section, a brief description of each proyect appears
only if was found some information of them in the sources.}:} generate
investment in information technology on Computational Infraestructure, Data
Grid, Data Processing and Data Mining.
			\end{itemize}
		\item BRAVO@LNA
			\begin{itemize}
				\item \textbf{Description:} making of a virtual observatory
dedicated to Southern Astrophysical Research Telescope (SOAR) data from
Brazilian astronomers.  
			\end{itemize}
		\item BRAVO@UFSC
			\begin{itemize}
				\item \textbf{Description:} researching of the of the power
spectral synthesis as a mean to estimate the physical properties of the
galaxies.
			\end{itemize}
		\item CYCLOPS
		\begin{itemize}
			\item \textbf{Description:} software which models the optical
emission from AM Her systems including the four Stokes parameters.
		\end{itemize}
	\end{itemize}
\end{itemize}

\subsubsection{Canadian Virtual Observatory (CVO)}
\begin{itemize}
	\item \textbf{Projects}
	\begin{itemize}
		\item Data Sharing (VOSpace 2.0)
			\begin{itemize}
				\item \textbf{Description:} a service which allows users to
share files and collaborate with team members.
			\end{itemize}
		\item Table Access Protocol (TAP-1.0)
			\begin{itemize}
				\item \textbf{Description:} a service which allows the access
to all the data described by the Common Archive Observation Model (CAOM) in use
at the CADC and tables from other projects.
			\end{itemize}
		\item Observation Model Core Components (ObsCore-1.0)
			\begin{itemize}
				\item \textbf{Description:} a model which implements a standard
view for \textbf{Table Access Protocol (TAP-1.0)}.
			\end{itemize}
		\item Simple Image Access (SIA-1.0)
			\begin{itemize}
				\item \textbf{Description:} a SIA-1.0 compliant query service
for easy access to calibrated images from most our data collections.
			\end{itemize}
	\end{itemize}
\end{itemize}

\subsubsection{Nuevo Observatorio Virtual Argentino (NOVA)}
The NOVA was founded by eight institutions\footnote{The institutions which
founded the NOVA are the Observatorio Astron\'{o}mico de C\'{o}rdova (OAC),
Facultad de Ciencias Astron\'{o}micas y Geof\'{i}sicas de La Plata/Universidad
de Nacional de la Plata (FCAGLP/UNLP), the Instituto de Astrof\'{i}sica de La
Plata (IALP), the Instituto Argentino de Radioastronom\'{i}a (IAR), the
Instituto de Astronom\'{i}a y F\'{i}sica del Espacio (IAFE), the Instituto de
Ciencias Astron\'{o}micas, de la Tierra y del Espacio (ICAFE), the Instituto de
Astronom\'{i}a Te\'{o}rica y Experimental (IATE), and the Complejo
Astron\'{o}mico El Leoncito (CASLEO).} among which important astronomical
institutes and the National University of La Plata through the Faculty of
Astronomical Sciences and Geophysics of La Plata. It was born in January 2009.
From June 2013, the NOVA will begin its operations and intends, in addition to
provide the astronomical observations from its official website, to implement a
platform web where it can work with the data\footnote{Agencia CTyS. (2013, May
9). Instituciones astron\'{o}micas lanzan el Nuevo Observatorio Virtual
Argentino. \textit{Agencia CTyS}. Retrieved from:
\url{http://www.ctys.com.ar/index.php?idPage=20&idArticulo=2585}}.

\begin{itemize}
	\item \textbf{Projects}
	\begin{itemize}
		\item NOVA@CASLEO
		\item NOVA@IAFE
			\begin{itemize}
				\item \textbf{Description:} building a database for the
observations\footnote{The more than 6 terabytes of date was stored in CDs and
DVDs.} reached by the HASTA solar telescope and its applications.
			\end{itemize}
		\item NOVA@IALP
		\item NOVA@IAR
		\item NOVA@IATE
		\item NOVA@ICATE
			\begin{itemize}
				\item \textbf{Description:} building a database for the
spectroscopic observations\footnote{Until 1987, the database was stored in
photographic plates. After that year, the information was stored in CDs and
DVDs.} available at ICATE.
			\end{itemize}
		\item NOVA@OAC
		\item NOVA@FCAGLP
	\end{itemize}
\end{itemize}

\subsubsection{US Virtual Astronomical Observatory (VAO)}
The VAO is the succesor of the NVO (National Virtual Observatory) and was
founded by the NSF and the NASA and. It is in charge of the VAO, LLC, an entity
created by the Associated Universities, Inc. (AUI) and the Association of
Universities for Research in Astronomy (AURA). VAO advise the VAO Science
Council \footnote{\url{http://www.usvao.org/governance/}}. The US VO is a
co-founder of the IVOA.

\begin{itemize}
	\item \textbf{Projects}
	\begin{itemize}
		\item Data Discovery Tool
			\begin{itemize}
				\item \textbf{Description:}
			\end{itemize}
		\item Iris: SED Analysis Tool
			\begin{itemize}
				\item \textbf{Description:}
			\end{itemize}
		\item Cross-Comparision Tool
			\begin{itemize}
				\item \textbf{Description:}
			\end{itemize}
		\item Time Series Search Tool
			\begin{itemize}
				\item \textbf{Description:}
			\end{itemize}
	\end{itemize}
\end{itemize}

\subsection{Europe}
\subsubsection{Armenian Virtual Observatory (ArVO)}
The ArVo is based on the Digital First Byukaran Survey (DFBS), a project
between Byurakan Astrophysical Observatory, Armenia; ``La Sapienza''
Universit\`{a} di Roma, Italia; Cornell University, USA and
VO-France\footnote{Mickaelian, A., Sargsyan, L., Gigoyan, K., Erastova, L.,
Sinamyan, P., Hovhannisyan, L., ...Mykayelyan, G. (2007, December). Science
with the Armenian Virtual Observatory (ArVo).  Retrieved from
\url{http://www.grid.am/pdf/Science_with_the_Armenian_Virtual_Observatory_(ArVO).pdf}}.
Its virtual observatory was launched in February 2008\footnote{Armenian
News-NEWS.am. (2012, February 18). Armenia creates virtual observatory server.
\textit{NEWS.am}. Retrieved from \url{http://news.am/eng/news/93843.html}}.

\begin{itemize}
	\item \textbf{Projects}
\end{itemize}

\subsubsection{Hungarian Virtual Observatory (HVO)}
\begin{itemize}
	\item \textbf{Projects}
    \begin{itemize}
        \item Spectrum Service for VO
            \begin{itemize}
                \item \textbf{Description:}
            \end{itemize}
        \item Synthetic Spectrum Service
            \begin{itemize}
                \item \textbf{Description:}
            \end{itemize}
        \item Photometric Redshift
            \begin{itemize}
                \item \textbf{Description:}
            \end{itemize}
        \item Linking WebServices to GRID clusters
            \begin{itemize}
                \item \textbf{Description:}
            \end{itemize}
        \item Electronic journals in the VO
            \begin{itemize}
                \item \textbf{Description:}
            \end{itemize}
        \item Solar observations in the VO
            \begin{itemize}
                \item \textbf{Description:}
            \end{itemize}
     \end{itemize}
\end{itemize}

\subsubsection{AstroGrid}
The AstroGrid is the United Kingdom's virtual observatory. It began as a
project in 2001 and was launched in April 2008 along its working service and
user software. It has been financed by the Particle Physics and Astronomy and
Research Council (PPARC) and the Science \& Technology Facilities Council
(STFC).

\begin{itemize}
	\item \textbf{Projects}
	\begin{itemize}
		\item Topcat
			\begin{itemize}
				\item \textbf{Description:} an interactive graphical viewer and
editor for tabular data for formats like FITS and VOTable.
			\end{itemize}
		\item VODesktop
			\begin{itemize}
				\item \textbf{Description:} an analysis tools wich allows limit
the choice of resources through specific data saving.
			\end{itemize}
		\item AstroRuntime
			\begin{itemize}
				\item \textbf{Description:} an API implemented in JAVA wich
facilitates the access to the \textbf{VODesktop} services from almost any
programming language \footnote{On the AstroGrid's official website there is a
document about how access VODesktop using Python script at
\url{http://www.astrogrid.org/agpython.html}}.
			\end{itemize}
	\end{itemize}
\end{itemize}

\subsubsection{European Space Agency Virtual Observatory (ESA-VO)}
\begin{itemize}
	\item \textbf{Projects}
\end{itemize}

\subsubsection{European Virtual Observatory (EURO-VO)}
The EURO-VO is the continuation of Astrophysical Virtual Observatory (AVO). The
AVO project conducted was conceived by the European Commision and six
organizations \footnote{The six organizations that founded the AVO togheter with
the European Commision are the European Southern Observatory (ESO), the European
Space Agency (ESA), the AstroGrid, the CNRS-supported Centre de Données
Astronomiques de Strasbourg (CDS), the University Louis Pasteur; the
CNRS-supported TERAPIX astronomical data centre, the Institut d'Astrophysique;
and the Jodrell Bank Observatory, the Victoria University of Manchester.} to
research about the scientific requirements and technologies necessary to build
an Euopean virtual observatory. The ``EURO-VO aims at deploying and operational
VO in Europe''\footnote{\url{http://www.euro-vo.org/}}.\\

\begin{itemize}
    \item \textbf{Projects}
    \begin{itemize}
        \item EuroVO-CoSADIE
            \begin{itemize}
                \item \textbf{Description:}
            \end{itemize}
        \item VOTECH
             \begin{itemize}
                \item \textbf{Description:}
             \end{itemize}
        \item EuroVO-DCA
             \begin{itemize}
                \item \textbf{Description:}
             \end{itemize}
        \item EuroVO-AIDA
             \begin{itemize}
                \item \textbf{Description:}
             \end{itemize}
         \item EuroVO-ICE
             \begin{itemize}
                \item \textbf{Description:}
             \end{itemize}
    \end{itemize}
\end{itemize}

\subsubsection{German Astrophysical Virtual Observatory (GAVO)}
The GAVO was launched in 2003\footnote{Its first publications dates from 2003.
In 2004, H. Adorf and GAVO Team talking about ``GAVO - after one year'' at the
Astronomical Data Analysis (ADA) III Sant'Agata sui due Golfi, Italy. The birth
of this VO is inferred from the above.}. It is financed through the Federal
Ministry of Education and Research (BMBF).

\begin{itemize}
	\item \textbf{Projects}
	\begin{itemize}
		\item GAVO Data Center
			\begin{itemize}
				\item \textbf{Description:} A growing collection of data and
services provided on behalf of third parties. Some of the GAVO services are
also available on \url{http://dc.zah.uni-heidelberg.de/}
			\end{itemize}
		\item GAVO Data Center
			\begin{itemize}
				\item \textbf{Description:} a collection of data and services
on behalf of third parties.
			\end{itemize}
		\item MPA Simulations access
			\begin{itemize}
				\item \textbf{Description:} a web service for querying the
results of the Millennium simulation using SQL.
			\end{itemize}
		\item MultiDark Database
			\begin{itemize}
				\item \textbf{Description:} a service wich gives access to data
from MultiDark and Bolshoi simulations using SQL queries.  It based on the
Millennium Web Application.
			\end{itemize}
		\item RAVE archive search
			\begin{itemize}
				\item \textbf{Description:} an access to a growing archive of
radial velocities for more than 400 000 stars.
			\end{itemize}
		\item TheoSSA
			\begin{itemize}
				\item \textbf{Description:} a service for providing spectral
energy distributions based on model atmosphere calculations.
			\end{itemize}
	\end{itemize}
\end{itemize}

\subsubsection{Observatoire Virtuel France (VO-France)}
\begin{itemize}
	\item \textbf{Projects}
\end{itemize}

\subsubsection{Spanish Virtual Observatory (SVO)}
The SVO started in June 2004 and its participants are the Centro de
Astrobiolog\'{i}a (INTA-CSIC), the Artificial Intelligence Department of the
National University of Distance Education (UNED, in its Spanish acronym), the
University of C\'{a}diz and the Center of Scientific and Academic Services of
Catalonia (CESCA, in its Spanish acronym).

\begin{itemize}
	\item \textbf{Projects}
	\begin{itemize}
		\item VOSA
			\begin{itemize}
				\item \textbf{Description:}
			\end{itemize}
		\item VOSED
			\begin{itemize}
				\item \textbf{Description:}
			\end{itemize}
		\item TESELA
			\begin{itemize}
				\item \textbf{Description:}
			\end{itemize}
		\item Filter Profile Service
			\begin{itemize}
				\item \textbf{Description:}
			\end{itemize}
	\end{itemize}
\end{itemize}

\subsubsection{Italian Virtual Observatory (VObs.it)}
The VObs.it besides being member of IVOA, it is a member of EURO-VO. It was
established and founded by the Italian National Institute for Astrophysics
(INAF) and it is coordinated by the Information Systems Units (SI) of this. 

\begin{itemize}
\item \textbf{Projects}
    \begin{itemize}
        \item SIAP
            \begin{itemize}
                \item \textbf{Description:} 
            \end{itemize}
        \item SSAP
            \begin{itemize}
                \item \textbf{Description:} 
            \end{itemize}
        \item CONE SEARCH
            \begin{itemize}
                \item \textbf{Description:} 
            \end{itemize}
        \item SKYNODE
            \begin{itemize}
                \item \textbf{Description:} 
            \end{itemize}
    \end{itemize}
\end{itemize}

\subsection{Asia}
\subsubsection{Chinese Virtual Observatory (China-VO)}
The China-VO was initiated in 2002 by the National Astronomical Observatories,
Chinese Academy of Sciences. It has four partners\footnote{The partners of
China-VO are the National Astronomical Observatories, Chinese Academy of
Sciences (NAOC); the TianJin University (TJU), the Central China Normal
University (CCNU), Kunming University of Science and Technology.} and more of
twelve collaborators\footnote{The China-VO's collaborators are the Computer
Network Information Center, the Purple Mountain Astro Obs, the Shangai Astro
Obs, the Yunnan Astro Obs, the Tsinghua University, the JHU, MSR, Caltech,
IUCAA, CDS, ICRAR (Australia), NAOJ (Japan), among others.}.

\begin{itemize}
	\item \textbf{Projects}
\end{itemize}

\subsubsection{Japanese Virtual Observatory (JVO)}
The JVO was implemented by the National Astronomical Observatory
of Japan (NAOJ) with collaboration of Fujitsu for the development of JVO
prototype systems.

\begin{itemize}
	\item \textbf{Projects}
\end{itemize}

\subsubsection{Russian Virtual Observatory (RVO)}
A reason for the construction of the RVO is that majority of observatories were
south of Ex-Soviet Union. After of desintegration of Ex-URSS, they were located
in the territories outside of
Russia\footnote{\url{http://www.inasan.rssi.ru/eng/rvo/project.html}}.

\begin{itemize}
\item \textbf{Projects\footnote{\url{http://synthesis.ipi.ac.ru/synthesis/projects}}}
	\begin{itemize}
		\item SYNTHESIS
			\begin{itemize}
				\item \textbf{Description:} SYNTHESIS group's framework
project.
			\end{itemize}
		\item INFOSEM
			\begin{itemize}
				\item \textbf{Description:}
			\end{itemize}
		\item SEMIMOD
			\begin{itemize}
				\item \textbf{Description:} Modelling and management of
semi-structured data for dynamic World Wide Web applications.
			\end{itemize}
		\item BIOMED
			\begin{itemize}
				\item \textbf{Description:} Methods and tools for development
of subject mediators of he\-te\-ro\-ge\-neous information collections for
distributed digital libraries.
			\end{itemize}
		\item REFINE
			\begin{itemize}
				\item \textbf{Description:}
			\end{itemize}
		\item VOINFRA
			\begin{itemize}
				\item \textbf{Description:} Devolopment of principles and
fundamentals of the information interoperability in the infraestructure of the
RVO.
			\end{itemize}
		\item MULTISOURCE
			\begin{itemize}
				\item \textbf{Description:} Methods for organization of
problems solving over multiple distributed he\-te\-ro\-ge\-neous information
sources.
			\end{itemize}
		\item RVOAG
			\begin{itemize}
				\item \textbf{Description:}
			\end{itemize}
		\item ASTROMEDIA
			\begin{itemize}
				\item \textbf{Description:} Methods and tools for supporting
subject mediators architecture in
AstroGrid\footnote{\url{http://www.astrogrid.org/}} infrastructure for the RVO.
			\end{itemize}
		\item UNIMOD
			\begin{itemize}
				\item \textbf{Description:}
			\end{itemize}
		\item SEMID
			\begin{itemize}
				\item \textbf{Description:}
			\end{itemize}
		\item SubjMed
			\begin{itemize}
				\item \textbf{Description:}
			\end{itemize}
		\item ConcMod
			\begin{itemize}
				\item \textbf{Description:} Development of methods and tools
for definition of scientific subject domains conceptual models and problems
solving support based on mediators subject in the hybrid grid-infrastructure.
			\end{itemize}
		\item RuleInt
			\begin{itemize}
				\item \textbf{Description:} Integration of rule-based
declarative programs and knowledge databases and services for scientific
problems solving over heterogeneus distributed information resources.
			\end{itemize}
		\item ASTROMEDIA Trial
			\begin{itemize}
				\item \textbf{Description:} Hybrid architecture of AstroGrid
and Mediator Middlewere.
			\end{itemize}
		\item Galaxies Search
			\begin{itemize}
                \item \textbf{Description:} Distant galaxy search applying
AstroGrid.
			\end{itemize}
		\item Star Classification
			\begin{itemize}
				\item \textbf{Description:} Eclipsing-binary stars
classification applying Ensembled Weka algorithm in AstroGrid.
			\end{itemize}
	\end{itemize}
\end{itemize}

\subsubsection{Virtual Observatory India (VOI)}
The VOI is a collaboration between the Inter University Center for Astronomy
and Astrophysics (IUCAA) and the Persistent Systems Ltd., and is supported by
the Ministry of Communication and Information Technology, Government of India.

\begin{itemize}
\item \textbf{Projects}
	\begin{itemize}
		\item VOIPorta
            \begin{itemize}
                \item \textbf{Description:}
            \end{itemize}
		\item Mosaic Service
            \begin{itemize}
                \item \textbf{Description:}
            \end{itemize}
		\item PyMorph Service
            \begin{itemize}
                \item \textbf{Description:}
            \end{itemize}
		\item VOPlot
            \begin{itemize}
                \item \textbf{Description:}
            \end{itemize}
		\item VOMegaPlot (Client-Server Version)
            \begin{itemize}
                \item \textbf{Description:}
            \end{itemize}
		\item AstroStat
            \begin{itemize}
                \item \textbf{Description:}
            \end{itemize}
		\item VOCat
            \begin{itemize}
                \item \textbf{Description:}
            \end{itemize}
		\item VOPlatform
            \begin{itemize}
                \item \textbf{Description:}
            \end{itemize}
		\item VOConvert (ConVOT)
            \begin{itemize}
                \item \textbf{Description:}
            \end{itemize}
		\item Android Cosmological Calculator
            \begin{itemize}
                \item \textbf{Description:}
            \end{itemize}
		\item Android Name Resolver
            \begin{itemize}
                \item \textbf{Description:}
            \end{itemize}
		\item CSharpFITS Package
            \begin{itemize}
                \item \textbf{Description:}
            \end{itemize}
		\item VOTable JAVA Streaming Writer
            \begin{itemize}
                \item \textbf{Description:}
            \end{itemize}
		\item C++ parser for VOTable
            \begin{itemize}
                \item \textbf{Description:}
            \end{itemize}
		\item Fits Manager
            \begin{itemize}
                \item \textbf{Description:}
            \end{itemize}
		\item HCT Data Archival Sys
	\end{itemize}
\end{itemize}
