\section{List of IVOA Virtual Observatories}
\subsection{North America}
\subsubsection{Canadian Virtual Observatory (CVO)}
The CVO \cite{website:cvo-home} \nocite{SchadeDowlerDurand2004} is a project
within the Canadian Astronomy Data Centre where is developing the following
projects:

\begin{itemize}
\item \textbf{Data Sharing (VOSpace 2.0)}:
a service that allows to share files and collaborate with team members.

\item \textbf{Table Access Protocol (TAP-1.0)}:
a service that allows the access to all the data
described by the Common Archive Observation Model (CAOM) in use at the CADC and
tables from other projects.

\item \textbf{Observation Model Core Components (ObsCore-1.0)}:
a model that implements a standard view for \textbf{Table Access Protocol
                                                    (TAP-1.0)}.

\item \textbf{Simple Image Access (SIA-1.0)}:
a SIA-1.0 compliant query service for easy access to calibrated images from most
our data collections.
\end{itemize}

\subsubsection{US Virtual Astronomical Observatory (VAO)}
The VAO \cite{website:vao-home} \nocite{DeYoung2010} is the succesor of the NVO
(National Virtual Observatory) and was founded by the NSF and the NASA. It is in
charge of the VAO, LLC, an entity created by the Associated Universities, Inc.
(AUI) and the Association of Universities for Research in Astronomy (AURA). VAO
advise the VAO Science Council
\footnote{\url{http://www.usvao.org/governance/}}. The US VO is a co-founder of
the IVOA. Its complete/development projects below:

\begin{itemize}
\item \textbf{Data Discovery Tool}:
a web tool that allows to find datasets from astronomical collections known to
the VO like the Hubble Space Telescope (HST), the Chandra X-ray Observatory, the
Spitzer Space Telescope, among other.

\item \textbf{Iris: SED Analysis Tool}:
a downloadable application for the finding, plotting and fitting the Spectral
Energies Distributions (SEDs). 

\item \textbf{Time Series Search Tool}:
a web tool that allows to access the time series data sets at the Harvard Time
Series Center (TSC), the NASA Exoplanet Archive and the Catalina Real-Time
Transient Survey, and analize them with the periodogram application of the NASA
Exoplanet Archive.

\item \textbf{Cross-Comparision Tool}:
a web tool that performs croos-comparisons between one table supplied by the
user and other of an online source catalog, for a user-specified match radius.
This returns the all sources in the online catalog that are within the radius.
\end{itemize}

\subsection{South America}
\subsubsection{Brazilian Virtual Observatory (BRAVO)}
The BRAVO \cite{website:bravo-home} was born with the Declarations of Intentions
signed on 18th of August, 2008 by six research institutes and the Brazilian
Astronomical Society (SBA, in its Portuguese acronym). Later, the Brazilian
Virtual Observatory was founded by the National Institute for Science and
Technology in Astrophysics (INCT-A, in its Portugese acronym). Its
complete/development projects below:

\begin{itemize}
\item \textbf{BRAVO@IAG}

\item \textbf{BRAVO@INPE}:
generate investment in information technology on Computational Infraestructure,
Data Grid, Data Processing and Data Mining.

\item \textbf{BRAVO@LNA}:
making of a virtual observatory dedicated to Southern Astrophysical Research
Telescope (SOAR) data from Brazilian astronomers.  

\item \textbf{BRAVO@UFSC}:
researching of the of the power spectral synthesis as a mean to estimate the
physical properties of the galaxies.

\item \textbf{CYCLOPS}:
a software that models the optical emission from AM Her systems including the
four Stokes parameters.
\end{itemize}

\subsubsection{Chilean Virtual Observatory (ChiVO)}
The ChiVO \cite{website:chivo-home} was accepted as IVOA's member on September
25th, 2013. The project developed in Chile was born as need to archive,
developed new analysis tools and better intelligent processing algorithms for
large volumes of astronomical data. It is an initiative of five
universities\footnote{Universidad T\'{e}cnica Federico Santa Mar\'{i}a,
Universidad de Chile, Pontificia Universidad Cat\'{o}lica de Chile, Universidad
de Concepci\'{o}n, Universidad de Santiago de Chile} and is supported by ALMA
\cite{webiste:alma-home} and REUNA \cite{website:reuna-home}. The VO intends
mainly to work with the Atacama Large Millimeter/submillimeter Array (ALMA)'s
data that daily is about 1 terabytev. Its complete/development projects below:

\begin{itemize}
\item \textbf{Automatic Astronomical Different Scales Structures Detection and
Classification within Astronomical Images}:
a magister thesis that intends to make a software tool that finds directly
astronomical objects within astronomical images through the wavelet mathematical
tool and a machine learning system.

\item \textbf{Conceptual Design of a Virtual Astronomical Observatory for ALMA}:
a degree thesis that through the researching and analysis of query languages,
formats and the semantic of OVA (in its Spanish acronym), and the definition of
it intends to make a conceptual design for the ALMA observatory.

\item \textbf{Indexing of Astronomical Objects}:
a degree thesis that intends to design and implement an software tool that allow
make an R-tree index of FITS astronomical images based on their celestial
coordinates.

\item  \textbf{Conceptual Design for the Search for Astronomical Patterns}:
 a degree thesis that through the researching and analysis o search and
recognition methods of patterns intends to minimize the search space with
minimal impact on the results. 
\end{itemize}

\subsubsection{Nuevo Observatorio Virtual Argentino (NOVA)}
The NOVA \cite{website:nova-home} was founded by eight institutions\footnote{The
institutions that founded the NOVA are the Observatorio Astron\'{o}mico de
C\'{o}rdova (OAC), the Facultad de Ciencias Astron\'{o}micas y Geof\'{i}sicas de
La Plata/Universidad de Nacional de la Plata (FCAGLP/UNLP), the Instituto de
Astrof\'{i}sica de La Plata (IALP), the Instituto Argentino de
Radioastronom\'{i}a (IAR), the Instituto de Astronom\'{i}a y F\'{i}sica del
Espacio (IAFE), the Instituto de Ciencias Astron\'{o}micas, de la Tierra y del
Espacio (ICAFE), the Instituto de Astronom\'{i}a Te\'{o}rica y Experimental
(IATE), and the Complejo Astron\'{o}mico El Leoncito (CASLEO).} among which
important astronomical institutes and the National University of La Plata
through the Faculty of Astronomical Sciences and Geophysics of La Plata. It was
born in January 2009.  From June 2013, the NOVA will begin its operations and
intends, in addition to provide the astronomical observations from its official
website, to implement a platform web where it can work with the
data\footnote{Agencia CTyS. (2013, May 9). Instituciones astron\'{o}micas lanzan
el Nuevo Observatorio Virtual Argentino. \textit{Agencia CTyS}. Retrieved from:
\url{http://www.ctys.com.ar/index.php?idPage=20&idArticulo=2585}}. Its
complete/development projects below:

\begin{itemize}
\item \textbf{NOVA@CASLEO}

\item \textbf{NOVA@IAFE}:
building a database for the observations\footnote{The more than 6 terabytes of
date was stored in CDs and DVDs.} reached by the H-Alpha Solar Telescope for
Argentina (HASTA) solar telescope and its applications.

\item \textbf{NOVA@IALP}

\item \textbf{NOVA@IAR}

\item \textbf{NOVA@IATE}

\item \textbf{NOVA@ICATE}:
building a database for the spectroscopic observations\footnote{Until 1987, the
database was stored in photographic plates. After that year, the information was
stored in CDs and DVDs.} available at ICATE.

\item \textbf{NOVA@OAC}

\item \textbf{NOVA@FCAGLP}
\end{itemize}

\subsection{Europe}
\subsubsection{Armenian Virtual Observatory (ArVO)}
The ArVo \cite{website:arvo-home} is based on the Digital First Byurakan Survey
(DFBS), a project between Byurakan Astrophysical Observatory, Armenia; ``La
Sapienza'' Universit\`{a} di Roma, Italia; Cornell University, USA and
VO-France\footnote{Mickaelian, A., Sargsyan, L., Gigoyan, K., Erastova, L.,
Sinamyan, P., Hovhannisyan, L., ...Mykayelyan, G. (2007, December). Science with
the Armenian Virtual Observatory (ArVo).  Retrieved from
\url{http://www.grid.am/pdf/Science_with_the_Armenian_Virtual_Observatory_(ArVO).pdf}}.
Its virtual observatory was launched in February 2008\footnote{Armenian
News-NEWS.am. (2012, February 18). Armenia creates virtual observatory server.
\textit{NEWS.am}. Retrieved from \url{http://news.am/eng/news/93843.html}}. Its
complete/development projects below:

\begin{itemize}
\item \textbf{``Search for new interesting objects of definite types by
low-dispersion template spectra''}:
``modeling of spectra [...] [for a] QSOs, Seyfert galaxies, white dwarfs, [...]
late-type stars (K-M, S, carbon)'' 

\item \textbf{``Optical identifications of new gamma, X-ray, IR and radio
sources''}:
using the Byurakan 2.6 [m] telescope, ``the first test resulted in 145 objects
found, 81 being known QSOs/Sys, and 64 new candidates (including 23 NVSS and
FIRST radio sources)''.

\item \textbf{``Identification of the newly found IR sources from Spitzer Space
Telescope (SST)''}:
``73 unidentified sources in the Bootes region have been found and clasified on
the DFBS plates''\footnote{Mickaelian, A. (2006, August). Science projects with
the Armenian Virtual Observatory (Arvo). Karel A.  van der Hucht (Ed.),
\textit{Highlights of Astronomy} (p. 529). Vol. 14. Prague: Cambridge University
Press.}.
\end{itemize}

\subsubsection{Hungarian Virtual Observatory \cite{website:hvo-home} (HVO)}
Its complete/development projects below:

\begin{itemize}
\item \textbf{Spectrum Service for VO}:
a proposal that intends to add several features and make two substantial
improvements\footnote{Does not specified what several features and the two
substantial improvements.} to the web services that contains spectra of galaxies
and the other astronomical objects.

\item \textbf{Synthetic Spectrum Service}:
a proposal that intends to serve, as a web service, the ready made spectra for
the users.

\item \textbf{Photometric Redshift Estimation}:
a proposal that intends to execute as a web service a method developed by
themselves that is capable to estimate redshift from photometry increasing by
two orders of magnitude the objects number of known distance. 

\item \textbf{Linking WebServices to GRID clusters}:
a proposal that intends, among other, to improvement the operating systems inter
communication, because there are simulations optimized for differents SOs and
the rewritten of the codes for one different in some cases results a
inaccessible task.

\item \textbf{Information Bulletin on Variable Stars}:
a bulletin on benhalf of the Commission
27\footnote{http://www.konkoly.hu/IAUC27/} and
42\footnote{http://www.konkoly.hu/IAUC42/} of the International Astronomical
Union (IAU), published by the Konkoly Observatory of the Hungarian Academy of
Sciences. 

\item \textbf{Debrecen Photoheliographic Data (DPD)}:
a sunspot catalogue with the heliographic positions and the areas of sunspots. A
continuation of Greenwich Photoheliographic Results (GPR) that had been
discontinued on 1976.
\end{itemize}

\subsubsection{AstroGrid}
The AstroGrid \cite{website:astrogrid-home} is the United Kingdom's virtual
observatory. It began as a project in 2001 and was launched in April 2008 along
its working service and user software. It has been financed by the Particle
Physics and Astronomy and Research Council (PPARC) and the Science \& Technology
Facilities Council (STFC). Its complete/development projects below:

\begin{itemize}
\item \textbf{Topcat}:
an interactive graphical viewer and editor for tabular data for formats like the
Flexible Image Transport System (FITS) and VOTable.

\item \textbf{VODesktop}:
an analysis tools wich allows limit the choice of resources through specific
data saving.

\item \textbf{AstroRuntime}:
an API implemented in JAVA wich facilitates the access to the \textbf{VODesktop}
services from almost any programming language \footnote{On the AstroGrid's
official website there is a document about how to access VODesktop using Python
script at \url{http://www.astrogrid.org/agpython.html}}.
\end{itemize}

\subsubsection{European Space Agency Virtual Observatory (ESA-VO)}
The ESA-VO \cite{website:esa-vo-home} aims to have the European Space Astronomy
Centre (ESAP) as the VO node for all European astronomy. It is a member of the
European Virtual Observaory (EURO-VO) and with European Southern Observatory
(ESO) co-funded the EURO-VO Facility Centre. It participated of EURO-VO Data
Centre Alliance (EURO-VO DCA) and EURO-VO Astronomical Infraestructure for Data
Access (EuroVO-AIDA), and work with the formal partners of Euro-VO International
Cooperation Empowerment (EuroVO-ICE) and EURO-VO Technology Centre (VOTECH). Its
complete/development projects below:

\begin{itemize}
\item \textbf{DALToolKit}:
a downloadable software based on JAVA that allows to publish in the VO following
the Data Access Layer (DAL) protocol. It converts incoming standard DAL requests
into database specific SQL queires, then serializes the database result into
VOTable responses. 

\item \textbf{IVOA Resource Registry}:
the official EURO-VO resource registry under the name of EURO-VO Full
Harvestable VO Resource Registry that 

\item \textbf{VOSpec}:
a multi-wavelength spectral analysis tool with access to atomic and molecular
databases, spectra and theorical models registered in the VO. It can be
downloadable and accessed through Internet.

\item \textbf{Science Activities in the VO}:
development at the European Space Astronomy Centre (ESAC) of research projects
based on VO, tutorial that teachs to use the tools made by the Science Archives
Team (SAT), among other.
\end{itemize}

\subsubsection{European Virtual Observatory (EURO-VO)}
The EURO-VO \cite{website:euro-vo-home} is the continuation of Astrophysical
Virtual Observatory (AVO). The AVO project conducted was conceived by the
European Commision and six organizations \footnote{The six organizations that
founded the AVO togheter with the European Commision are the European Southern
Observatory (ESO), the European Space Agency (ESA), the AstroGrid, the
CNRS-supported Centre de Données Astronomiques de Strasbourg (CDS), the
University Louis Pasteur; the CNRS-supported TERAPIX astronomical data centre,
the Institut d'Astrophysique; and the Jodrell Bank Observatory, the Victoria
University of Manchester.} to research about the scientific requirements and
technologies necessary to build an Euopean virtual observatory. The ``EURO-VO
aims at deploying and operational VO in
Europe''\footnote{\url{http://www.euro-vo.org/}}. Its complete/development
projects below:

\begin{itemize}
\item \textbf{VOTECH}:
the first project that implemented the concept of the EURO-VO Technology Centre
(EURO-VOTC) as part of the Euro-VO.

\item \textbf{EURO-VO Data Centre Alliance (EuroVO-DCA)}:
it is supported by the European Union (EU) in the framework of the FP6
e-Insfraestructure Communication Network Development initiative (project
RI031675). It began on 1st of September, 2006, and ended on 31th of December,
2008.

\item \textbf{EURO-VO Astronomical Infraestructure for Data Access
              (EuroVO-AIDA)}:
it is supported by the European Union (UE) in the framework of the FP7
e-Infrastructure Scientific Research Repositories initiative (project
RI2121104). It began on 1st of February, 2008, and ended on 31th of July, 2010.

\item \textbf{Euro-VO International Cooperation Empowerment (EuroVO-ICE)}:
as a Coordination Action supported by the European Union (UE) in the framework
of the FP7 INFRA-2010-2.3.3 Research Infrastructures initiative (project
261541). It began on 1st of September, 2010 and ended on 31th of August, 2012,
and succeed EuroVO-AIDA and the EuroVO-DCA.

\item \textbf{EuroVO-CoSADIE}:
as a Coordination Action supported by the EU in the framework of the FP7
INFRA-2012-3.3 Research Infraestructure initiative (project 312559). It began on
1st of September, 2012, and will end on 31th of August, 2014.
\end{itemize}

\subsubsection{German Astrophysical Virtual Observatory (GAVO)}
The GAVO \cite{website:gavo-home} was launched in 2003\footnote{Its first
publications dates from 2003. In 2004, H. Adorf and GAVO Team talking about
``GAVO - after one year'' at the Astronomical Data Analysis (ADA) III Sant'Agata
sui due Golfi, Italy. The birth of this VO is inferred from the above.}. It is
financed through the Federal Ministry of Education and Research (BMBF). Its
complete/development projects below:

\begin{itemize}
\item \textbf{GAVO Data Center}:
a growing collection of data and services provided on behalf of third parties.
Some of the GAVO services are also available on
\url{http://dc.zah.uni-heidelberg.de/}

\item \textbf{MPA Simulations access}:
a web service for querying the results of the Millennium simulation using SQL.

\item \textbf{MultiDark Database}:
a service wich gives access to data from MultiDark and Bolshoi simulations using
SQL queries.  It based on the Millennium Web Application.

\item \textbf{RAVE archive search}:
an access to a growing archive of radial velocities for more than 400 000 stars.

\item \textbf{TheoSSA}:
a service for providing spectral energy distributions based on model atmosphere
calculations.
\end{itemize}

% \subsubsection{Observatoire Virtuel France (VO-France)}
% \begin{itemize}
% \item \textbf{Projects}
% \end{itemize}

\subsubsection{Spanish Virtual Observatory (SVO)}
The SVO \cite{website:svo-home} started in June 2004 and its participants are
the Centro de Astrobiolog\'{i}a (INTA-CSIC), the Artificial Intelligence
Department of the National University of Distance Education (UNED, in its
Spanish acronym), the University of C\'{a}diz and the Center of Scientific and
Academic Services of Catalonia (CESCA, in its Spanish acronym). Its
complete/development projects below:

\begin{itemize}
\item \textbf{VO Sed Analyzer (VOSA)}:
a tool that allows to analyze stellar and galactic data reading user
photometry-tables, querying ``several photometrical catalogs accessible through
VO services'', querying ``VO-compliant theorical models (spectra)'', performing
``a statistical test to determinate which model reproduced best observed
data''\footnote{\url{http://svo2.cab.inta-csic.es/theory/vosa/helpw.php?action=help2&what=int                     ro&otype=star}}, among other. 

\item \textbf{VOSED}:
a service that builds Spectral Energy Distributions (SEDs) gathering information
from the spectrocopic services in VO. It has two modes depending of the query
objects number.

\item \textbf{TESELA}:
a service that allows to acces the catalog of blank regions. It is based on the
application of the Delaunay triangulation of the sky.
\end{itemize}

\subsubsection{Italian Virtual Observatory (VObs.it)}
The VObs.it \cite{website:vobs.it-home} besides being member of IVOA, it is a
member of EURO-VO. It was established and founded by the Italian National
Institute for Astrophysics (INAF) and is coordinated by the Information Systems
Units (SI) of this. Its complete/development projects below:

\begin{itemize}
\item \textbf{SIAP}:
a web services that provides the public Hubble Space Telescope/Advanced Camera
for Surveys (HST/ACS) Great Observatories Origins Deep Survey (GOOD) data within
the VIMOS\footnote{VIMOS is a \textbf{VI}sible imaging
\textbf{M}ulti-\textbf{O}bject \textbf{S}pectrograph, a spectrograph for the
European Southern Observatory Very Large Telescope array (ESO-VLT).}-VLT Deep
Survey-Chandra Deep Field South (VVDS-CDFS).

\item \textbf{SSAP}:
a web service that allows to access the VVDS-F02-DEEP spectra.

\item \textbf{CONE SEARCH}:
a web service that allows to query in the VVDS-CDFS catalog. 

\item \textbf{SKYNODE}:
a web service that allows to query in the VVDS catalogs. 
\end{itemize}

\subsubsection{Ukrainian Virtual Observatory (UkrVO)}
The concept of UkrVO \cite{website:ukrvo-home} was developed by a working group
of Ukrainian Astronomical Association (UAA) in 2009-2010. In 2009 June, the 8th
Congress of UAA approved the resolution of creation of a consortium of voluntary
joining where each member had the duty to prepare it records of obsevations in
accordance to the VO concept. Its complete/development projects below:

\begin{comment}
\begin{itemize}
\item \textbf{Variable Star Calculator}:

\item \textbf{Database of Golosiiv Plate Archive (DBGPA)}:

\item \textbf{CoLiTec (CLT)}:

\item \textbf{Joint Digital Archive (JDA)}:

\item \textbf{FON Astrographic Catalogue (FONAC)}:

\item \textbf{Mykolaiv Astronomical Observatory (MykAO) digital archive}:

\item \textbf{Variable Star Calculator}:

\item \textbf{Variable Star Calculator}:

\item \textbf{Variable Star Calculator}:
\end{itemize}
\end{comment}

\subsection{Africa}
\subsection{South African Astroinformatics Alliance (SA$ ^{3}  $)}
The SA cubed \cite{website:sa3-home} is managed by the National Research
Foundation (NRF) as a collaborative project among the South African Astronomical
Observatory (SAAO), the Hartebeesthoek Radio Astronomy Observatory (HartRAO) and
the Square Kilometer Array South Africa (SKA-SA). The VO intends to facilitate
the access and sharing the astronomical data obtained in South African by the
Southern African large Telescope (SALT), the Karoo Array Telescope (MeerKAT) and
Square Kilometre Array (SKA) to the international commmunity. % Its
% complete/development projects below:

\subsection{Asia}
\subsubsection{Chinese Virtual Observatory (China-VO)}
The China-VO \cite{website:china-vo-home} was initiated in 2002 by the National
Astronomical Observatories, Chinese Academy of Sciences. It has four
partners\footnote{The partners of China-VO are the National Astronomical
Observatories, Chinese Academy of Sciences (NAOC); the TianJin University (TJU),
the Central China Normal University (CCNU), Kunming University of Science and
Technology.} and more of twelve collaborators\footnote{The China-VO's
collaborators are the Computer Network Information Center, the Purple Mountain
Astro Obs, the Shangai Astro Obs, the Yunnan Astro Obs, the Tsinghua University,
the JHU, MSR, Caltech, IUCAA, CDS, ICRAR (Australia), NAOJ (Japan), among
other.}. Its complete/development projects below:

\begin{itemize}
\item \textbf{FITS Manager (FM)}:
a downloadable application, as a collaboration project between China-V0 and VOI,
to manage FITS, VOTable, among other files, hosted in personal computers
\footnote{FITS Manager (FM) is downloadble from
\url{http://fm.china-vo.org/app/fm.zip}.} 

\item \textbf{VO Data Access Service (VO-DAS)}:
a data access framework that allows to query large volumes of astronomical
resources like catalogs, spectrum and images through the command line, a
graphical interface or webpage.

\item \textbf{FITS Header Archiving System (FitHAS)}:
a downloadable tool that allows to view and import the FITS header files into a
database table for single or multiple files. It was also developed by the IBM
Center, Tianjin University (TU) and e-Science application research center,
Computer Network Information Center (CNIC), Chinese Academy of Sciences (CAS).

\item \textbf{Imaging Processing and analysis tool for China\_VO (VO\_IMPAT)}:
a downloadable imaging processing and analysis tool developed in JAVA that
allows to visualize sky images and access related data from the Beijing
Astronomical Data Center (BADC). 

\item \textbf{WDC-Astronomy Archives and Catalogs}:

\item \textbf{China-VO Grid Registry Area}:

\item \textbf{VOTFilter, XSL filter for OpenOffice}:

\item \textbf{VOTable2XHTML}:
a XSLT stylesheet that can be used to transform VOTable file into XHTML file.

\item \textbf{SkyMouse}:
a search engine that allows to access astronomical services like web service and
CGI service.

\item \textbf{Astrophysical Integrated Research Environment}:

\item \textbf{VO Tools Worlwide}:

\item \textbf{LAMOST Online Collaboration Platform}:

\item \textbf{Chinese Astronomical Data Center}:

\item \textbf{WWT Community Beijing}:

\item \textbf{Glossary of Astronomical Terms}:
\end{itemize}

\subsubsection{Japanese Virtual Observatory (JVO)}
The JVO \cite{website:jvo-home} \nocite{IshiharaMizumotoOhishiKawaray2004} was
implemented by the National Astronomical Observatory of Japan (NAOJ) with
collaboration of Fujitsu for the development of JVO prototype systems. It
started in 2002, its VO data services are interoperated from 2004 and an
analysis system was integrated in 2005 to the JVO prototype\footnote{Shirasaki
Y., Tanaka, M., Kawamoto, S., Honda, S., Ohishi, M., Mizumoto, Y., ...Sakamoto,
M. (2006, April 8) Japanese Virtual Observatory (JVO) as an advanced
astronomical research enviroment. Retrieved from:
\url{http://arxiv.org/pdf/astro-ph/0604593v1.pdf}}. Its complete/development
projects below:

\begin{itemize}
\item \textbf{JVO portal service}:
a site as portal to various kind of astronomical resources from the Subaru
Telescope, Sloan Digital Sky and ALMA, among other.
\end{itemize}

\subsubsection{Russian Virtual Observatory (RVO)}
A reason for the construction of the RVO \cite{website:rvo-home} is that
majority of observatories were south of Ex-Soviet Union. After of desintegration
of Ex-URSS, they were located in the territories outside of
Russia\footnote{\url{http://www.inasan.rssi.ru/eng/rvo/project.html}}. Its
complete/development projects below:

\begin{itemize}
\item \textbf{SYNTHESIS}:
SYNTHESIS group's framework project.

\item \textbf{INFOSEM}:
it aims to ``investigate, prototype and disseminate methodologies and basic
techniques allowing construction of semantically interoperable information
systems based on the pre-existing heterogeneous information
resources''\footnote{http://synthesis.ipi.ac.ru/synthesis/projects/InfoSem/}.

\item \textbf{SEMIMOD}:
``Modelling and Management of Semi-Structured Data for Dynamic [World Wide Web
applications]''.

\item \textbf{BIOMED}:
``Methods and tools for development of subject mediators of
he\-te\-ro\-ge\-neous information collections for distributed digital
libraries''.

\item \textbf{REFINE}:
``Modeling of compositional specifications intended for automated proof of
correctness of refinement of specifications of requirements by pre-existing
components in course of a compositional development of information systems''.

\item \textbf{VOINFRA}:
``Devolopment of principles and fundamentals of the information interoperability
in the infraestructure'' of the RVO.

\item \textbf{MULTISOURCE}:
``Methods for organization of problems solving over multiple distributed
heterogeneous information sources''.

\item \textbf{RVOAG}:
a RVO public utility center based on AstroGrid.

\item \textbf{ASTROMEDIA}:
``Methods and tools for supporting subject mediators architecture in AstroGrid
infrastructure'' for the RVO.

\item \textbf{UNIMOD}:
``Development and prototyping of experimental system for constructing the
unifying information representation models for interoperable integrating systems
of heterogeneous information sources''.

\item \textbf{SEMID}:
``Research and development of methods and tools for semantic identification of
specifications of heterogeneous information resources relevant to a scientific
problem and their integration in the specifications of the problem at the
scientific information systems''.

\item \textbf{SubjMed}:
``Investigation of methods and tools for subject mediation middleware aimed at
problems solving over heterogeneous distributed information resources''.

\item \textbf{ConcMod}:
``Development of methods and tools for definition of scientific subject domains
conceptual models and problems solving support based on mediators subject in the
hybrid grid-infrastructure''.

\item \textbf{RuleInt}:
``Integration of rule-based declarative programs and knowledge databases and
services for scientific problems solving over heterogeneus distributed
information resources''.

\item \textbf{ASTROMEDIA Trial}:
``Hybrid architecture of AstroGrid and Mediator Middlewere''.

\item \textbf{Galaxies Search}:
``Distant Galaxies Search Applying AstroGrid''.

\item \textbf{Star Classification}:
``Eclipsing-binary Stars Classification applying Ensembled Weka [algorithm] in
AstroGrid''\footnote{\url{http://synthesis.ipi.ac.ru/synthesis/projects}}.
\end{itemize}

\subsubsection{Virtual Observatory India (VOI)}
The VOI \cite{website:voi-home} is a collaboration between the Inter University
Center for Astronomy and Astrophysics (IUCAA) and the Persistent Systems Ltd.,
and is supported by the Ministry of Communication and Information Technology,
Government of India. Its complete/development projects below:

\begin{itemize}
\item \textbf{VOIPortal}:
an entry to all VOI web services. Can be browse the data downloading or through
VOIMosaic and PyMorph web applications.

\item \textbf{Mosaic Service}:
a software that allows to make mosaic, with
SWarp\footnote{\url{http://www.astromatic.net/software/swarp}} and
SExtractor\footnote{\url{http://www.astromatic.net/software/sextractor}} and
STIFF, of images retrieved from
SDSS\footnote{\url{http://casjobs.sdss.org/vo/DR7SIAP/SIAP.asmx}},
2MASS\footnote{\url{http://irsa.ipac.caltech.edu/applications/2MASS/IM/}} and
HST\footnote{\url{http://archive.stsci.edu/siap/search.php}} image servers.

\item \textbf{PyMorph Service}:
a software that allows to derive morphological parameters for galaxy images. Is
possible to provide to it the output FITS files generated by Mosaic Service.

\item \textbf{VOPlot}:
a software tool developed in JAVA that allows to visualize astronomical data
available in VOTable, ASCII and FITS formats.

\item \textbf{VOMegaPlot}:
a software tool developed in JAVA that allows to visualize astronomical data
available in VOTable format. It looks just like VOPlot. There is a client-server
version.

\item \textbf{AstroStat}:
a software tool that allows astronomers to use both and sophisticated statical
routines on large datasets uploaded in VOTable or ASCII format.

\item \textbf{VOCat}:
a software tool that converts astronomical catalogs to MySQL databases. 

\item \textbf{VOPlatform}:
a software tool developed in JAVA that allows to place their frequently used VO
tools and datasets with others resourcers like documents, links, among other.

\item \textbf{VOConvert (ConVOT)}:
a software tool that converts ASCII to VOTable files, FITS to VOTable and
VOTable to ASCII.

\item \textbf{Android Cosmological Calculator}:
an Android aplication that allows to input the Hubble constant, $ \Omega_{m} $
(matter), $ \Omega_{\lambda} $ (vacuum) and the redshift($ z $), and returns the
current age of the Universe, the co-moving radial distance and volume and the
angular size distance at the specified redshift, and the luminosity distance.

\item \textbf{Android Name Resolver}:
an Android application that allows to input the name of celestial object and
returns information of this like RA/DEC values, redshift, proper motion,
parallax, among other.

\item \textbf{CSharpFITS Package}:
a C\# .NET port of Tom McGlynn's nom.tam.fits JAVA
package\footnote{\url{http://heasarc.gsfc.nasa.gov/docs/heasarc/fits/java/v0.9/javadoc/}}.

\item \textbf{VOTable JAVA Streaming Writer}:
a software that converts data streams in non-VOTable format, like ASCII or FITS,
to the VOTable format.

\item \textbf{C++ parser for VOTable}:
a C++ library to access VOTable files. It has a non-streaming and streaming
version.

\item \textbf{Fits Manager}:
a web-based tool for viewing, creating, adding extensions and converting FITS
files.

\item \textbf{HCT Data Archive System}:
a web-based system that archives the observational data generated by the
Himalayan Chandra Telescope (HCT), a 2 [m] aperture optical-infrared telescope
manufactured by the EOS Technologies Inc. and remotely operated via dedicated
satellite link.
\end{itemize}

\subsection{Australia}
\subsubsection{Australian Virtual Observatory (Aus-VO)}
% Its complete/development projects below:

\begin{comment}
\begin{itemize}
\item \textbf{RAVE}:

\item \textbf{Remote Visualisation System}:

\item \textbf{AA Grid Demo 2003}:

\item \textbf{2QZ}:

\item \textbf{Machine Learning for Source Marching}:

\item \textbf{MACHO Archive}:

\item \textbf{ATCA Archive}:

\item \textbf{HIPASS Project}:

\item \textbf{SkyCat}:

\item \textbf{Volume}:

\item \textbf{Conesearch}:
\end{comment}
