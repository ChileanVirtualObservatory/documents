\section{Virtual Observatory Registry }

The description of the work done by each IVOA member
in the previous section is an accurate political, organizational 
and geographical division. However, in practice the services
are sparsely distributed between universities, observatories
and institutes around the world. In fact, some of them 
are part of more than one country-VO. 
This makes very complex to track the available services, 
and to search in the VO as a whole. Fortunately, 
IVOA has developed a \emph{registry} standard to 
exploit the interoperability of the VO nodes.

This section explains very briefly the registry standard,
and use the existing registry services for obtaining VO 
statistics mainly for two purposes:
\begin{itemize}
\item showing the advantages of the interoperability in the VO,
\item and to survey the types of services, data and wavebands that
the VO supports as a whole.
\end{itemize}

\subsection{The Registry}

% description of the registry
% the registry that we will use

\subsection{VO Resources by Type of Service}

\subsection{VO Resources in the Electromagnetic Spectrum}

\subsection{VO Resources by Observation Subject}


%Most of virtual observatories usually contribute to development of tools for
%for national astronomical instruments. These tools can be new source codes for
%the exploitation of the capability of some specific instrument, facilitation or
%improvement of the queries, reading, writing, conversion of the astronomical
%data of files like VOTable, ASCII and FITS formats, among other. 
%This means that each VO concentrates in providing services for specific
%data types, which in astronomy are highly correlated with the wavelengths values
%on which their work. This section surveys the ranges for which 
%each virtual observatory is developing their tools, 
%
%\begin{landscape}
%\begin{figure}%[h]
%\begin{center}
%\begin{tikzpicture}
%\pgfmathsetmacro{\lineLenght}{15}
%\pgfmathsetmacro{\lineDivNum}{17}
%\pgfmathsetmacro{\markPlace}{\lineLenght/\lineDivNum}
%\def\limitsList{5,8,10.3,10.6,13,15}
%\def\regionArray{{2.5,(5+8)/2,(8+10.3)/2,(10.3+10.6)/2,(10.6+13)/2,(13+15)/2,
%                  (15+17)/2}}
%\foreach \x in {0,1} {
%  \draw[<->,thick] (0,\x) -- (\lineLenght,\x);
%}
%\foreach \x in \limitsList {
%  \draw[thick] (\x*\markPlace,0.8) -- (\x*\markPlace,1.2);
%}
%\foreach \x in {1,2,...,16} {
%  \pgfmathtruncatemacro{\b}{\x-2*\x+4} 
%  \draw[thick] (\x*\markPlace,-0.2) -- (\x*\markPlace,0.2)
%  node[pos=-0.6]{$10^{\b}$};
%}
%\node [anchor=north east,align=center] at (0,0) {Wavelenght\\(in meters)};
%\node [anchor=south east,align=center] at (0,1) {Radiation\\type};
%\node [above] at (\regionArray[0]*\markPlace,1.5) {Radio};
%\node [above] at (\regionArray[1]*\markPlace,1.5) {Microwave};
%\node [above] at (\regionArray[2]*\markPlace,1.5) {Infrared};
%\node [above] at (\regionArray[3]*\markPlace,2) {Visible};
%\node [above] at (\regionArray[4]*\markPlace,1.5) {Ultraviolet};
%\node [above] at (\regionArray[5]*\markPlace,1.5) {X-ray};
%\node [above] at (\regionArray[6]*\markPlace,1.5) {Gamma ray};
%
%% Spectrum value or range for each instrument where some virtual observatory
%% contributes
%
%\pgfmathsetmacro{\factor}{\markPlace/2}
%% ALMA
%\draw[<->,thick] (7.7*\markPlace,-4) -- (8.6*\markPlace,-4);
%\pgfmathparse{multiply(8.6-7.7,\factor)}
%\node at (\pgfmathresult+7.7*\markPlace,-3.5) {ALMA};
%\node at (\pgfmathresult+7.7*\markPlace,-4.5) {\textbf{ChiVO}};
%\node at (\pgfmathresult+7.7*\markPlace,-5.0) {\textbf{JVO}};
% BAO
%\draw[<->,thick] (3.58*\markPlace,-2) -- (3.88*\markPlace,-2);
%\pgfmathparse{multiply(3.88-3.58,\factor)}
%\node at (\pgfmathresult+3.58*\markPlace,-1.5) {BAO};
%\node at (\pgfmathresult+3.58*\markPlace,-2.5) {\textbf{ArVO}};
%% Chandra X-ray Telescope 
%\draw[<->,thick] (13*\markPlace,-2) -- (15*\markPlace,-2);
%\pgfmathparse{multiply(15-13,\factor)}
%\node at (\pgfmathresult+13*\markPlace,-1.5) {Chandra X-ray Telescope};
%\node at (\pgfmathresult+13*\markPlace,-2.5) {\textbf{VAO}};
%% HASTA 
%\node at (3.4312*\markPlace,-4) [inner sep=1pt,circle,draw] {};
%\node at (3.4312*\markPlace,-3.5) {HASTA};
%\node at (3.4312*\markPlace,-4.5) {\textbf{NOVA}};
%% HCT
%\draw[<->,thick] (8*\markPlace,-6) -- (10.3*\markPlace,-6);
%\pgfmathparse{multiply(10.3-8,\factor)}
%\node at (\pgfmathresult+8*\markPlace,-5.5) {HCT};
%\node at (\pgfmathresult+8*\markPlace,-6.5) {\textbf{VOI}};
%% HST  
%\draw[<->,thick] (8*\markPlace,-10) -- (13*\markPlace,-10);
%\pgfmathparse{multiply(13-8,\factor)}
%\node at (\pgfmathresult+8*\markPlace,-9.5) {HST};
%\node at (\pgfmathresult+8*\markPlace,-10.5) {\textbf{VAO, VObs.it}};
%% SOAR 
%\draw[<->,thick] (9*\markPlace,-4) -- (10.68*\markPlace,-4);
%\pgfmathparse{multiply(10.68-9,\factor)}
%\node at (\pgfmathresult+9*\markPlace,-3.5) {SOAR};
%\node at (\pgfmathresult+9*\markPlace,-4.5) {\textbf{BRAVO}};
%% SST
%\draw[<->,thick] (8.82*\markPlace,-2) -- (9.7*\markPlace,-2);
%\pgfmathparse{multiply(9.7-8.82,\factor)}
%\node at (\pgfmathresult+8.82*\markPlace,-1.5) {SST};
%\node at (\pgfmathresult+8.82*\markPlace,-2.5) {\textbf{VAO, ArVO}};
%% Subaru Telescope
%\draw[<->,thick] (8*\markPlace,-8) -- (10.6*\markPlace,-8);
%\node at (\pgfmathresult+8*\markPlace,-7.5) {Subaru Telescope};
%\node at (\pgfmathresult+8*\markPlace,-8.5) {\textbf{JVO}};
%
%% Perpendicular long dotted lines
%
%\def\firstLinesEnds{3.58,3.88,8.82,9.7,13,15}
%\foreach \x in \firstLinesEnds {
%  \draw[dotted] (\x*\markPlace,0) -- (\x*\markPlace,-2);
%}
%\def\secondLinesEnds{3.4312,7.7,8.6,9,10.68}
%\foreach \x in \secondLinesEnds {
%  \draw[dotted] (\x*\markPlace,0) -- (\x*\markPlace,-4);
%}
%\draw[dotted] (8*\markPlace,0) -- (8*\markPlace,-6);
%\draw[dotted] (10.3*\markPlace,0) -- (10.3*\markPlace,-6);
%\draw[dotted] (8*\markPlace,0) -- (8*\markPlace,-8);
%\draw[dotted] (10.6*\markPlace,0) -- (10.6*\markPlace,-8);
%\draw[dotted] (8*\markPlace,0) -- (8*\markPlace,-10);
%\draw[dotted] (13*\markPlace,0) -- (13*\markPlace,-10);
%\end{tikzpicture}
%\caption{Virtual observatories that contributes to specifics astronomical
%instruments.}
%\end{center}
%\label{figure:wavelength}
%\end{figure}
%\end{landscape}
%
%In Figure \ref{figure:wavelength}, the different types of
%radiation (in order) with their common names are shown. Below, the value of
%the wavelengths in meters and the respective telescope/instrument are presented. 

%A higher $ \lambda $
%value is related with a lower frequency and a
%lower $ \lambda $ is related with a higher frequency. 
%The electromagnetic
%spectrum value or range for each instrument (above) and virtual observatory
%(below) is represented with a dot or line, respectively.\\

%\scriptsize
%\begin{table*}[h!t]
%\centering
%\begin{tabular}{|p{1cm}|p{4cm}|p{5cm}|p{6cm}|}
%    \hline                                                                      
%    \textbf{VO} & \textbf{Instrument} & \textbf{Location} &
%    \textbf{Spectrum value or range} \\
%    \hline                                                                      
%    BRAVO & Southern Astrophysical Research
%    Telescope (SOAR) & Cerro Pach\'{o}n, Chile & blue (320 nm) to near infrared
%    \cite{website:SOAR_EMS} \\
%    \hline                                                                      
%    NOVA & H-Alpha Solar Telescope for
%    Argentina (HASTA) & Leoncito, San Juan, Argentina & H-Alpha (656.28 nm)
%    \cite{website:HASTA_EMS} \\
%    \hline
%    ChiVO & Atacama Large Milimeter/submilimeter
%    Array (ALMA) & Llano de Chajnantor Observatory, Atacama Desert, Chile &
%    initially in 400 $ \mu $m to 3 mm \cite{website:ALMA_EMS} \\
%    \hline
%    \multirow{3}{3cm}{VAO} & Hubble Space
%    Telescope (HST) & 569 km above the surface of Earth & near-ultraviolet,
%    visible and near-infrared light with WFC3; ultraviolet light with COS;
%    visible light with ACS; ultraviolet, visible and near-infrared light with
%    STIS; infrared light with NICMOS \cite{website:HST_EMS} \\
%     \cline{2-4}
%     & Chandra X-ray Observatory & 139,000 km above the surface of Earth & X-ray
%     \cite{website:Chandra_EMS} \\
%     \cline{2-4}
%     & Spitzer Space Telescope (SST) & 176,602,814 km above the surface of Earth
%     \cite{website:SST_EMS_1} & 3 $ \mu $m to 180 $ \mu $m
%     \cite{website:SST_EMS_2} \\
%    \hline
%    ArVO & Byurakan Astrophysical Observatory
%    (BAO) & Mount Aragats, Armenia & 1.2 m and 4.2 m \cite{website:BAO_EMS} \\
%    \hline
%    ArVO & Spitzer Space Telescope (SST) &
%    176,602,814 km above the surface of Earth & 3 $ \mu $m to 180 $ \mu $m \\
%    \hline
%    Vobs.it & Hubble Space Telescope (HST) & 569
%    km above the surface of Earth & near-ultraviolet, visible and near-infrared
%    light with WFC3; ultraviolet light with COS; visible light with ACS;
%    ultraviolet, visible and near-infrared light with STIS; infrared light with
%    NICMOS \\
%    \hline
%    \multirow{2}{3cm}{UkrVO} & Main Astronomical
%    Observatory (MAO) & Kiev, Ukraine & \\
%     \cline{2-4}
%     & Mykolaiv Astronomical Observatory & Mykolaiv, Ukraine & \\
%    \hline
%    \multirow{3}{3cm}{JVO} & Subaru Telescope &
%    Mauna Kea, Big Island, Hawaii & visible and infrared light
%    \cite{website:Subaru_EMS} \\
%     \cline{2-4}
%     & Sloan Digital Sky Survey (SDSS) & Sacramento Mountains, Sunspot, New
%     Mexico, USA & \\
%     \cline{2-4} 
%     & Atacama Large Milimeter/submilimeter Array (ALMA) & Llano de Chajnantor
%     Observatory, Atacama Desert, Chile & initially in 400 $ \mu $m to 3 mm \\
%    \hline
%    VOI & Himalayan Chandra Telescope (HCT) & Mount
%    Saraswati, Digpa-ratsa Ri, Hanle, India & infrared light 
%    \cite{website:HCT_EMS} \\
%    \hline
% \end{tabular}
%\caption{Virtual observatories that contributes to specifics astronomical
%instruments.}
%\label{table:vo_EMS}
%\end{table*}
%%\end{longtable}
%%\end{center}
%\normalsize

