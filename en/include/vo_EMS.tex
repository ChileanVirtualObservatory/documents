\section{Virtual Observatory in the Electromagnetic Spectrum}
The table \ref{table:vo_EMS} shows the wavelenghts values and range wavelenghts
for which there is a virtual observatory that contributes to design,
construction, implementation, support and/or improving of a databse with
astronomical data obtained by a specific instrument.\\

% \begin{table}%[h!t]                                                           
% \centering
\begin{center}  
% \begin{tabular}{|m{3cm}|m{3cm}|m{3cm}|m{5cm}|}
\begin{longtable}{|m{3cm}|m{3cm}|m{3cm}|m{5cm}|}
    \hline                                                                      
    \textbf{Virtual Observatory} & \textbf{Instrument} & \textbf{Location} &
    \textbf{Spectrum value or range} \\
    \hline                                                                      
    \endfirsthead
    \hline                                                                      
    \textbf{Virtual Observatory} & \textbf{Instrument} & \textbf{Location} &
    \textbf{Spectrum value or range} \\
    \hline                                                                      
    \endhead
    Brazilian Virtual Observatory (BRAVO) & Southern Astrophysical Research
    Telescope (SOAR) & Cerro Pach\'{o}n, Chile & blue (320 nm) to near infrared
    \cite{website:SOAR_EMS} \\
    \hline                                                                      
    Nuevo Observatorio Virtual Argentino (NOVA) & H-Alpha Solar Telescope for
    Argentina (HASTA) & Leoncito, San Juan, Argentina & H-Alpha (656.28 nm)
    \cite{website:HASTA_EMS} \\
    \hline
    \multirow{3}{3cm}{US Virtual Astronomical Observatory (VAO)} & Hubble Space
    Telescope (HST) & 569 km above the surface of Earth & near-ultraviolet,
    visible and near-infrared light with WFC3; ultraviolet light with COS;
    visible light with ACS; ultraviolet, visible and near-infrared light with
    STIS; infrared light with NICMOS \cite{website:HST_EMS} \\
     \cline{2-4}
     & Chandra X-ray Observatory & 139,000 km above the surface of Earth & X-ray
     \cite{website:Chandra_EMS} \\
     \cline{2-4}
     & Spitzer Space Telescope (SST) & 176,602,814 km above the surface of Earth
     \cite{website:SST_EMS_1} & 3 $ \mu $m to 180 $ \mu $m
     \cite{website:SST_EMS_2} \\
    \hline
    \multirow{2}{3cm}{Armenian Virtual Observatory (ArVO)} & Byurakan
    Astrophysical Observatory (BAO) & Mount Aragats, Armenia & 1.2 m and 4.2 m
    \cite{website:BAO_EMS} \\
     \cline{2-4}
     & Spitzer Space Telescope (SST) & 176,602,814 km above the surface of Earth
     & 3 $ \mu $m to 180 $ \mu $m \\
    \hline
    Italian Virtual Observatory (Vobs.it) & Hubble Space Telescope (HST) & 569
    km above the surface of Earth & near-ultraviolet, visible and near-infrared
    light with WFC3; ultraviolet light with COS; visible light with ACS;
    ultraviolet, visible and near-infrared light with STIS; infrared light with
    NICMOS \\
    \hline
    \multirow{2}{3cm}{Ukrainian Virtual Observatory (UkrVO)} & Main Astronomical
    Observatory (MAO) & Kiev, Ukraine & \\
     \cline{2-4}
     & Mykolaiv Astronomical Observatory & Mykolaiv, Ukraine & \\
    \hline
    \multirow{3}{3cm}{Japanese Virtual Observatory (JVO)} & Subaru Telescope &
    Mauna Kea, Big Island, Hawaii & visible and infrared light
    \cite{website:Subaru_EMS} \\
     \cline{2-4}
     & Sloan Digital Sky Survey (SDSS) & Sacramento Mountains, Sunspot, New
     Mexico, USA & \\
     \cline{2-4} 
     & Atacama Large Milimeter/submilimeter Array (ALMA) & Llano de Chajnantor
     Observatory, Atacama Desert, Chile & initially in 400 $ \mu $m to 3 mm
     \cite{website:ALMA_EMS} \\
    \hline
    Virtual Observatory India (VOI) & Himalayan Chandra Telescope (HCT) & Mount
    Saraswati, Digpa-ratsa Ri, Hanle, India & infrared light 
    \cite{website:HCT_EMS} \\
    \hline
% \end{tabular}
\caption{Virtual observatories in the electromagnetic spectrum.}
\label{table:vo_EMS}
% \end{table}
\end{longtable}
\end{center}
