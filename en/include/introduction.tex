\section{Introduction}
The Virtual Observatory (VO) \cite{website:ivoa-home} \nocite{Borne2013}
\nocite{HanischQuinn2003} is a international initiative that allows the access
to astronomical files and data centres to astronomers and any person through
Internet. With the standardization of the information and methods is possible
study the astronomical data without the physical requirement of the tools and
location.\\

In June 2002, was made the International Virtual Observatory Alliance (IVOA) to
``facilitate the international coordination and collaboration necessary for the
development and deployment of the tools, systems and organizational structures
necessary to enable the international utilization of astronomical archives as an
integrated and interoperating virtual observatory''.  Actually, the IVOA is
composed of 19\footnote{On the official website in \textbf{What is the IVOA}
``the IVOA now comprises 17 VO projects'', but in \textbf{Member Organizations}
appears 19 members listed.} projects of America, Asia, Europa and Oceania; its
members meet two times each year in Interoperability  Workshops to have
discussions face-to-face and resolve technical questions.\\

An initiative led by Ph. D. Mauricio Solar alongisde students of Federico Santa
Mar\'{i}a Technical University intends to develop an astro-informatics platform
to manage and analyse intelligently large-scale data based on the IVOA
standards. Due the above, this document aims to:

\begin{itemize}
	\item Publicize the distribution of the virtual observatories worldwide.
    \item List the tools developed by the virtual observatories and their status
from the information provided on their official websites on Internet.
    \item Get an idea about what additional tools could be developed to
guarantee fulfilling the main objective\footnote{\textit{``Desarrollo de una
plataforma astro-inform\'{a}tica para la administraci\'{o}n y an\'{a}lisis
inteligente de datos a gran escala''} according to the name of Fondef D11$
\vert $1060 project}.
\end{itemize}

For these purposes, were reviewed the official websites of the IVOA and its
members, were read sections of documents about virtual observatories like
``Virtual Observatories, Data Mining, and Astroinformatics'' of Kirk Borne,
George Mason University, among others.\\

Finally, in this document, the distribution of the virtual observatories in the
world is first by continent ordered alphabetically, then alphabetically by
contries.\\
