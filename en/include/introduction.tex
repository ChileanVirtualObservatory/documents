\section{Introduction}
The Virtual Observatory (VO) \cite{Borne2013,HanischQuinn2003} is an
international initiative that allows accessing the astronomical files hosted in
several data centers around the world. The standardization of methods and information 
enables astronomers, and in general any person on the Internet, to study the 
astronomical data without the need of physical displacement of the researchers 
or their tools. Beside the obvious advantages in terms of time and budget of 
using a virtual service, the VO provides a broad access to previous observations
and related data, helping astronomers to produce better and updated
science. Furthermore, VO members provide in some cases on-line processing and
data analysis tools, which enrich even more the research possibilities of
astronomers. This standardization goes beyond interoperable services: common
semantics and generic data models also serve as patterns for VO software
development. 

In June 2002, the International Virtual Observatory Alliance (IVOA)
\footnote{\texttt{http://www.ivoa.net/}} was created, with the main objective of
defining standards to produce synergy and interoperability between the
VO members. The members of the alliance are local organizations that represent
a whole country VO, and propose and defend their interest in IVOA.
Since it inception, IVOA had grown up to 21 members in the 5 continents,
reaching reasonable maturity and consensus on how VO data should be
modelled and accessed\footnote{Despite this, the arrival of new observatories and instruments 
require a continuous revision of the standards and protocols to support new
data.}.
Nevertheless, most of the VOs aim beyond data access, where grid and cloud computing
are probably the most interesting challenges for the near future. 

% on order to ``facilitate the international 
%coordination and
%collaboration necessary for the development and deployment of the tools, systems
%and organizational structures necessary to enable the international utilization
%of astronomical archives as an integrated and interoperating virtual
%observatory''. Currently, the IVOA is composed of 19\footnote{On the official
%website in \textbf{What is the IVOA} ``the IVOA now comprises 17 VO projects'',
%but in \textbf{Member Organizations} appears 19 members listed.} projects of
%America, Asia, Europa and Oceania; its members meet two times each year in
%\emph{Interoperability Workshops} to have discussions face-to-face and resolve
%technical questions.

A rather uncommon property, 
compared to other virtual centers, 
that virtual observatories hold, 
is the will of collaboration and openness of IVOA members, motivated by 
the scientific culture of astronomers towards public access. This allows not only good
interoperability, but the natural specialization of the VOs with respect to
the instruments, data and science that each country possesses. 
Unfortunately, IVOA only keep track of the standards and technical discussions
of interoperability, so there is no broad perspective nor unifying view of what
the VO offer as a whole.

This paper provides this broad view of the VO, including its basic architecture
and popular services in Section~\ref{sec:ivoa}. Then, in
Section~\ref{sec:vo_services} we present the projects, 
services and specializations that each IVOA member have developed (or plan to
develop). In Section~\ref{sec:registry}, we present examples of the benefits of IVOA 
standardization in practice, which allows us to discover
resources through the whole VO around the world.


%\rem{MA}{Paper outline}

%An initiative led by Ph. D. Mauricio Solar alongisde students of Federico Santa
%Mar\'{i}a Technical University intends to develop an astro-informatics platform
%to manage and analyse intelligently large-scale data based on the IVOA
%standards. Due the above, this document aims to:

%\begin{itemize}
%	\item Publicize the distribution of the virtual observatories worldwide.
%    \item List the tools developed by the virtual observatories and their status
%from the information provided on their official websites on Internet.
%    \item Get an idea about what additional tools could be developed to
%guarantee fulfilling the main objective\footnote{\textit{``Desarrollo de una
%plataforma astro-inform\'{a}tica para la administraci\'{o}n y an\'{a}lisis
%inteligente de datos a gran escala''} according to the name of Fondef D11$
%\vert $1060 project}.
%\end{itemize}

%For these purposes, were reviewed the official websites of the IVOA and its
%members, were read sections of documents about virtual observatories like
%``Virtual Observatories, Data Mining, and Astroinformatics'' of Kirk Borne,
%George Mason University, among others.\\

%Finally, in this document, the distribution of the virtual observatories in the
%world is first by continent ordered alphabetically, then alphabetically by
%contries.\\
