\section{Conclusions and Recommendations}
%At present the distribution of the virtual observatories in the world is not
%related with the astronomical installations, e.g., only ESO (European Souther
%Observatory) operates in three places in the north of Chile: La Silla, Paranal
%and
%Chajnantor\footnote{\url{http://www.eso.org/public/chile/about-eso/cooperation.html}},
%but there still does not exist the presence of the Virtual Observatory (VO).
%The 47\% of virtual observatories have been founded by the members of European
%Community.\\

The membership of IVOA does not
guarantee a constant contribution from its members: the alliance only 
intends to share the astronomical
knowledge between them and the community in a standardized manner. 
In fact, during this research, we have realized that several VOs have not 
updated the status of their projects, and moreover several official sources 
or data is not accessible from a web platform. This paper intent to be
a first step to keep an updated list of services, tools and projects
developed by the different VOs.

%The development and implementing of a virtual observatory in Chile is urgent.
%Chile is an astronomer's
%paradise\footnote{\url{http://www.bbc.co.uk/news/world-latin-america-14205720}}.
%A platform under the IVOA's standards from there allows to facilitate the
%Chileans and global astronomical contributions, among others.\\

%Countless of tools could be developed from a Chilean virtual observatory. These
%applications would respond to the currents and future needs for any member of
%IVOA. The International Virtual Observatory Alliance has the Working Groups
%which works to development the standards that would later all members will be
%submitted.\\
