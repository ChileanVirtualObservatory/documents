\section{Conclusions and Recommendations}
At present the distribution of the virtual observatories in the world is not
related with the astronomical installations, i.e., only ESO (European Souther
Observatory) operates in three places in the north of Chile: La Silla, Paranal
and
Chajnantor\footnote{\url{http://www.eso.org/public/chile/about-eso/cooperation.html}},
but there still does not exist the presence of the Virtual Observatory (VO).
The 47\% of virtual observatories have been founded by the members of European
Community.\\

The membership of the International Virtual Observatory Alliance does not imply
the constant contribution from the astronomy activity, this is independent,
because the IVOA is a initiative which intends to share the astronomical
knowledge between them and the community in a standardized manner. On the other
hand, during the research, was visualized that have not been updated the status
projects of some virtual observatories from the official sources or the data is
not accesible from a web platform.\\

The development and implementing of a virtual observatory in Chile is urgent.
Chile is an astronomer's
paradise\footnote{\url{http://www.bbc.co.uk/news/world-latin-america-14205720}}.
A platform under the IVOA's standards from there allows to facilitate the
Chileans and global astronomical contributions, among others.\\

Countless of tools could be developed from a Chilean virtual observatory. These
applications would respond to the currents and future needs for any member of
IVOA. The International Virtual Observatory Alliance has the Working Groups
which works to development the standards that would later all members will be
submitted.\\
