\section{Virtual Observatory Services and Tools}

The IVOA architecture only defines a general framework and
standards that their members should follow, but each VO develop
their own services and tools depending on their specific goals.
In this section, some of these initiatives are briefly
described by region.

\subsection{North America}

Both the CVO and the NVO are very active members of IVOA,
providing several data access, imaging and analysis services 
and tools. 

The CVO has largely focus on the CANFAR Virtual Storage
System\footnote{\url{http://www.canfar.phys.uvic.ca/canfar/}, which
allows accessing very large resources for both storage and processing, 
using a cloud based framework \cite{}. 
%Virtualization and Grid Utilization within the CANFAR Project
%Gaudet, ADASS
%http://aspbooks.org/custom/publications/paper/442-0061.html
This is a very generic framework for accessing and processing 
large astronomical data sets that implements most of the
VOSpace standard of IVOA \cite{VOSPace}. CVO also has implemented
IVOA's data access services like TAP for metadata queries 
\cite{} and SIA for
image access \cite{}.



