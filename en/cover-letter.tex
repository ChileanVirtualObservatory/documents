\documentclass[a4paper]{article}

\usepackage[utf8]{inputenc}
%\usepackage{framed}
\usepackage{changebar}
\usepackage{enumitem}
\usepackage{moreverb}
\usepackage{amsmath}

\usepackage{xcolor}

\usepackage{amsmath}
\usepackage{amsfonts}
\usepackage{mathrsfs}
\usepackage{amssymb}
\usepackage{amsthm}
\usepackage{mathtools}

\usepackage{ifthen}
% comments \nb{label}{color}{text}
\newboolean{showcomments}
\setboolean{showcomments}{true}
\ifthenelse{\boolean{showcomments}}
	{\newcommand{\nb}[3]{
		{\colorbox{#2}{\bfseries\sffamily\scriptsize\textcolor{white}{#1}}}
		{\textcolor{#2}{\sf\small$\blacktriangleright$\textit{#3}$\blacktriangleleft$}}}
	 \newcommand{\version}{\emph{\scriptsize$-$Id$-$}}}
	{\newcommand{\nb}[3]{}
	 \newcommand{\version}{}}
\newcommand{\mauricio}[1]{\nb{MA}{green!70!black}{#1}}
\bibstyle{plain}
%\newenvironment{response}{\cbstart \em}{\cbend}
%\newcommand{\response}[1]{\cbstart {\em #1 } \cbend}
\newcommand{\quoting}[1]{
  
%  \cbend
  
  \colorbox{lightgray}{
    \begin{minipage}{0.98\linewidth}
      \em #1
    \end{minipage}
  }
  
%  \cbstart
  
}

\title{Cover Letter NEWAST-D-14-00117}

\author{Mauricio Araya and Mauricio Solar and Jonathan Antognini}

\begin{document}

\maketitle

%\olivier{%
%  Useful commands:
%  \begin{itemize}
%  \item $\backslash$mauricio$\{$[text]$\}$ for Mauricio's comments
%  \item $\backslash$vincent$\{$[text]$\}$ for Vincent's comments
%  \item $\backslash$olivier$\{$[text]$\}$ for Olivier's comments
%  \item $\backslash$response$\{$[text]$\}$ for our responses to the
%    reviewers (italic text on grey background)
%  \item $\backslash$cbstart -- to start a section with a vertical bar
%    in the margin (i.e., the citation of the reviews)
%  \item $\backslash$cbend -- to end a section with a vertical bar
%    in the margin
%  \end{itemize}
%}


\section{Response to the Editor}

\quoting{
Dear Dr. Araya,

One or more reviewers have now commented on your article. In view of the
comments made, your paper is acceptable for publication. However, there are a
few issues that need to be addressed.
You are requested to submit the revision within 6 weeks from now.

For your guidance, please find below the reviewer comments.
}
\vspace{0.5cm}

We thank the reviewer for his useful feedback. We generally agree
with the minors comments of the review, and this document 
explains the changes made to the original manuscript to be 
considered for publication by the editor.
\vspace{0.5cm}

%We thank the editor for considering t

\section{Response to the Reviewer}

\quoting{
Reviewer \#1: The paper present a simple survey of the Virtual Observatory
framework. 
}
\vspace{0.5cm}

We agree. We have changed the title of
the paper to ``A Brief Survey on the Virtual Observatory'' in order to provide
a more accurate description of the paper.
\vspace{0.5cm}

\quoting{
So, while 2.4 cap illustrates very simply the main VO standards, section 3 looks
more focusing on the geographical distribution of the IVOA partners and does not
put evidence enough to the tools developed over the standards.
}
\vspace{0.5cm}
We have added section 2.5, that summarizes the VO toolkits for publishing data
and their current status.
\vspace{0.5cm}

\quoting{
One of the main
success of the IVOA is not only to develop international standards to make data
interoperable but also sponsored tools that allowing the easy access to these
data.
}
\vspace{0.5cm}

We have added an updated list of the most used VO tools in section 2.6, to emphasize the 
importance of them as the visible side of IVOA. 
\vspace{0.5cm}

\quoting{
Also the big work done about the semantics is not nominated (UCD): it's an
essential part of the VOTable standard needed to move astronomical data to the
web 2.0.
}
\vspace{0.5cm}


We have added a new paragraph in section 2.4 that highlight the work done in
semantics. We have specifically related this to the VOTable standard and
the usage of VO tools.
\vspace{0.5cm}

\quoting{
About cap 4. It is not well clear what does it means. 
}
\vspace{0.5cm}

We have clarified the section introduction, explaining that the section
shows the benefits of the registry interface by taking simple statistic
of the registry. 
\vspace{0.5cm}

\quoting{
Also table 3 : is it a list ( example ) of Vo resources ??  are you sure that
all services are active ?
}
\vspace{0.5cm}

This is a valid concern, yet IVOA is still discussing which is the
best way of managing the availability information of the resources. 
In the meantime, the search registries harvest this information
periodically, so all the listed resources in active publishing registries
are the ones shown. However, we cannot guarantee that each resource is
working correctly.
\vspace{0.5cm}

\quoting{
Since this is a simple survey, add a list like tab 1 with standards and tools
can be usefull.
}
\vspace{0.5cm}

As stated before, we have included these tables in sections 2.5 and 2.6.

\vspace{0.5cm}


\end{document}
